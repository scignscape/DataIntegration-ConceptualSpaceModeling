
\twocolumn[\begin{@twocolumnfalse}
{\begin{minipage}{7.75in}\begin{center}\begin{minipage}{5in}

\begin{abstract}
\noindent{}This chapter will take a quick look at 
certain technological requirements associated 
with \q{precision,} \q{personalized,} or 
\q{patient-centered} medicine.  We will 
emphasize how the goals of precision medicine 
are advanced through bioimaging, and also 
how precision medicine may influence 
trial design.  We use features of 
image-analysis and clinical trials to consider 
our overview of biomedical research methods 
initiated in Chapter 2.  We also discuss 
biomedical text mining as a further 
dimension in research methodology, 
summarizing the CORD-19 corpus on 
Covid-19 related research as a case-study. 
\end{abstract}

\end{minipage}\end{center}\end{minipage}}
\end{@twocolumnfalse}]
\thispagestyle{empty}
\cleardoublepage{}


\addcontentsline{toc}{section}{Chapter 3: Data Mining and Predictive 
Analytics for Cancer and Covid-19}

\addtocounter{page}{-1}


\hoctitle{Chapter 3: Data Mining and 
Predictive \makebox{Analytics} for Cancer and Covid-19}

\section{Introduction}

\p{This chapter will touch on research methodology and 
clinical trial design in light of Patient-Centered paradigms 
and Precision Medicine.  We will focus in particular 
on bioimaging/radiology, on trial architecture 
(particularly \visavis{} Covid-19) and on text and data mining.}

\p{After examining how Precision Medicine 
affects requirements for bioimaging software, 
we will consider patient-centered priorities 
in the context of clinical trials.  The goal 
of incorporating more granular patient-specific 
data within clinical observations in general 
translates over to clinical trials in particular, 
with trial-specific data models needing to 
incorporate patient-centered details 
at several stages, including recruitment
of patients for trials, information management 
while trials are being conducted, and 
subsequent follow-up studies and/or analyses.}

\p{This chapter will conclude by examining 
text and data mining techniques oriented toward 
biomedical publications and data sets, on the 
premise that text and data mining represent a 
computationally sophisticated methodology in 
their own right.  We will focus, as a 
case-study, on \Cnineteen{}, a large corpus 
of open-access Covid-related articles.}

%
\section{Personalized Medicine in the Context of Covid-19}
\p{Within just a few months after Covid-19 became a global pandemic, 
doctors and researchers were developing and experimenting with a 
diverse range of treatments and remedies against SARS-CoV-2.  
The treatments which were in some sense 
tested or adopted include Monoclonal Antibodies, 
remdesivir, dexamethasone and other steroids, anti-inflamatory drugs
(such as tocilizumab and sarilumab), 
convalescent plasma, baricitinib (a rheumatoid arthritis 
drug for which Covid-19 is an off-label use), and 
anti-malarial drugs (specifically chloroquine and 
hydroxychloroquine).\footnote{\i{See} \bhref{https://www.mayoclinic.org/diseases-conditions/coronavirus/expert-answers/coronavirus-drugs/faq-20485627}.}
These treatments, that is to say, were either approved (albeit 
provisionally or \q{for emergency use}) by the United States \FDA{} and other 
government agencies, pending clinical trials, 
or reported anecdotally to be helpful in combating Covid-19.  In either 
case they were relatively widely adopted in clinical settings.  
More rigorous attempts to validate claims of any treatments' 
effectiveness, however, have proven inconclusive (at least 
as of mid-2021) --- certainly 
no pharmaceutical or clinical intervention has demonstrated success 
rates comparable to the first wave of vaccines that were approved 
toward the end of 2020.  In short, every well-studied Covid-19 intervention 
(other than vaccination) appears to be successful in some patients and 
not others.  These mixed results generate questions of their 
own --- what are the biologic mechanisms that 
cause different treatments to play out differently 
in distinct patient populations?}

\p{Of course, it is possible that associations between 
treatments and outcomes --- at least in the SARS-CoV-2 context 
--- is driven mostly by random chance, or that treatments 
are correlated to, rather than causative of, outcomes.  
The Covid-19 mortality rate among hospitalized patients 
appears on average to be less than 25\% (even though  
over four million people worldwide have died from 
the pandemic, as of summer 2021).  Accordingly, many people 
even with serious cases of Covid-19 will recover just 
via statistical distribution.  While it 
is possible that treatments received in 
hospitals account for some of the recovery rate ---  
the mortality rate amongst hospitalized patients has diminished 
over time, which implies that hospitals have become more 
skilled in caring for Covid-19 patients and that treatment 
plans have become more refined --- many 
patients hospitalized for the disease might well have recovered 
anyhow (even without additional pharmaceutical or clinical 
interventions).  
Given this interpretive uncertainty, it has been difficult 
for researchers to definitively show that particular 
treatments do in fact improve recipients' chances of 
recovering from Covid-19.  Nevertheless, both observational evidence and 
clinical trials suggest that a number of Covid-19 treatments 
do have some positive benefits, even if not for all patients.}

\p{This situation then leads to the question of why 
particular treatments benefit some patients more than 
others.  Such considerations are of scientific interest, 
of course --- clarifying how drugs or other interventions 
interact with the SARS-CoV-2 virus enhances our knowledge 
about its infectious mechanisms and the progression of 
Covid-19 --- but more immediately it is of clinical 
interest, because doctors need guidelines on 
when to administer which treatment(s) to which individual 
patients.  Since no Covid-19 remedy (apart from vaccination) clearly 
outperforms others under most circumstances, scientists have 
attempted to study Covid-19 treatments with the goal of 
giving doctors more detailed information to work 
with when making these sorts of clinical decisions.  The overarching 
project of a significant subset of Covid-19 research, 
in short, has been to establish criteria allowing doctors 
to predict which treatments are most likely to succeed 
for individual patients, given details of their 
immunological profile and prior medical history.}

\p{In this sense the goals of Covid-19 research overlap with 
methodologies that have been established in the context 
of other research areas --- notably cancer and AIDS --- 
under the general rubric of \q{Precision Medicine.}  
As a field of research, Precision Medicine is focused on 
the correlation between patient-specific biomedical details 
and the likelihood that particular interventions will have 
positive effects for particular patients.  In contrast to 
conventional clinical trials, which consider patient-specific details 
only at a rather coarse level --- merely observing obvious 
data-points such as age, gender, and ethnicity --- precision-medicine 
research attempts to curate a much more detailed picture of 
patients' medical and sociodemographic profiles, either as 
part of formal trials or as observational details in a clinical 
setting.  Ideally, Precision Medicine is motivated by the 
goal of analyzing patient-profile-to-outcomes correlations 
not only retroactively (detecting statistical patterns in 
prior outcomes) but also prognostically.  In effect, once a 
particular sort of treatment has been identified as 
especially favorable for patients with certain characteristics, 
it is reasonable to project that future patients with similar 
characteristics should be given similar treatments.  
This intuitive and obvious point, however, masks empirical 
and practical difficulties --- in particular, it is not 
obvious how to quantify the notion that current patients 
are \q{similar} to prior ones.  Indeed, one of the 
provisional results of precision-medicine research thus far 
has been to highlight how statistically significant points of 
resemblance between patient profiles do not necessarily line 
up with how we \textit{intuitively} group patients together by 
visible factors such as age, race, or gender.}

\p{In sum, Precision Medicine has been guided by the thesis that 
compiling detailed patient-centered information can advance 
clinical medicine by identifying which treatment options 
are more likely to succeed for individual patients.  In the 
context of Covid-19, both the diversity of legitimate 
clinical interventions and the failure of any one 
treatment to show unambiguous success for a broad spectrum of 
patients point to the potential usefulness of patient-centered 
approaches.  If doctors have detailed information about 
Covid-19 patients' immunological profiles and 
clinical history they can --- at least in theory --- make 
informed decisions about which treatment options have a 
higher probability of success.  Such predictions would 
not be a matter of guesswork; instead, statistical analysis 
of Covid-19 cases in the past, preferably backed by 
Machine Learning, would detect correlations between 
patient data and recommended treatments.  In effect, the 
spectrum of possible Covid-19 interventions (antibodies, 
steroids, convalescent plasma, and so forth) serves as a 
natural classifier, grouping patients into clusters based 
on pre-treatment profiles indicating that one or another 
Covid-19 intervention has, with some probability, a good 
chance of helping the patient.  Researchers have therefore 
been seeking empirical clues for how to classify patients 
into categories based on recommended treatment plans.}

\p{We will examine in this chapter how the 
goals of Precision Medicine have spurred scientists to propose 
more sophisticated data-integration 
and data management tools in the context of clinical trials 
and biomedical laboratory investigations.  Aside from 
opening new avenues for empirical research, the \textit{goals} 
of Precision Medicine have spurred new ways of 
thinking about the computational ecosystem which supports 
biomedical research and clinical practice. 
}

\subsection{Precision Medicine as a Catalyst for Biomedical Data Sharing}

\p{Precision medicine is based on the scientific realization that the effectiveness of a certain clinical intervention --- for instance, 
immunotherapy as a cancer treatment --- is dependent on factors that 
vary significantly between and among patients.  In the case of immunotherapy for cancer, assessing 
the likelihood of favorable outcomes requires genetic, serological, and oncological tests which need to be 
administered prior to the commencement of treatment.  
Therefore, analysis of precision-medicine outcomes requires detailed 
immunoprofiling --- building immunological profiles of each patient before 
(and perhaps during/after) the immunotherapy regimen --- alongside 
an evaluation of how well the patient responds 
to the treatment, with the best outcome being 
cancer remission or non-progression.}

\p{The contemporary emergence of Precision  
Medicine is driven, in part, by advances in diagnostic 
equipment which enable immunological profiles to be 
much more fine-grained than in previous decades.  This 
level of patient-profiling detail enables granular three-way analysis 
involving (1) patients' immunology; (2) treatment 
regimens; and (3) clinical outcomes.  This three-way approach 
has the goal of 
identifying signals in immunological profiles that indicate 
which therapeutic interventions are most likely to engender 
favorable outcomes.  As with any statistical analysis, 
predictive accuracy increases in proportion to the amount 
of data available.  As such, further refinement of 
precision medicine  
depends in part on data aggregation: pooling a number of 
observational studies where patients' immunological profiles 
are described, in detail, alongside reports on treatment plans 
and patient outcomes.}

\p{Within the scientific research community, organizations such as the 
Society for Immunotherapy of Cancer (\SITC{}) and 
the Parker Institute for Cancer Immunotherapy (\PICI{}) 
have developed 
programs and tools to share immunotherapy research 
data, which join older (but less cutting-edge) 
projects, such as Cancer Commons or the \RSNA{} 
(Radiological Society of North America) Image Share program.  
In the Covid-19 population we are similarly 
presented with multidimensional patient data.  
Such information traverses 
molecular testing to identify the virus's genetic material 
or the unique markers of the pathogen itself; antigen testing 
(albeit less accurate) to identify specific proteins found on 
the outer surface of the virus; mapping the genomes of SARS-CoV-2 
to learn how it mutates; analyzing blood samples for the presence/absence 
of antibodies in response to a prior SARS-CoV-2 infection; using 
high-dimensional flow cytometry to perform taxonomic breakdown of 
patients into distinct immunotypes related to disease severity and 
clinical parameters; profiling patients' immunological state prior 
to the start of treatment and quantifying patients' immunological 
response once treatment has begun; calculating plasma viscosity (\PV{}) 
for detection of unusually high levels of fibrinogen --- 
leading to atypical blood clots (that are refractory to 
standard anticoagulant therapy); monitoring 
cognitive, neurological, or cardiovascular  
symptoms in \q{long haulers}; and so forth.}

\p{One area where data-integration is significant for Covid-19 research 
is that of protein biomechanics.  For instance,
\cite{QiongqiongAngelaZhou} identifies  
64 different proteins which are therapeutically 
relevant to Covid-19.  Each of these proteins can be connected to molecular data, 
bioassay information and, in many cases, to clinical 
trials that test therapies where the specific protein acts as a 
target for inhibiting SARS-CoV-2.  
In combination, aggregating all of this information yields a
heterogeneous (but interconnected) 
data space, characterized 
by data derived from multiple scientific disciplines and multiple 
laboratory methods.}


%\section{Overcoming Impediments to Data-Sharing Initiatives in Precision Medicine}
%\subsection{Data-Sharing Solutions for Immunoprofiling/Immunotherapy and Covid-19}

\p{Because biomedical diagnostic and investigative 
laboratories need to use highly specialized software, 
raw laboratory data is often excluded from data-sharing initiatives.  
Instead, the information which is shared between 
hospitals or research centers tends to be a simplified 
overview --- adhering to standards curated by 
groups such as  
\OMOP{} (Observational Medical Outcomes Partnership) 
Common Data Model or \CDISC{} (Clinical Data Interchange Standards Consortium) 
--- skews more toward succinct summaries of diagnoses, treatments, and/or 
outcomes.  Limitations in sharing more granular 
diagnostic or prognostic data have been identified as obstacles diminishing 
the value of data-sharing initiatives.  As one example, in a 
paper discussing research  
into the sequencing of immunoglobulin repertoires (Ig-seq), 
\cite{SimonFriedensohn} comments: 

\vspace*{.5em}
\begin{displayquote}
A major challenge when performing Ig-seq is the production
of accurate and high-quality datasets [because] the conversion of
mRNA ... into antibody sequencing libraries relies on a number of 
reagents and amplification steps ... which potentially introduce errors
and bias [that] could alter quantitation of critical 
repertoire features. ... One way to address
this is by implementing synthetic control standards, 
for which the sequence and abundance is known prior to sequencing, thus
providing a means to assess quality and accuracy.
\end{displayquote}
The underlying 
problem in this context is how different laboratories may use 
different techniques and protocols to achieve similar diagnostic/investigative 
goals (\i{see also} \cite{LauraLopezSantibanezJacome}, 
\cite{MarksDeane}, \cite{MartijnMVanDuijn}, \cite{YanlingWu}, 
etc.).  Consequently, when data is merged from multiple hospitals, it is likely that 
the raw data derives from different labs, which can lead to situations where  
protocol variations across each site can introduce errors
and bias that contaminate the aggregate data-sharing results.  
(We should note that Ig-seq and related \q{repertoire sequencing} 
is quite relevant to coronavirus immunology because 
these techniques help quantify patients' immune response 
to infection, which is consequential both for 
explaining the differences between mild and severe 
cases and observing the effectiveness of interventions 
such as vaccinations or antibody treatments; 
\i{see} \cite{KatjaFink}, \cite{AnastasiaAMinervina}, 
\cite{ChristophSchultheiss}, \cite{NataliaTFreund}, \cite{JacobDGalson}, 
\cite{AleksandraMWalczak}, \cite{CheMaiChang}, 
 \cite{SupriyaRavichandran}, \cite{RishiRGoel}, 
etc.)}

\p{In the context of Covid-19, Shrestha \textit{et al.} 
\cite{GentleSunderShrestha} argue 
for \q{precision-guided studies} to be prioritized \q{[r]ather 
than conducting trials using the conventional trial designs and 
poor patient selection} (page 1).  To accomplish this, the authors 
recommend \q{large multicenter trials} which incorporate 
\q{predictive enrichment strategies ... to 
identify and thus target specific phenotypes [patient-profile 
characteristics], potentially raising the possibility of positive 
trial outcomes.}  The underlying problem identified by Shrestha \textit{et al.} 
is acquiring sufficiently large trial cohorts 
in contexts where trials are to be targeted at 
fine-grained patient populations with specific pre-treatment 
immunological profiles.  This problem can be 
ameliorated by merging prospective patients from 
multiple institutions, such that concurrent trials spanning 
multiple health-care settings can be launched as part 
of a comprehensive approach to comparing treatment options.  
Such large multicenter trials  
can produce \q{large cohorts of patients in a shorter 
time period} while also steering patients toward 
more favorable treatment courses.}

\p{In short, Shrestha \textit{et al.} 
explicitly challenge the conventional wisdom which 
assigns \q{gold standard} status only to \textit{randomized} 
trials, arguing that the benefits of larger trial sizes 
and quicker trial initiations outweigh whatever 
statistical value is compromised by earmarking 
patients into trials based on educated guesses as 
to favorable outcomes (which is warranted by 
patient-care ethics in any case).  The authors also  
argue for a \q{robust data 
infrastructure} which would combine the efforts of clinicians, researchers, 
and data scientists.  In effect, aggressive data-curation would 
substitute for double-blind trial design as a means 
of ensuring the scientific value of trial outcomes.}

\subsection{Software Alignment for Covid Phylogeny Studies}% across Multiple Populations}
\p{The study of SARS-CoV-2 mutations is another area  
where software alignment can prove to be important.   
Analyses in 2020 suggested that variations in the 
viral strain causing Covid-19 symptoms may be partly 
responsible for divergent immunological responses to 
the virus across the patient 
population \cite{OsmanShabir}.  If one 
patient 
responds 
either less favorably or more favorably than the average patient-response 
to a 
given treatment, clinicians need to assess whether this 
difference can be explained solely by the patient's 
prior immunological profile or whether the patient has 
been exposed to a genetically divergent viral strain.
In 2021, of course, predictions about SARS-CoV-2 mutations 
came to pass, with at least four variants appearing to be 
sufficiently dangerous (by virtue of their infectiousness 
and/or lessened effectiveness of treatments or vaccines) 
to undo progress which has been made against Covid-19 
in many parts of the world.}

\p{Comprehensive models of Covid-19 variants require a 
combination of genetic, proteomic, anatomic, and 
epidemiological information, because mutations can 
only take hold if they modify the virus's morphology 
and/or infectiousness in ways that are conducive to 
that strain replicating.  The \b{delta} strain, 
for example, which (as of mid-2021) has been the 
most wide-spread mutation \cite{LizSzabo}, 
carries a genetic variation (designated 
as \q{\b{D614G}}) 
which results in a denser 
array of spike proteins, thereby reinforcing 
the biomechanic pathway which the virus uses 
to infect the host's respiratory system 
\cite{BetteKorber}, \cite{AnwarMohammad}, 
\cite{UtsavPandey}, \cite{LizhouZhang}, 
\cite{DonaldJBenton}, \cite{ErikVolz}, 
\cite{ShiZhao}, etc.}

\p{Modeling SARS-CoV-2 evolution across the globe is a 
massive project.  There have been over 200 million Covid-19 cases 
worldwide (as of mid-2021), in virtually every nation on earth, so a 
complete phylogenetic picture of SARS-CoV-2 in 
humans would need to pool data from many different 
healthcare systems.  Yet, even technically 
detailed analysis of the phylogeny of SARS-CoV-2, such as that 
conducted by Dearlove \textit{et al.} (as 
reported at the end of 2020) 
only considered 27,977 patients (about 0.1\% of 
global cases), with almost half from the 
United Kingdom \cite{BethanyDearlove}.  
Hence, achieving something resembling a holistic picture of 
SARS-CoV-2 mutations and how they might affect clinical 
treatments, would require many parallel studies analogous 
to that of Dearlove \textit{et al.}.}

\p{This then raises questions of study alignment: calculating 
viral phylogeny requires making technical decisions about how genetic 
sequences should be acquired and analyzed, decisions which 
may vary among research teamss.  
For example, Dearlove \textit{et al.} describe several computational 
steps which they had performed 
both to normalize each SARS-CoV-2 genome sequence in their 
data set for cross-comparison and to run predictive 
simulations (used to estimate whether divergence between 
sequence-pairs are the result of localized, random mutations or, 
conversely, an indication that SARS-CoV-2 is evolving into 
further distinct strains).    
Clustering SARS-CoV-2 genomes into variants --- that 
is, identifying which mutations are random and which 
appear to be propagating to subsequent viral generations 
--- involves making computational and biological 
assumptions, such as how to statistically 
marshal genomic data so as to quantify the prevalence of a 
mutation, and how to estimate whether a particular mutation confers 
an adaptive benefit to the viral agent (e.g., an 
ability to elude antibodies targeting structural proteins).}

\p{Given that modeling viral phylogeny requires 
certain computational assumptions and biological 
guesswork, data from multiple studies can only be 
reliably integrated if there is some degree of 
alignment across their methodology.  As such, 
research teams should document their protocols 
in a manner that permits assessment as to whether 
protocol differences might compromise the 
resulting data.  One way to achieve this is to 
model the protocol itself as a data type in a 
general-purpose programming 
language (such as \Cpp{}, for sake of argument).  For each study, such as 
Dearlove \textit{et al.} cited above, there would then 
be a \Cpp{} object encapsulating details of 
the researchers' protocols and computational workflows.  
Protocol-alignment would in this context 
be one part of a common framework to quantify the 
epidemiological significance of SARS-CoV-2 mutations.}

\p{In short, a holistic 
global picture of SARS-CoV-2 must represent 
SARS-CoV-2 mutations which have been deemed 
phylogenically and/or clinically significant (i.e., having 
potential either to influence the overall evolution of 
Covid-19 and/or to have some bearing on clinical treatments), 
and must \textit{also} represent divergent SARS-CoV-2 strains 
carrying those mutations.   
These data-points 
are then be the basis of further details such as: 
when did a given strain and/or mutation first appear?  Is 
the strain/mutation geographically localized?  What is the 
proportion of different strains/mutations in a geographic 
area?  Is there evidence that different strains/mutations 
affect a patient's immunological response to Covid-19 and/or 
the effectiveness of vaccines, antibody regimens, steroids, 
or other clinical interventions?  How can genetic mutations within 
the SARS-CoV-2 virus be correlated with structures in the spike 
proteins encoded by the viral genes?  This last question points 
to the importance of integrating genomic data with \ThreeD{} molecular 
models (\textit{see} \cite{WingerCaspari}, \cite{SandipanChakraborty}, 
\cite{IvanMercurio}, \cite{AliFAlsulami}, \cite{SumanPokhrel}, 
\cite{VictorPadillaSanchez}, \cite{FedaaAli}, for example).  
Whereas data structures modeling the viral genome 
are composed of nucleotides --- and, at a higher scale, Open Read 
Frames (\ORF{}s) --- data structures describing the biophysics of 
glycoproteins involve protein architecture and chemical 
bonds \cite{LiangweiDuan}, \cite{AnshumaliMittal}, 
\cite{XiuyuanOu}, \cite{GiwanSeo}, \cite{HanhTNguyen}, 
etc.  Analyzing how 
\makebox{SARS-CoV-2} genes 
affect the production of glycoproteins, therefore, requires 
annotating and cross-referencing nucleotide/\ORF{} data structures 
with \ThreeD{} molecular models encoded in formats 
such as \MOL{} or Protein Data Bank (\PDB{}) 
\cite{YasunoriWatanabe}, \cite{YongfeiCai}, \cite{AlexandraCWalls}, 
\cite{GennadyMVerkhivker}, \cite{CheolminKim}, etc.}

\p{Object-Oriented models for genomic phylogeny analysis 
have been presented in \cite{JoshuaBSinger} 
(Java) and \cite{GuanghongZuo} (\Cpp{}), although 
these do not provide object models to extend 
from genetic information toward clinical, proteomic, 
or imaging data.  We are not aware of Object-Oriented models 
proposed as representational devices 
for Covid Phylogeny in this more holistic and 
interdisciplinary, but this form of 
software design would be consistent with code 
developed in contexts such as oncology, for 
example the computational simulation of 
tumor growth, which we will discuss in Chapter 4.  
Covid-Phylogeny Object Models 
could serve as a nexus for merging temporal and 
geographical data concerning the epidemiology of \makebox{SARS-CoV-2} 
mutations with genomic data demonstrating  
which mutations are significant, as well as clinical 
data tracking correlations between mutations and treatment outcomes.  
Supporting multi-trial data integration would, therefore, 
also introduce new requirements for clinical trial software, 
which we discuss further next chapter.}

\subsection{Personalized Medicine and Immuno-Profiling}
\p{While some of the data related to immunological profiles 
may be sociodemographic or part of a patient's medical history (fitting 
nicely within conventional Clinical Research 
Network models), contemporary 
immunoprofiling is powered by 
highly specialized diagnostic equipment and methods --- 
which require special-purpose file formats and software.  
For instance, one dimension of immunological profiling is \q{immune repertoire};
the more robust a person's repertoire, the wider variety of antibodies they can produce to fight off pathogens.  Immune repertoire is often measured by studying genetic diversity in B-cells; in recent years, 
this has been done using \q{Next Generation Sequencing} (\NGS{}), which produces files in formats 
such as \FASTQ{}.  Another dimension of immunological profiling is quantifying the proportions of different sorts of blood cells in a patient's blood sample, which is typically done via Flow Cytometry or Mass Cytometry, yielding \FCS{} (Flow Cytometry Standard) files.  The immunological evaluations which are the goal of these methods are sometimes called Cell-Type Classification or \q{Automated Cell-type Discovery and Classification} 
(\ACDC{}).}

\p{As outlined in the below table (Figure~\ref{fig:tabl}), immunological profiles draw on a diversity of data formats and lab/data-acquisition modalities (this table is not intended to be a complete list of criteria or file formats, but rather to indicate the range of diagnostic technologies and data formats which are relevant to immunoprofiling).  This table hopefully indicates how immuno-oncology data sharing 
is inherently more complex when compared to data-sharing 
initiatives which focus primarily on 
clinical outcomes --- such as  
\OMOP{}, \CDISC{}, or the Patient-Centered Outcomes Research Network 
(\PCORnet{}).}


\begin{figure*}
\caption{Table Outlining Several Data Formats and the Contexts where they are Acquired}
{\large
\vspace{1.5em}
\begin{tabular}{ l | r | c }
\underline{Immunological Profile Dimension} \hspace{.5em} & \underline{Lab/Clinical Method} \hspace{.5em}   & \underline{File Format} \\
Sociodemographic                & \hspace{.3em} Scan Patient Records \hspace{.3em} & CSV, SQL \\
Medical History                 & \hspace{.3em} Scan Patient Records \hspace{.3em} & CDISC, OMOP, PCOR \\
Immune Repertoire               & \hspace{.3em} NGS \hspace{.3em}                 & FASTA, FASTQ \\
Cell Type Classification        & \hspace{.3em} Flow/Mass Cytometry \hspace{.3em} & FCS \\
Immune/Tumor Microenvironment         & \hspace{.3em} Confocal Microscopy \hspace{.3em} & \hspace{.4em} DICOM, OME-TIFF, CoCo

\end{tabular} 
\vspace{1.5em}
}
\label{fig:tabl}
\end{figure*}




\p{Formats such as the  \OMOP{} Common Data Model (\CDM{}), the \PCORnet{} \CDM{}, 
or the \CDISC{} specifications promote data sharing primarily through 
\SQL{}-style tables, where 
data analysis and extraction can be achieved via conventional 
\SQL{} queries.  The situation is very different, however, when 
the data that must be exchanged derives from specialized hardware 
and software, which demands special-purpose file formats, parsers, 
and query engines.  This problem-space is accentuated 
when preparing multi-site sharing that can span dozens of 
hospitals, research centers, and/or laboratories.  
Problems of cross-institutional data integration 
will be analyzed further next chapter.}

\p{Prior to that discussion, the following section 
will examine in detail the file formats and data structures 
which were briefly mentioned thus far in this chapter.  
Our goal in this discussion is to document the 
kinds of data which are endemic to different branches 
of biomedical research.  A separate analysis --- one 
which to some extent depends on first describing the data 
formats involved --- concerns how to fuse multiple 
data formats into a common overarching format, such as a 
\q{Common Data Model of Everything.}  One recurring 
theme we will encounter in the subsequent discussion is the 
goal of merging certain data profiles into others 
(e.g., \FCS{} into \DICOM{}), or else adopting 
common interchange formats (such as \XML{}) in lieu of 
domain-specific binary formats which demand 
special-purpose parsers.  We will examine both the strengths 
and weaknesses of proposals for adopting common formats 
rather than idiosyncratic \q{legacy} formats that 
are tied to particular laboratory methodologies for 
historical reasons.  However, our main purpose in the 
current discussion is to establish basic facts about 
data formats in current use.  Later chapters will 
analyze problems connected to the integration of these 
formats into common data models.} 

\p{Note also that the following inventory of biomedical 
data formats is by no means exclusive.  In particular, we 
have largely neglected antigen tests, biochemical assays, 
and many other lab techniques which rely on chemical 
reactions to obtain lab/diagnostic findings.  Our discussion 
here is oriented more toward methodologies which require 
relatively complex intermediate computational processing 
to arrive at clinically useful findings.}





\section{Precision Medicine and Bioimaging}

\p{Patient-centered research in the radiological 
context is centered on improving the precision of 
diagnostic-imaging techniques and corresponding 
clinical interventions.  Indeed, the goal of contemporary 
radiology is not only to confirm diagnoses, 
but also to extract cues from 
medical images that suggest which course of 
treatment has the highest probability of 
favorable outcomes.  A related 
goal is curating collections of diagnostic 
images so as to improve our ability to 
identify such diagnostic and prognostic cues, potentially 
using Machine Learning and/or Artificial 
Intelligence applied to large-scale 
image repositories.}

\p{The goal of building \q{searchable} image repositories 
has inspired projects such as the Semantic Dicom Ontology 
(\SeDI{})\footnote{See \bhref{https://bioportal.bioontology.org/ontologies/SEDI}.} and the \ViSion{} \q{structured 
reporting} system.\footnote{See \bhref{https://epos.myesr.org/esr/viewing/
index.php?module=viewing_poster&task=&pi=155548}.}  
As explained in the context of \SeDI{}: 

\begin{displayquote}
\makebox{[If]} a user has a CT scan, and wants to retrieve the [corresponding] radiation treatment plan ... he has to search for the 
RTSTRUCT object based on the specific CT scan, and from there 
search for the RTPLAN object based on the RTSTRUCT object.  This is an inefficient operation because all RTSTRUCT [and] RTPLAN files for the patient need to be processed to find the correct treatment 
plan. \cite[page 1]{JohanVanSoest}
\end{displayquote}  Even relatively simple queries such as 
\q{display all patients with a
bronchial carcinoma bigger than 50 cm$^3$} cannot 
be processed by \PACS{} systems: \q{although there are various powerful clinical applications to process image data and image data series to create significant clinical analyses, none of these analytic results can be merged with the clinical data of a single patient.}\footnote{See 
\bhref{https://semantic-dicom.com/starting-page/}.}
These 
limitations partly reflect the logistics of how 
information is transferred between clinical institutions 
and radiology labs.  In response, and in an effort 
to advance the science 
of diagnostic image-analysis, organizations such as 
the Radiological Society of North America 
(\RSNA{}) have curated open-access data sets encompassing 
medical images as well as image-annotations (encoding feature 
vectors) that can serve as reference sets and test 
corpora for investigating analytic methods.  Such 
repositories are designed to integrate data from 
multiple hospitals and multiple laboratories --- bypassing 
the conventional data flows wherein radiological 
information is shared between clinicians 
and radiologists, but is not also merged into broad-spectrum 
corpora.}

\p{This renewed focus on patient outcomes 
has intriguing consequences for the scope and 
requirements of diagnostic-imaging software.  
In particular, the domain of radiological applications 
is no longer limited to \PACS{} workstations 
where pathologists perform their diagnostic analysis, 
with the results transferred back to the referring institution 
(and subsequently available only through that institution's 
medical records, if at all).  In the older,  
conventional workflow, radiographic images are requested by 
some medical institution for diagnostic purposes.  
Relevant information is therefore shared between 
two end-points: the institution which prescribes 
a diagnostic evaluation and the radiologist or 
laboratory which analyzes the resulting images.  
Building radiographic data repositories complicates 
this workflow because a third entity becomes 
involved --- the organization responsible 
for aggregating images and analyses is generally 
distinct from both the prescribing institution and 
the radiologists themselves.  As a result, both radiologists and prescribing 
institutions, upon participation in the formation 
of the target repository, must identify which image 
series and which patient data are proper candidates 
for the relevant repository.}

\p{For a concrete example, \RSNA{} has announced the 
forthcoming publication of an open-access image 
repository devoted to 
Covid-19 (specifically the organization established 
two \q{task forces} in 2020; as of mid-2021 the overall 
project appears still to be under development, although 
several recent studies on Covid-19 in general 
have been published in \RSNA{} 
journals).\footnote{See \bhref{https://www.rsna.org/covid-19}.}  
This repository is being curated in collaboration with 
multiple European, Asian, and South American organizations 
so as to collect data from hospitals treating 
Covid-19 patients.  Such a collaboration requires 
protocols both for data submission and for patient privacy 
and security.}

\subsection{The Basic Synthesis Between Bioimaging and Precision Medicine}
\p{\noindent{}Patient-centered research in 
the radiological context has several dimensions, 
including analysis of the rationales for diagnostic 
imaging in the first place: how well does image-based 
diagnostics actually correlate with improved patient 
outcomes?  How can we quantify the experience of 
image-based testing itself (e.g., to identify 
factors such as cost, discomfort, and 
radiation danger which can rank some tests 
as less desirable than others, as one parameter 
to consider when deciding whether to prescribe 
imaging, and which modality)?  These were 
among the questions addressed by 
the Patient-Centered Outcomes Research Institute's 
(\PCORI{}'s) \q{Patient-Centered Research for Standards 
of Outcomes in Diagnostic Tests} (\PROD{}) study 
\cite{ZigmanSuchsland}, 
which also established a useful protocol for how 
medical institutions could provide reports 
on the experience and effectiveness of imaging 
from a patient's as well as clinician's 
perspective.}

\p{Assuming diagnostic imaging of a given 
modality \textit{is} 
appropriate, an additional priority should then 
be to structure the diagnostic workflow --- the 
image procurement, analysis, and data/metadata sharing 
protocols --- to maximize the probability 
that subsequent clinical interventions are 
chosen which promote favorable outcomes on 
multiple patient-centered criteria, including 
quality of life and patient engagement.  In 
the best case scenario, the goal of radiology is 
not only to confirm a diagnosis, 
but also to extract cues from 
medical images that suggest which course of 
treatment has the highest probability of 
favorable treatment outcomes.}

\p{While most radiologists and pathologists 
would probably agree with this assessment 
--- that in the best-case scenario 
image-processing can yield predictive 
analytics which would sort patient populations 
into groups steered toward distinct treatments 
deemed most likely to be effective --- 
there are technological and operational 
challenges to making patient-centered 
perspectives a central feature of diagnostic-imaging 
methodology.  Effectively cross-referencing 
imaging and outcomes data requires integrating 
heterogeneous information obtained at different 
times and places.  Some clinical data is associated 
with each patient at the time that a radiological 
(or other imaging) study is prescribed.  The 
images themselves, and subsequent diagnostic  
reports, then provide layers of data that exist prior to the initiation 
of a course of treatment.  Moreover, a rigorous 
data-integration protocol would need to incorporate 
information emerging \textit{after} the treatment 
starts: descriptions and assessments of clinical outcomes, 
and, potentially, new data garnered by applying 
different image-analysis techniques.  
In brief, data-management protocols must be 
in effect before, during, and after image-analysis itself.}

\p{Ideally, image analysis can be powerful enough 
not only to identify a pathology, but to 
classify diagnoses into clusters based on 
recommended course of treatment.  In order 
for analytic techniques to achieve this 
level of detail, however, it is necessary 
to arrive at a feedback loop where known clinical 
outcomes are associated with prior images, 
so that developers have those outcomes available 
as a further dimension of clinical data 
that may be statistically cross-referenced 
with image analysis.}


\p{The \PROD{} 
study demonstrates that experiential 
factors should be evaluated --- both in 
terms of testing itself (and its risks/costs) 
and in terms of post-diagnostic quality of 
life --- as part of the data modeling 
treatment outcomes, and the comparative 
effectiveness of a selected course of treatments 
compared to alternative diagnostic 
methods and/or clinical interventions.  
From the perspective of standard 
data models, initiatives to cross-reference 
imaging and outcomes data include several 
Semantic Web ontologies, such as the Semantic \DICOM{} Ontology 
(\SeDI{}) and the \ViSion{} \q{structured 
reporting} system (both referenced earlier). 
The purpose of these ontologies is to 
standardize the terms through which 
radiographic procedures, analyses, and 
recommendations are described --- more 
precisely or predictably than older 
technologies such as \DICOM{} headers, 
\DICOMRT{}, and the \RadLex{} lexicon.  
By properly aligning image metadata 
spanning multiple patients, it is possible 
to create \q{searchable} image archives such 
that images can be selected or classified 
within a larger image collection, 
yielding image series or patient cohorts 
that can be studied through the lens of 
predictive modeling or patient-centered 
outcomes.  Projects such as \SeDI{} 
implement \q{semantic} \PACS{} workstations 
where the space of known images is defined 
by a particular \PACS{} system, but analogous 
techniques could be used to construct 
larger-scale image corpora as well, for 
research purposes, data mining, or as 
test-beds for code and algorithms.  
Patient-centered data points, such as 
those formulated via \PROD{}, may then 
be incorporated as supplemental 
data.}  

\subsection{Case Study: The Cancer Imaging Phenomics Toolkit}
\p{At the forefront of diagnostic imaging 
are new analytic techniques which statistically 
analyze image data, often detecting signals that 
are invisible to the naked eye, or at 
least processing groups of image-series on a scale 
beyond the reach of human radiologists.  The proliferation of 
image-analysis methodologies places a new emphasis 
on \textit{extensibility,} in which the 
capabilities of bioimaging software can be 
expanded in a decentralized, modular fashion.}

\p{A representative example of 
this new paradigm is \CaPTk{} (Cancer Imaging  
Phenomics Toolkit), which provides 
a primary component supplying a centralized User Interface and 
taking responsibility for acquiring and loading 
radiographic/microscopy images, paired with multiple \q{peer} 
applications which can be launched from \CaPTk{}'s main window, 
or alternatively used as standalone programs \cite[especially 
section 4]{ChristosDavatzikos}.  
In particular, \CaPTk{} provides an implementation 
(apparently the only \Cpp{}-based implementation) 
of the Common Workflow Language (\CWL{}), 
using this workflow model in 
conjunction with the \Qt{} Reflective Programming 
system to implement workflows connecting the 
central \CaPTk{} application with its analytic 
extensions.\footnote{No native \sC{} or \sCpp{} 
libraries are described on the \sCWL{} website 
among the tools and parsers available for \sCWL{}, 
but \sCaPTk{} is mentioned on a corresponding discussion 
thread concerning \sCpp{} libraries.  It seems 
therefore that the \sCaPTk{} \q{utilities} repository 
provides the de-facto standard \sCpp{} implementation of 
\sCWL{}, at least according to the \sCWL{} group 
themselves.}   In effect, \CaPTk{} achieves a 
workflow and messaging protocol for what they 
term \q{native,} \q{standalone} applications, 
yielding an extensible architecture through 
which new image-analysis techniques can be 
integrated into an underlying \PACS{} system
(a detailed discussion of \CaPTk{} module 
implementation 
is outside the scope of this chapter, but 
this book's supplemental materials include a 
more technical overview of \CaPTk{}; for now 
we just use \CaPTk{} as a case-study 
in modular design for bioimaging). }

\p{Taking the \RSNA{} Covid-19 repository as a case study for 
promoting research into post-diagnostic outcomes, 
this repository is possible because an international 
team of hospitals and institutions have agreed to 
pool radiological data relevant to SARS-CoV-2 infection 
according to a common protocol.  Taking \CaPTk{} as a 
case-study in multi-modal image analysis, this 
system is likewise possible because analytic modules 
can be wrapped into a plugin mechanism which allows 
many different algorithms to be bundled into a common 
software platform.  Of course, these two areas 
of data-sharing overlap: one mission of repositories 
such as the \RSNA{}'s is to permit many different 
analyses to be performed on the common image 
assets.  The results of these analyses then become 
additional information which enlarges the 
repository proportionately.}

\p{The \CaPTk{} project serve as a useful 
case-study for the analytic convergence 
or cross-referencing between image-analysis 
and outcomes/patient-centered data 
because is extensible as part of 
its essential design (although in terms of large-scale adoption 
more conventional \PACS{} clients may also 
be used simply because \CaPTk{} has certain 
software-engineering innovations which 
make it an outlier from an 
implementational point of view).  Analysis 
of the aforementioned \RSNA{} Covid-19 images, 
for example, could be enacted via 
specialized \CaPTk{} modules, which 
could in turn be provisioned with 
Patient Outcomes and Comparative Effectiveness Research (\CER{}) 
capabilities.  That is, modular design 
at the image-processing level can be 
leveraged to incorporate \CER{} and 
predictive-analytic information sources 
(such as clinical records and 
immunological profiles) alongside 
image data proper (such as feature vectors 
calculated via Computer Vision algorithms).}

\p{A further level of 
integration between \CaPTk{} and \CER{} initiatives 
(again, staying with \CaPTk{} as a 
representative example of bioimaging applications in general) 
can be achieved if one observes that clinical 
outcomes may be part of the analytic parameters 
used by \CaPTk{} modules.  As presently constituted, 
\CaPTk{} analytic tools are focused on extracting 
quantitative (or quantifiable) features from 
image themselves, without considering additional 
patient-centered context.  There is no technical 
limitation, however, which would prevent the 
\CaPTk{} system from sharing more detailed clinical 
information with its modules, allowing these 
analytic components to cross-reference image features 
with clinical or patient information.  This 
then raises general questions about sharing 
clinical data \i{as well as} information 
derived from bioimage analysis, bringing 
us to more general themes in 
lab/clinical data-sharing.}

\subsection{Multi-Application Networks in the Context of 
Scientific Research Data}
\p{Architecturally, the pattern of organization 
just described --- semi-autonomous applications 
linked together (often by virtue of being common 
extensions to an overarching \q{core} software 
platform) --- is 
analogous to the collection of software components 
that may share access to a data repository or 
a research-data corpus, include a corpus of 
medical/diagnostic images.  The purpose of 
research data archives --- particularly when 
they embrace contemporary open-access standards 
such as \FAIR{} (Findable, Accessible, Interoperable, 
Reusable) \cite{TrifanOliveira}  
and the Research Object 
Protocol\footnote{see \bhref{http://www.researchobject.org/scopes/}} --- is to promote reuse and reproduction of 
published data and findings, such that multiple subsequent research 
projects could be based on data originally published 
to accompany one book or article.  As a result, it is 
expected that numerous projects may overlap in their 
use of a common underlying data set, which potentially 
means a diversity of software components implementing 
a diversity of analytic techniques, each offering a unique 
perspective on the underlying data.}


\p{Implementing a robust research-data software 
framework involves integrating multiple scientific 
applications, but also (ideally) extending these 
applications with features specifically 
of interest to those conducting or reviewing 
research using published data sets and/or 
described in academic literature: for 
instance, capabilities to download data sets 
from open-access scientific portals; to 
parse microcitation formats; and to interoperate 
with document viewers.  This review of data-publishing technology is 
relevant to radiology and to Patient-Centered 
Outcomes because it typifies the emerging ecosystem 
where scientific research and open-access data 
is being disseminated.  
The architecture employed by \CaPTk{} 
is a useful example of how multiple autonomous, 
stand-alone, native 
applications can be federated into a decentralized 
but unified platform, logistically embodying the kinds of application networks 
appropriate for the technology supporting 
archives of research data (including 
diagnostic-imaging repositories).}

\p{Initiatives such as Research Objects and 
\FAIR{} advocate for a technological infrastructure 
characterized by a diverse software ecosystem 
paired with open-access research data sets \cite{KhalidBelhajjame}.
Although formats such as Research Objects have been 
standardized over the last decade, there has not 
been a comparable level of attention given to 
formalizing how multiple software applications 
should interoperate when manipulating 
overlapping data.  The Common Workflow Language 
(\CWL{}), which has been explicitly included in the 
Research Object 
model,
documents one layer of inter-application messaging, 
including the encoding of parameters via command-line 
arguments (as mentioned earlier, 
\CaPTk{} provides the most complete \Cpp{} implementation of 
\CWL{}, using it to pass initial data between 
modules).  Serializing larger-scale data structures 
is of course a generic task of canonical encoding 
formats such as \JSON{}, \XML{}, \RDF{}, and 
Protocol Buffers --- not to mention text or binary 
resources serialized directly from runtime objects 
via, for instance, \textbf{QTextStream} and \textbf{QDataStream.}  
This means that some level of inter-application 
communications is enabled via \CWL{}, and 
that essentially any computationally tractable 
data structure can be encoded via formats such 
as \XML{}.  These solutions, however, are 
sub-optimal: \XML{} (as well as \JSON{} and analogous 
formats) is limited because it takes additional 
development effort to compose the code that marshals 
data between runtime and serial formats.  
Similarly, although \CWL{} can model information 
passed between applications, it provides 
only an indirect guide for programmers implementing 
each application's \q{operational semantics} --- 
viz., the procedures which must be executed 
before and after the event wherein data is 
actually passed between endpoints.}

\p{In the context of \CaPTk{}, for example, 
integrating peer modules with the \CaPTk{} core 
application involves more than simply 
ensuring that these endpoints communicate 
via a standardized data-serialization format: 
the plugins must be \textit{registered} 
with the core application, which affects the 
core in several areas, including the build/compile 
process and construction of the main \GUI{} 
window.  Modeling the interconnections between 
semi-autonomous modules, as \CaPTk{} demonstrates, 
therefore requires more detail than simply 
modeling their shared data encodings; it 
is furthermore necessary to represent all 
procedural and User Interface requirements 
in each component that may be affected by 
the others.  Despite the standardization 
efforts that have been invested in 
Research Objects and the Common Workflow Language, 
we contend that this fully detailed 
protocol for multi-application interop has not 
yet been rigorously formalized.}


\p{Rigorous models of application networks among semi-autonomous 
components acquire an extra level of complexity precisely 
because of this intermediate status: protocol definitions 
have to specify both the functional interdependence 
and the operational autonomy of different parts of 
the application network.  Although 
one application does not need detailed 
knowledge of the other's internal procedure 
signatures (which would break encapsulation), the 
functional interdependence between applications can 
accordingly be modeled by defining protocols which 
must be satisfied by procedure-sets internal 
to each end-point --- the relevant information 
from an integrative standpoint is not the 
actual procedures involved, but confirmation 
that the relevant procedure sets adhere 
to the relevant multi-procedural protocol.\footnote{Reviewing the source code and documentation 
for \CaPTk{} confirms that multi-application 
messaging along these lines is implicitly adopted 
by \CaPTk{}; \i{see} for example 
\bhref{https://www.med.upenn.edu/cbica/assets/user-content/images/captk/2018\_ISBI\_CaPTk .0404.Part2.pdf}, particularly 
the material starting on the 30th slide of that presentation.}} 

\p{While we have initially approached multi-application networking 
from the bioimaging perspective, this topic is equally 
applicable to multi-site trial architecture, the theme of next section.}






\section{Precision Medicine in Trial Design}
\p{\noindent{}Converting basic research to 
clinical practice directly benefiting patients
--- sometimes called \i{translational informatics},
a \q{research cycle, which involves the translation of knowledge
and evidence [to] provision of evidence-based care
in the clinical or public health settings}   
\cite[page 2]{PhilipROPayne} --- is sometimes 
represented as a two-stage process which 
first involves translating research to 
clinical trials, and then formulating 
point-of-care practices on the basis of trial 
results (\i{ibid.}).  Data-sharing 
initiatives need to pay particular attention to 
the logistics of translational informatics in 
contexts where granular patient-specific information is important, 
such as immunoprofiling.  Questions which should be addressed include 
(1) where data is to be hosted; (2) how participating 
institutions should submit data to a central repository; 
(3) how participating institutions and/or outside investigators 
should access previously-deposited data; (4) how to 
ensure anonymization of patient-specific records; (5) how to ensure 
that different labs used by different hospitals are utilizing 
compatible protocols, so that results from multiple labs/hospitals 
can be seamlessly merged in a shared data commons; 
(6) how to ensure proper alignment between software employed 
at different institutions; and (7) how to incorporate data 
curated within the context of a multi-institutional data-sharing initiative 
into scientific papers documenting 
research findings.  Each of these areas of concern 
are technically demanding because of the complex and 
heterogeneous nature of immunological profiles.}

\p{As a case-study in clinical-trial software 
engineering, consider again the proposals 
in Shrestha \textit{et al.} \cite{GentleSunderShrestha} 
which we reviewed last chapter.  As these authors 
recommend, Covid-19 trials should be designed to focus on 
specific patient groups which are more likely to benefit from the interventions 
that form the basis of the relevant clinical trials.  Moreover, 
toward the goal of applying precision medicine to Covid-19 clinical practice, 
it should be possible to construct a quantifying domain of patient-profile signals
(antecedent to trial commencement) 
to quantify the statistical probability that a given treatment will have a 
favorable patient outcome in relation to all the prior data in a patient's 
profile.  Since researchers assume that 
certain factors in a patient's profile 
will be statistically correlated with favorable outcomes in conjunction 
with specific treatment plans, part of the trial's purpose 
is to determine which parts of the patient profile 
are, in fact, statistically relevant.}

\p{In practical terms, then, setting up Covid-19 trials would involve 
defining patient-selection criteria and implementing systems to 
screen for patients who may be good candidates for different trials.  
This would require two steps: (1) constructing a format where trial criteria 
can be rigorously notated; and (2) implementing software at 
participating sites to search for trial candidates.  
This software would need to access, represent, and analyze 
patient-specific trial-eligibility factors that covering a 
broad spectrum of data types (sociodemographics,  
medical history, lab/image results, etc.).  
Proposing an automated recruitment 
engine for cancer trials, \cite{SaytaSSahoo} argued (in 2014) that

\begin{displayquote}
Many ... tools have been developed for accessing
the institutional data warehouse to screen patients for clinical 
trials or for creating an alert system for physicians ... 
However, ... existing systems have limited access to the complete patient information, such as the latest laboratory test results, and are not integrated with the clinical systems used in routine patient care.  Further, existing tools have limited support for structured entry of trial information, interactive user interfaces (UIs) that allow clinicians to review the matching results and re-execute the matching process with updates to patient records.  [Although] there has been extensive research in creating formal, computable representation of eligibility trial specifications that can be used together with electronic representation of patient data in EHR systems [an] important challenge for computational representation of trial information is
the lack of suitable interfaces for entering eligibility criteria. (page 2)
\end{displayquote}  
\noindent{}Ten years earlier, \cite{JohnSMcIlwain} advocated 
for trial models \q{focused on interoperability, using modern object-oriented techniques} (page 3):  
\begin{displayquote}
Object-oriented software ... is
more reliable because fewer software code changes are required as one's needs evolve; and, therefore, more 
changes can be made without effectively changing the core structure of the 
application ... The difference between object-oriented software and traditional software is substantial. (page 6)
\end{displayquote} 
\noindent{}However, he concludes, \q{Very few CTMS products are object-oriented.}  
Both \cite{SaytaSSahoo} and \cite{JohnSMcIlwain}, we should 
note, are not neutral observers, but writing to highlight 
features of software their own teams designed.  Whatever 
the merits of their systems, these design principles have not 
apparently been incorporated into mainstream \CTMS{} 
in intervening years.}

\p{Given the sheer scale of the SARS-CoV-2 pandemic, 
there are likely to be many candidates for almost any Covid-19 
trial.  However, methods for recruiting patients would 
need to be aligned across multiple institutions, at 
least in the case of multi-site trials.  For example, in the 
context of antigen tests (measuring virus antibody levels) 
the US Centers for Disease Control recommends 
or has authorized a large list of 
assays performed by many different companies, using 
many different biochemical 
methods.\footnote{\textit{See}, for example, 
\bhref{https://www.fda.gov/medical-devices/coronavirus-disease-2019-covid-19-emergency-use-authorizations-medical-devices/vitro-diagnostics-euas\#individual-antigen}.}
Because the data format resulting from immunoassays depends 
on the specific biochemical mechanism which (within each 
assay) yields quantitative data, a broad spectrum of 
antigen tests requires a diverse array of data formats 
which need to be integrated.  As such, whenever 
Covid-19 immunoassays are considered 
as factors in immunological profiles for mapping 
patients to appropriate Covid-19 clinical trials, querying for good 
trial candidates means querying across a 
wide spectrum of structurally different data types that 
correspond to this broad array of antigen tests --- specifically 
to the mechanisms through which laboratory instruments generate 
quantifiable data and to the computational procedures which process 
such data.  Here is an example of why specialized trial software 
can be warranted: the heterogeneity of trial data, 
may call for 
integrative procedures implemented directly 
in programming languages such as \Cpp{} 
(rather than query languages such as \SQL{}).  
The complexity of trial criteria was a motive 
for the software-based \CTMS{} systems we cited 
earlier; Covid-19 serves as a trenchant case-study 
supporting recommendations along those lines 
(\i{see also}, say, \cite{LiMinLiu}, 
\cite{CrengutaBogdan}, \cite{MichaelSouillard}, 
\cite{CanhamOhmann}, 
\cite[especially pages 20-29]{KrisVerlaenen}).}


\subsection{Customizing Clinical Trial Management Software}
\p{Once trials embrace heterogeneous data models
(which require special-purpose software for accessing 
some of the trial data),  Clinical Trial Management Systems 
(\CTMS{}) requirements become 
more complex.  In these situations, \CTMS{}s may need 
to model and in some cases replicate complex computational 
workflows, such as those employed by Dearlove \textit{et al.} 
for calculating SARS-CoV-2 genomic sequences from patients' blood samples.  
The \CTMS{} software may also need to interoperate 
with domain-specific applications, as in bioimaging and 
image analyses, signal processing (e.g., for \EKG{} analysis), 
Flow Cytometry, biochemical assays, genomic analysis, 
epidemiological modeling and so forth.  If possible, such 
applications should be configured or extended to work 
with the clinical trial software.  For instance, if a \DICOM{} 
(Digital Imaging and Communications in Medicine) client 
is used to study an image derived from a specific trial 
--- e.g., a radiological scan of a Covid-19 patient's lungs 
--- the \DICOM{} software could be provided with a plugin 
that would show trial information in a separate window, 
which could then be juxtaposed with the main-image 
view.    
In this context, software alignment means that 
all institutions participating in a trial could use 
the \textit{same} plugins, so that the trial's central 
\CTMS{} system could interoperate with special-purpose 
software in a consistent manner.  This would also 
aid in establishing, as part of the trial design, 
protocols for depositing special-purpose data assets  
(such as \FCS{} or \DICOM{} files) alongside 
clinical data and \eCRF{}s.}

\p{A further benefit of \CTMS{} customization is that 
custom software adds flexibility for trial design.  
By definition, trials allow researchers to test 
biomedical hypothesis in a controlled manner.  
Trials are, therefore, defined around the premise that 
observational information resulting from the trial 
is empirically significant, revealing something new 
about what the trial was designed to investigate.  
For instance, a Covid-19 trial might assess how 
well patients with varying prior immunological 
profiles respond to monoclonal antibody (\mAb{}) treatments.  
The relevant observations in this case derive from 
the subsequent course of the disease for each 
patient, as well as potential adverse reactions, but there are  
inherent complicating factors: were patients receiving other 
treatments as well?  For patients who recover, how do 
we know that the antibodies expedited that recovery? 
How quickly was the recovery?  And, did patients continue 
to suffer from Covid-related symptoms even when they 
were no longer infectious?  Situational details specific to the trial ---  
such as each patient's \mAb{} dosage level, 
prior Covid-19 risk factors, or viral-load change 
over time --- also belong in the trial's unique 
data models.  Moreover, a comprehensive investigation could well incorporate both 
information about the patient's unique immunological profiles and the 
nature of the SARS-CoV-2 variant/strain found in the patient.}

\p{In addition, customizing trial-management software 
has the added potential benefit of greater 
flexibility for incorporating personalized/patient-centered 
data into the overall trial results.  Since patients'  
reactions to interventions are difficult to anticipate in 
full specificity ahead of time (especially when subjective 
experience is taken into account) allowing data 
models to evolve during the course of a trial 
can help trial designs respect the experiential 
dimensions intrinsic to patient-centered 
paradigms of care.}

\subsubsection{Toward Fine-Grained Sociodemographic Models}
\p{Patient-centric data could likewise include sociodemographic information 
about the patient, supplemented by epidemiological 
metrics, such as contact tracing.  Designers need to identify, 
for example, what dimensions of patients' immunological 
and sociodemographic profiles are likely to be 
consequential when analyzing treatment outcomes 
(indeed, biomedical research has been criticized 
in recent years for bias
toward certain populations, e.g., white, middle-class non-seniors).  
This results in
uncertainties as to how well trial results carry over to populations at large --- 
that is, populations characterized by a heterogeneous mix of demographic factors. 
Researchers can mitigate these concerns by demonstrating 
sociodemographic diversity among trial participants. 
Those goals, along with more precise predictive 
analytics, could be advanced by adopting 
more detailed sociodemographic reporting 
standards \cite{AaronMOrkin}, \cite{MarittKirst}, 
\cite{ShelbyMeyer}, etc.}

\p{Demonstrating sociodemographic 
diversity, however, 
calls for transparency about how sociodemographic 
details are represented.  The process of grouping 
patients into ethnic/racial and/or socioeconomic 
strata can be equivocal at times.  For example, if a trial 
participant is a graduate student at the University of 
Chicago, should their socioeconomic status be assessed 
on the basis of their own income or that of their parents?  
If their zip code places them on 
the South Side of Chicago, a region with both a prestigious 
campus and pockets of extreme poverty, should they be 
demographically classified alongside residents 
of that neighborhood?  What about a varsity linebacker 
who appears to be in excellent health before a Covid-19 infection?  
Should his status as an athlete be taken to indicate 
being extremely fit prior to the disease, or might 
his background as a football player intimate a potential history of 
brain trauma which may compound his neurological damage due to 
Covid-19?}

\p{In short, sociodemographic data can
be notoriously imprecise. It is well-known that patients of lower socioeconomic status have higher Covid-19 infection rates and mortality rates than 
patients of higher socioeconomic status,
but this disparity may
be explained by several causal factors: more infectious workplaces, inferior post-infection treatment, poorer state of health before the onset of Covid, and so forth.  
Teasing apart these factors demands fine-grained
analysis of the individual patient's 
pre-Covid history; sociodemographic generalizations can exclude
important details. For example, for purposes of analysis, in the case of a graduate student with
middle-class parents but no health insurance of their own, how should we quantify their degree
of access to health care? How well does geographic location serve as a proxy for socioeconomic
status?}

\p{The case of a middle-class student living in a well-off near-campus corner of an otherwise
impoverished neighborhood suggests that geospatial metrics (such as zip code or congressional
district) are imperfect proxies for wealth; but other factors --- such as air/water pollution or the
risk of being the victim of a crime --- may be statistically correlated among geographically proximate
residents even if they have otherwise divergent sociological profiles.
These examples illustrate that the value of sociodemographic data is proportionate to the level
of statistical detail with which the data may be analyzed.  Instead of a broad and vague designation, such as \q{low income,} one may want 
to derive more detailed subclasses, incorporating
information about patients' employment, access to health care, physical fitness, and so forth. Two
patients with similar income levels, for instance, may have different 
levels of access to health care 
(depending
on factors such as whether the patients have employer-based insurance or geographic proximity
to healthcare facilities) or different levels of exposure to 
Covid-19 in the workplace, depending on the nature of their job.
Even if a trial does not quantify 
granular sociodemographic assessments --- such as patients' 
workplace conditions and access to health care, which might 
serve as a more accurate  
estimation of how socioeconomic status causally affects 
disease outcomes --- subsequent researchers may determine 
ways to analyze or add on to data generated by a trial (e.g., 
through follow-up studies of enrolled patients), 
so as to make such sociodemographic 
granularity part of the quantifiable framework.}

\subsubsection{Measuring Cognitive and Neurological Effects}  
\p{Similar interpretive issues of interpretation also apply to post-treatment 
observations.  How should researchers decide which observations 
qualify as clinically significant 
consequences of Covid-19?  
We have seen that as the pandemic has unfolded, 
a fair number of cases have been cited in the professional literature 
describing Covid-19 patients who suffer certain
cognitive/neurological effects, such as muscular fatigue or weakness, 
mental confusion or poor concentration (sometimes referred to as \q{brain fog}), 
or symptoms of Guillain–Barr\'{e} syndrome spectrum. 
Almost certainly, some 
of these symptoms may be the product of cognitive/neurological effects due 
to SARS-CoV-2.  
At the same time, Covid-19 patients --- 
even those who fight off the infection 
successfully or who test positive but remain asymptomatic
--- may find their lives so disrupted by the pandemic 
that this may (indirectly) cause cognitive and neurological problems.
For instance, prolonged inactivity 
(for a typically active person), which commonly occurs as the result of  
a quarantine, may contribute to poor concentration and other 
diminished forms of mental acuity \cite[say]{SoniaDifrancesco}.  
Given the fact that most 
people's lives during the pandemic are not \q{normal,} 
it may be difficult to establish 
which symptoms experienced by a patient are actually biological effects of
the disease itself or, alternatively, 
indirect consequences of  
lifestyle restrictions.  
This sort of ambiguity also applies to potential adverse 
side-effects of Covid-19 treatment.  
Rigorous design practice suggests, for instance, 
that parameters should be established framing the post-treatment 
and post-recovery window of time where patients' symptoms might be noted as 
potential effects of the disease or of the administered therapies 
themselves.  How long after
a treatment is administered should a patient's 
symptoms be considered potential side-effects of
the therapy itself or, alternatively, the result of 
nagging uncertainties and a 
disrupted lifestyle imposed by the pandemic?}

\p{The fact that these questions have no predetermined answers indicates 
that trial designs need to anticipate a shifting clinical and 
information landscape ad the trial evolves.  
For example, because SARS-CoV-2 was initially believed to  
affect lung functioning primarily, the risk of long-term 
cognitive/neurological damage was not widely anticipated when 
considering treatment over the course of Covid-19 infection.  
Consequently, because tests of a cognitive/neurological/physiological/radiographic 
(except for basic lung scans, with respect to radiology) 
had not been a common facet of early Covid-19 trials/observational 
studies, there were no 
corresponding data structures included in those studies for capturing this full spectrum of neuropsychological and neurological data.}

\p{Systematically tracking Covid-19's 
cognitive/neurological deficits entails neurological, laboratory, neuropsychological and radiographic tests to understand the full extent of their cognitive and neurological impairments. For example, the use of
\MRI{}s when considering Covid-related ischemic  
stroke \cite{RossWPaterson}, \cite{FifiMocco}, 
the use of electrophysiological tests, cerebrospinal 
fluid tests (\CSF{}), or the \MRC{} 
(Medical Research Council) Scale for muscle-strength test when considering 
Covid-related 
Guillain-Barr\'{e} symptoms \cite{SamirAbuRumeileh}, 
or the use of neuropsychological tests, such as Trail Making Test 
(\TMT{}), Sign Coding Test (\SCT{}), 
Continuous Performance Test (\CPT{}), and Digital Span Test 
(\DST{}) --- which measure a patient's executive abilities (letter and number recognition mental flexibility, visual scanning, and motor function) and sustained and selective attention, along with other cognitive and neurological functioning --- when considering 
cognitive/neuropsychological impairments after a serious bout with 
Covid-19 \cite{HetongZhou}.}

\p{These cognitive, neurological, physiological, laboratory, 
and radiographic data 
structures thereby become 
an integral part of the information relevant to trial evaluation, 
because they document symptoms 
which are presumptively attributable to Covid-19.  
However, in practice, prior to clinicians having been alerted to 
the fact that Covid-19 
may cause lingering cognitive/neuro\-logical damage, %physiological or neurological data 
Covid-19 trials were not designed to 
incorporate neurological or cognitive 
data in a systematic manner.  This scenario points to how 
trial data models can benefit from 
a built-in capacity to be redesigned while the 
trial is ongoing, so as to accommodate new and emerging 
information.  On this basis it is reasonable to 
adopt for clinical trials malleable information models such as 
graph databases and/or Object-Oriented software 
components.}

\subsubsection{Aggregating Trial Data via Graph Models}
\p{A useful data-integration case-study is the University of Pennsylvania's Carnival project (which achieves 
data integration by adopting property-graph databases, illustrating 
the flexibility of graph models in a way that we can 
also apply to trial design).  Carnival synthesizes heterogeneous biomedical data  
by translating information from disparate sources into a common 
property-graph representation and then querying this 
data with the Gremlin Virtual Machine.  Gremlin is a 
\q{step-based} virtual machine where \q{steps} between 
potential focus elements in a property graph play the role 
of primitive processing instructions; querying and traversing 
property graphs involves executing a series of Gremlin 
steps.
%\footnote{The theoretical foundations of step-based Virtual Machines 
%are presented in Marko Rodriguez, 
%``Stream Ring Theory,'' February 14, 2019 (\bhref{https://zenodo.org/record/2565243#.X3vzqS4pDeQ}).} 
Most Gremlin implementations are based on the Java programming language and 
the Java Virtual Machine 
(\JVM{}), so that queries 
themselves are written in a \JVM{} language (Groovy, in the 
case of Carnival).  The challenge 
for any database engine which employs a relatively complex data-representation 
strategy --- such as a hypergraph, property-graph, tuple-store (a 
collection of records with varying numbers of fields) or a multi-dimensional 
(possibly sparse) array --- is to efficiently map 
the high-level data structures manipulated within the database itself 
to the lower-level memory units which are stored to disk.  Lower-level 
data structures are typically modeled via simpler database constructions 
such as key-value stores, memory cells, or relational tables, 
so there must be a translation pipeline between high-level 
structures (properties, hypernodes, hyperedges, and so forth) and low-level   
points (record cells, shared memory address, elements in key-value pairs, etc.)}

\p{Consider how patient profiles usually notate the 
medications each patient is taking --- any patient (a node) could in principle be 
connected to any medication (another node); some patients 
may be taking \textit{no} medications, others may take just \textit{one}, and some may  
take \textit{two or more}.  Also, connections between patients and medications 
can be the basis of further details that emerge over time 
(and are registered in the graph) inasmuch as medications are prescribed 
to patients by a specific doctor at a specific time, in a specific 
dosage, in response to specific diagnostic tests, and so forth.  
In short, information can \q{fan out} from the patient-to-medication 
connection in a relatively free-form manner.  In general, then, as a 
subset of overall patient profiles, information about 
medication evinces the structural features which are, in many 
contexts, optimally represented via free-form labeled graphs.    
Meanwhile, patient profiles may also 
consider medical history, which can be modeled as a graph with detailed 
logical and temporal inter-node connections.  According to this 
representational strategy, each patient's history is a series of 
events and observations which are temporally ordered --- it is 
possible to query or traverse the graph in a manner which takes 
before/after relations 
into account --- and where there are also logical or causal connections 
defined between nodes.  For instance, an edge might assert that 
a given medication was prescribed to a patient (an event) \textit{because of} 
the results of a given lab test (an observation).  In these examples, different sorts of clinical data --- sociodemographic, 
pharmacological, medical-history --- are modeled according to different 
sorts of graph structures (hypernodes, nonschematic labeled edges, temporalized 
graphs, and so forth).}

\p{Insofar as different formations within data-structures 
tend to be conveyed via different graph-database constructions, 
the partition of graph-database technology itself 
into competing (partially incompatible) systems 
(distinct architectures, query languages, query-evaluation 
strategies, and so on) can also be a hindrance 
to data-integration, even after adopting flexible 
graph-database technologies.  This is one motivation 
for the hybrid graph models we discuss in later 
chapters, which attempt to recognize numerous 
structuring elements endemic to different flavors 
of graph systems, accommodated into a single hypergraph-based 
query framework.}

\subsection{Representing Trial Data via Object Models}
\p{As a concrete example of open-ended data-integration 
concerns being anticipated as a central element of 
trial design, consider how Object-Oriented models 
for Covid Phylogeny (which we briefly considered 
last chapter) could serve as a nexus for integrating SARS-CoV-2 phylogenetic 
data across multiple studies and healthcare systems.  
Such an Object Model may be extended in different 
ways for different clinical trials examining a range 
of Covid-19 treatments.}

\p{In the context of antibody regimens, 
for instance, scientists need 
to quantify how well the antibodies disable 
Covid-19 spike proteins directly and/or 
how well the antibodies block 
the virus's ability to attach to human cells.  These measurements 
generate data which gauge a patient's 
immune response to Covid-19, 
whether innate (\q{naive}) or boosted by the 
administered antibodies.  Researchers 
have mined such data from different angles, 
including contrasting symptoms presented 
with different levels of severity \cite{ZacharyMontague}, 
\cite{YonggangZhou}, biochemically describing 
the mechanisms of immune response so as 
to augment or simulate that response via 
monoclonal antibodies or similar antiviral 
therapies \cite{PingpingWang}, \cite{YaoQingChen}, 
identifying 
factors explaining act-risk groups' 
greater susceptibility to severe cases 
\cite{LisaPaschold}, or comparisons of SARS-CoV-2 against 
other coronaviruses \cite{TeresaAydillo} 
(these citations are representative examples 
of work conducted by many research groups; 
they could be expanded with similar references 
we included in Chapter 2 when discussing 
immune repertoires, for example).   
  %\cite{}, 
  %\cite{PingpingWang}, 
  %\cite{TeresaAydillo}, .
A Covid-19 software ecosystem 
should then ensure that such immune-response 
data can be effectively parsed and integrated into Object Models 
describing how SARS-CoV-2 is evolving around the 
globe, so researchers (as much as logistically feasible) 
could track each patients' immune response in light of their particular acquired 
SARS-CoV-2 strain/variant and their personal immunological 
profile.  The relevant information for geospatial and 
sociodemographically tagged studies would then include 
patients' prior immuno-profiling and risk assessment, 
immune response (naive or affected by treatment) and 
clinical outcomes, as well as genetic 
tests for the viral variant.  Cross-sectionally 
mapping such data geographically and along 
sociodemographic/socioeconomic contours could 
provide a holistic picture of the pandemic 
at a global scale.}  

%, and to competently track both current and 
%emerging SARS-CoV-2 strains throughout the population.}


\p{Object Models intended to facilitate such 
globally-scoped data integration could then facilitate trials designed 
according to the protocol proposed by Shrestha \textit{et al.}, 
discussed above, where each trial would study a 
preselected (non-random) cohort of patients 
for whom both pre-treatment immunoprofiling data and post-treatment 
outcomes data would be available, so as to compare multiple trials 
against one another.  Object Models customized for each trial 
would generate a data framework through which 
the causal relations between patient profiles and treatment 
outcomes would be investigated.  Customized trial software 
would, accordingly, provide a Reference Implementation demonstrating 
each trial-specific Object Model.  If adopted for multiple trials conducted 
across multiple clinics/hospitals, trials' 
data models may help doctors better understand which 
aspects of patient profiles are particularly 
significant when matching Covid-19 patients to the most 
salutary treatment.}  




\section{Text and Data Mining via CORD-19}

\p{The third methodological area we will concentrate 
on in this chapter is text and data mining, 
using analyses of large-scale document corpora 
and/or biomedical data sets to discover connections 
between research work which might be less evident 
to individuals reading publications in isolation. 
Text Mining and Natural Language Processing 
(\NLP{}) is certainly one methodology 
which has been explored for 
personalized/precision medicine, on the 
theory that automated searches across 
biomedical publication archives  
may detect diagnostically or prognostically 
significant connections between 
individual patients' profiles/symptoms 
and prior studies which doctors 
might not discover otherwise.  
In general, text-mining has been developed 
as one branch of data-mining and Machine Learning 
in general applied to document corpora 
under the aegis of precision-medicine 
(\i{see} for instance \cite{BretonnelCohenHunter}, 
\cite{AyushSinghal},  
\cite{MichaelSimmons}).}

\p{Biomedical text mining is also a good 
case-study in data-integration workflows, 
because typically the text-mining 
process requires synthesizing multiple 
\NLP{} workflows and reading data 
from multiple input sources 
--- \cite{ObiLGriffith}, for example, 
is a good precision-medicine context 
example --- including sentence-parses, 
datasets of biomedical nomenclature, 
domain-specific knowledge bases 
(for gene-sequences, cancer variants, 
genomic-proteomic or genomic-antigen 
assocations, and so forth), manual 
text annotations, etc.}

\p{Despite the perceived potential 
of patient-centered text mining, 
some scientists caution against 
overestimating the power of 
automated \NLP{} platforms 
(\i{see} \cite{HolgerFrohlich}, for instance).  
These critiques are not rejecting text-mining 
in general, but rather observing limitations 
in existing document-encoding formats, 
which are derived from publishing technologies 
whose primary targets are human readers rather 
than machine-automated text processing.  More 
systematic text representation and document 
annotation could alleviate the need 
for probabilistic \NLP{} reasoning engines, 
making text-mining operations more 
precise and reliable.}

\p{As an example of the possibilities and challenges 
of text and data mining scenarios we wilk 
consider \lCnineteen{}, a collection of Covid-19-related 
research articles which was developed (starting 
in Spring 2020) in conjunction with a White 
House \q{call to action} to spur Covid-19 research.  
This White House initiative was described as a 
\q{call to action ... to develop new text and data mining techniques that can help the science community answer high-priority scientific questions related to COVID}.\footnote{See \href{https://www.whitehouse.gov/briefings-statements/call-action-tech-community-new-machine-readable-covid-19-dataset/}{https://www.whitehouse.gov/briefings-statements/call-action-tech-community-new-machine-readable-covid-19-dataset/}}  As raw data for this initiative, the US government helped  
spearhead a consortium of industry and academic 
institutions, headed by the Allen Institute for AI Research, 
who curated a \q{machine-readable Coronavirus literature collection} 
which includes article metadata and (in most cases) 
publication text  
for over 280,000 coronavirus research papers (as of 
mid-2021) \cite{CORD}, \cite{LucyLuWang}.  This
corpus is paired with links to publisher portals 
(including Springer Nature, 
Wiley, Elsevier, the American Society for Microbiology, and the New England Journal of Medicine) providing full open access to \Covid{}-related 
literature; these resources collectively constitute 
\Cnineteen{} (the \q{Covid-19 Open Research Dataset}).}

\subsection{Overview of CORD-19}
\p{The \Cnineteen{} 
collection was formulated with the explicit goal 
of promoting both text mining and data mining solutions 
to advance coronavirus research.  This means that 
\Cnineteen{} is intended to be used both as a document 
archive for text mining and as a repository for 
finding and obtaining coronavirus data for subsequent 
research.  The White House announcement directly requests 
institutions to develop \textit{additional} technologies 
which would help scientists and jurisdictions to 
take advantage of \Cnineteen{} as it was initially 
published.  In short, \Cnineteen{} was released with the 
explicit anticipation that industry and academia would 
augment the underlying data by layering on additional 
software.}

\p{Despite the obvious benefit to researchers, the 
health-care community, and the public at large 
in publishers choosing to release a substantial 
quantity of Covid-19 related literature in 
Open-Access fashion, \Cnineteen{} is not without 
certain limitations.  These largely stem from 
how the articles are encoded into an ostensibly 
(\JSON{}-based) machine-readable format.  To be 
fair, the problems we identify here reflect the 
current authors' own personal assessments of 
the \Cnineteen{} corpus; they are not broad 
criticisms which have been asserted by researchers 
working directly with \Cnineteen{} or discussed in 
peer-reviewed literature.  With that caveat, however, 
we assert that certain issues deserve mention: 

%\vspace{-2em}
\begin{description}
	
\item[Transription Errors]  
Transcription errors can cause the machine-readable 
text archive to misrepresent the structure 
and content of documents, hindering text-mining 
technology that targets the archive.  In \Cnineteen{}, 
for instance, there are cases of scientific notation and terminology 
being improperly encoded.  As a concrete example, \colorq{2{\textquotesingle}-C-ethynyl} is encoded in \Cnineteen{} as \colorq{2 0 -C-ethynyl}, 
which could stymie text searches 
against the \Cnineteen{} corpus (see \cite{Eyer} for 
the human-readable publication where this error is 
observed; the corresponding index in the corpus is 9555f44156bc5f2c6ac191dda2fb651501a7bd7b.json).

\item[Poorly Indexed Research Data]  Although 
\Cnineteen{} provides a structured representation 
of a large collection of research 
\textit{papers}, there is no easy-to-use tool 
for finding research \textit{data} 
through \Cnineteen{}.

\item[Poorly Integrated Research Data]  The 
research data which \textit{can} be 
accessed through \Cnineteen{} evinces 
a wide variety of technical fields 
and formats, with distinct software 
requirements; as a result, it is a 
difficult task to merge and integrate 
different data sets related to 
\Covid{}.  At present, \Cnineteen{} 
does not include any software tools 
or computer code that would facilitate 
data integration.   

\item[Disconnect Between Text Data and Publisher Portals]  
Although most of the \Cnineteen{} manuscripts represent 
peer-reviewed literature which can be accessed through 
document portals (for instance, the 
National Center for Biotechnology Information website), 
the \Cnineteen{} archival schema does not represent 
these links (except indirectly via Document Object 
Identifiers).  As such, there is no easy way for 
researchers to find and read publications which have been 
flagged by text-mining algorithms as being potentially 
of interest to them.  Furthermore, there is no direct 
mechanism to enlarge the \Cnineteen{} corpus with papers 
newly added to publisher portals.
\end{description}}

\p{To clarify the final comment: the Allen Institute for AI, which 
curated \Cnineteen{}, encourages publishers to contribute new 
(or newly-available) articles to the corpus.  However, integration 
with \Cnineteen{} is not developed as a formal step 
in the publication workflow.  In particular, publishers 
are not themselves generating machine-readable document infosets 
that can be integrated with the \Cnineteen{} schema (which, 
in turn, causes transcriptions errors and other problems 
as just outlined).}



\p{With respect to text mining, an immediate problem arises 
in \Cnineteen{}'s archive-construction methodology: 
especially, how the text was parsed from \PDF{} files.  
This is a process 
which almost inevitably causes imprecise or inaccurate text representation, which can degrading the quality of 
the archive unless manual or automated corrections are made.  In particular, the \Cnineteen{} library evinces  
transcription errors, as mentioned above (especially in relation to 
technical or scientific phrases and terminology); 
scientific notation in particular may be improperly 
encoded.  Moreover, there is no 
semantic marking identifying that (say) the \colorq{2 0 -C-ethynyl} 
text segment has a specific technical meaning.  These 
errors or limitations arise in part from unavoidable 
anomalies which occur when reading texts from \PDF{} files 
rather than from machine-readable, structured 
formats such as \XML{}.}
 
\p{It is also worth observing that the \JSON{} format used 
for encoding \Cnineteen{} manuscripts presents some 
logistical challenges for any operations related to 
text-mining or to cross-referencing publications and 
data sets.  In particular, \Cnineteen{} makes 
partial use of \q{standoff annotation}; specifically, 
document features such as citations and references are 
notated through character offsets into the paragraph where 
they appear.  As a result, accurately reading these document 
elements requires synthesizing data points parsed from several 
distinct objects in the \JSON{} code, which is only feasible given 
a client library built to interface with the \Cnineteen{} files 
in accord with their specific schema.  Such a client 
library would implement convenience procedures to handle 
recurring tasks, such as obtaining the full bibliographic 
reference affixed to a given location in a manuscript.}

\p{With respect to \textit{data} mining in the \Cnineteen{} 
context, the limitations in the currently available raw 
\Cnineteen{} data are even more pronounced than in the 
context of text mining.  In particular, 
neither the article 
metadata nor the full open-access document collections have 
any mechanism for actually obtaining data published 
alongside research papers, or even identifying which papers 
have accompanying data in the first place.  The Springer 
Nature collection which was originally one component within \Cnineteen{} illustrates the limitations of this relatively unstructured 
data-publishing approach (this following analysis will 
focus on Springer Nature, but the problems identified 
are no less pronounced on the other \Cnineteen{} portals 
--- 
if anything, because Springer Nature 
allows readers to browse articles in \HTML{} within the 
web portal directly, one can ascertain whether research data 
exists for an article without downloading and reading 
it; with other \Cnineteen{} resources it is actually 
harder to locate supplemental data when available).  
Initially, the Springer Nature portal 
encompassed 43 articles, 
of which 15 were accompanied by research data that 
could be separately downloaded (this number does 
not include papers that document research 
findings only indirectly, via tables or graphics 
printed inline with the text).  Collectively 
these articles referenced over 30 distinct data 
sets (some papers were linked to multiple 
data sets), forming a data collection which could 
be a valuable resource for \Covid{} research --- 
not only through the raw data made available 
but as a kernel around which new coronavirus data 
could accumulate.  However, there is currently no 
mechanism to make this overall collection available 
as a single resource.\footnote{As \Cnineteen{} has 
evolved, the publisher-specific sections therein 
appear to be merged into portals such as 
Springer Nature directly, so our above comments 
based on isolating Springer Nature articles are 
probably more applicable to the original 
archive design than the current technology.  However, 
insofar as the current portal simply defers to 
publisher-specific search features, we would argue that 
accessing Covid-19 data sets through \Cnineteen{} 
is if anything more difficult than before.}}

\p{This problem demonstrates, among other things, 
how document-metadata formats such as 
the Research Object protocol are limited in applying 
only to \textit{single} articles.  As a result, there is no 
commensurate protocol for publishing \textit{groups} 
of articles which are tied to groups of data sets unified 
into an integral whole.  Open-access \Covid{} papers also 
reveal limitations of exiting online document portals, 
especially with respect to how publications are 
linked to data sets.  In particular, there is no clear indication that a 
given paper is associated with downloadable data; usually 
readers ascertain this information only by reading or 
scrolling down to a \q{supplemental materials} or 
\q{data availability} addendum near the end of the article.  
Moreover, because the Springer Nature portal (and similar 
publisher resources) aggregates 
papers from multiple sources, there is no consistent 
pattern for locating data sets; each journal or 
publisher has their own mechanism for alerting readers 
to the existence of open-access data and citing where 
they could be downloaded.}

\subsection{Data Integration within CORD-19}
\p{Aside from the issues which are likely to hinder text and data mining 
across \Cnineteen{}, the 
collective group of \Covid{} data sets also illustrates 
the limitations of information spaces pieced 
together from disconnected raw data files with little 
additional curation.  The files included in this 
group of data sets encompass a wide array of file types 
and formats, including \FASTA{} (which stands for Fast-All, 
a genomics format), \SRA{} (Sequence Read Archive, for 
\DNA{} sequencing), \PDB{} (Protein Data Bank,  
representing the \ThreeD{} geometry of protein 
molecules), \MAP{} (Electron Microscopy Map), \EPS{} 
(Embedded Postscript), and \CSV{} (comma-separated values).  
There are also tables represented in Microsoft Word 
or Excel formats.  Although these various formats are 
reasonable for storing raw data, not all of them 
are actually machine-readable; in particular, 
the \EPS{}, Word, and Excel files need manual processing 
in order to use the information they provide in a 
computational manner.  A properly curated data collection 
would need to unify disparate sources 
into a common machine-readable representation (such as \XML{}).}

\p{Going further, productive data curation should also 
aspire to \textit{semantic} integration, unifying disparate 
sources into a common data model.  For example, multiple 
spreadsheets among the Springer Nature \Covid{} data sets 
hold sociodemographic and epidemiological information relevant 
to modeling the spread of the disease.  These different 
sources could certainly be integrated into a canonical 
social-epidemiology-based representational paradigm which 
recognizes the disparate data points which might be 
relevant for tracking the spread of \Covid{} (with the 
potential to unify data from many countries and 
jurisdictions).}

\p{This is not only an issue of data
\textit{representation} (viz., how data is physically 
laid out), but also of data types and computer code.  
According to the Research Object protocol, 
data sets should include a code base 
which provides convenient computational access to the 
published data.  In the case of \Covid{}, this entails 
creating a sociodemographic and epidemiological code 
library optimized for \Covid{} information, which would 
be the primary access point for researchers seeking to 
use the data which has been published in conjunction with 
the 43 manuscripts examined here that were aggregated 
on Springer Nature, along with any other coronavirus 
research which comes online.  Similar comments 
apply not only to tabular data represented in spreadsheet 
or \CSV{} form, but to more complex molecular or 
microscopy data that needs specialized scientific software 
to be properly visualized.}

\p{Considering the overall space of \Covid{} data, it is 
unavoidable that some files require special applications 
and cannot be directly merged with the overall collection.  
For instance, there is no coherent semantics for 
unifying Protein Data Bank files with sociodemographics 
and epidemiology; files of the former type have specific scientific 
uses and can only be understood by special-purpose software.  
Nevertheless, a well-curated data-set collection can 
make using such special-purpose data as convenient as 
possible.  In the case of Protein Data Bank, a downloadable 
\Covid{} archive can include source code for \IQmol{}, a 
molecular-visualization application that supports 
\PDB{} (among other file formats) and has few 
external dependencies (so it is relatively easy to 
build from source).}

\p{Indeed, a curated \Covid{} archive 
might include an enhanced version of software 
such as that \IQmol{} prioritizes 
\Covid{} research, with the goal of integrating biomolecular 
and social-epidemiological data as much as possible.  
For example, as \Covid{} potentially mutates in different 
ways in different geographic areas, it will be important 
to model the connections between \q{hard} scientific 
\Covid{} information and sociodemographics.  
As the pandemic evolves, genomic and biochemical information 
may be linked to particular strains of virus, which 
in turn are linked to sociodemographic profiles: certain 
strains will be more prevalent in certain populations.  
Consequently, models of \Covid{} variants will need to be 
formulated and then integrated with both chemical/molecular 
data and sociodemographic/epidemiological data.  Different 
\Covid{} strains then form a bridge linking these different 
sorts of information; researchers should be able to pass 
back and forth from molecular or genomic visualizations of 
\Covid{} to social-epidemiological charts and tables based 
on viral strains.  Ideally, capabilities for this 
sort of interdisciplinary data integration would be 
provided by a \Covid{} archive as enhancements to applications, 
such as \IQmol{}, that scientists would use to study the 
published data.}

\p{It is worth noting that a data-mining platform requires 
\textit{machine-readable} open-access research data, 
which is a more stringent requirement than simply publishing 
data alongside which can be understood by domain-specific 
software.  For example, radiological imaging can be a source 
of \Covid{} data insofar as patterns of lung 
scarring, such as \q{ground-glass opacity}, is a leading 
indicator of the disease.  Consequently, diagnostic 
images of \Covid{} patients are a relevant kind of 
content for inclusion in a \Covid{} data set 
(see \cite{Shi} as a case-study).  However, 
diagnostic images are not in themselves 
\q{machine readable.}  When medical imaging is 
used in a quantitative context (e.g., applying 
Machine Learning for diagnostic pathology), 
it is necessary to perform Image Analysis to convert the raw data 
(viz., in this case, radiological graphics) into 
quantitative aggregates (for instance by using image 
segmentation to demarcate geometric boundaries and 
then defining diagnostically relevant features, such 
as opacity, as a scalar field over the segments).  
In short, even after research data is openly published 
by article authors, it may be necessary to perform 
additional analysis on the data for it to be 
a full-fledged component of a 
machine-readable information space.\footnote{%
This does not mean that diagnostic images (or 
other graphical data) should not be placed in a 
data set; only that computational reuse of such 
data will usually involve certain numeric 
processing, such as image segmentation.  
Insofar as this subsequent analysis is performed, 
the resulting data should wherever possible 
be added to the underlying image data as a 
supplement to the data set.}} 

\p{Another concern in developing an integrated \Cnineteen{} 
data collection is that, logistically speaking, 
not all \Covid{} data is practical 
to reuse as a downloadable package.  This is especially 
true for genomics; several of the aforementioned 
43 coronavirus papers included data published via 
online data banks capable of hosting data sets that 
are too large for an ordinary computer.  In these 
situations scientists formulate queries or 
analytic scripts that are sent remotely to the online 
repositories, so that researchers access the actual 
published data only indirectly.  Nevertheless, access to 
these data sets can still be curated as part of 
a \Covid{} package; in particular, computer code 
can be provided which automates the process of 
networking with remote genomics archives through the 
accession numbers and file-formats which those archives 
recognize.}

\p{As a final point on the topic of integrating 
disparate \Cnineteen{} research data, note that 
an overarching framework for indexing \Covid{} data 
sets would also facilitate the process of cross-referencing 
article text and research data.  In particular, 
the annotation system employed for 
\Cnineteen{} could profitably be enhanced 
by a system of \textit{microcitations} that apply 
to portions of manuscripts \textit{as well as} data sets.  
In the publishing context, a microcitation is defined as a 
reference to a partially isolated fragment of a larger 
document, such as a table or figure illustration, or a 
sentence or paragraph defining a technical term, 
or (in mathematics) the statement/proof of a definition, axiom, 
or theorem.  In data publishing, \q{data citations} are 
unique references to data sets in their entirety or to 
smaller parts of data sets.  A data microcitation is then a 
fine-grained reference into a data set: for example, 
\q{the precise data records actually used in a study} (as 
defined by the Federation of Earth Science Information Partners; 
see \cite{ESIP}), 
one column in a spreadsheet, or one statistical parameter in a 
quantitative analysis.}

\p{Ideally, the text-mining and 
data-mining notions of microcitation should be combined into a unified 
framework.  In particular, text-based searches against 
the \Cnineteen{} corpus should also try to find matches in the 
data sets accompanying articles within the corpus.  As a concrete example, 
a concept such as \q{expiratory flow} appears in \Cnineteen{} 
both as a table column in research data and as a medical concept 
discussed in research papers; a unified microcitation framework 
should therefore map \textit{\color{drp}{expiratory flow}} as a keyphrase 
to both textual locations and data set parameters.  
Similarly, a concept such as 
\textit{\color{drp}{2{\textquotesingle}-C-ethynyl}} (mentioned earlier 
in the context of transcription errors) 
should be identified both as a phrase in 
article texts and as a molecular component 
present within compounds whose scientific 
properties are investigated through \Cnineteen{} 
research data, so that a search for this 
concept can trigger both publication and 
data-set matches. 
Implementing this kind of unified search mechanism requires that data 
sets be \textit{annotated} with techniques similar to 
those used for marking up Natural Language techniques.}


\p{Considering the inter-disciplinary nature of \Covid{} research, 
it is unavoidable that different scientists will need 
different sorts of specialized software to analyze the 
kinds of information relevant to their research.  For 
instance, the 
computational techniques applicable to diagnosing 
coronavirus infection are scientifically very different 
from those used for genomic or epidemiological studies 
of the disease; it is impractical to expect 
pathologists to use the same software as bioinformaticians 
studying the pathogen, or for either to use the same 
software as virologists modeling the (potential or observed) 
spread of the disease.  In short, even if 
scientists from disparate disciplines start with a 
common pool of raw data, they will need to 
analyze this data through a diverse set of 
supplemental computational tools, which will vary 
not only across disciplines but also in terms of 
the software and laboratory facilities available 
to different researchers through their institutions.  
In this sense it is impossible to unify all 
\Cnineteen{} data into a \textit{fully} self-contained 
information space.}

\p{Nevertheless, committing to \q{standalone} data publishing 
remains a valuable goal even in a context where published 
data sets will invariably migrate to different 
digital ecosystems.  Although scientists may use 
external digital tools \textit{when necessary} to 
perform certain calculations, or when interfacing with 
laboratory equipment, we strongly recommend that 
the degree of variation across different domain-specific extensions to 
\Cnineteen{} be greatly minimized.  
Ideally, that is, the version of \Cnineteen{} 
(along with its supporting technology) 
found in a biomedical setting should be as 
similar as possible to that found in a 
biochemical context, or a health-policy 
context.  In the absence of 
any initiative to limit this drift, \Cnineteen{} 
could easily devolve into a federation of 
separate resources which have no interconnection 
apart from their nominal focus on \Covid{}.}


\p{This section has highlighted limitations of data 
sets published in conjunction with coronavirus articles made 
available as open-access resources on Springer Nature 
(and, by extension, \Cnineteen{}).  
The central point here is to argue for a distinct data-curation 
stage in the publication process, with data curators 
playing a role distinct from that of both 
authors and editors.\footnote{%
The point here is not to critique the work of individual 
authors; curating data sets according to exacting 
scientific standards demands a separate vein of 
expertise which typically lies outside researchers' 
disciplinary scope.  The point is rather that 
publishers should recognize data curation as a 
distinct process and skill-set complementary 
to both writing and editing research works.}  
Moreover, the discussion has hopefully highlighted problems 
with current data-sharing paradigms, even those such 
as the Research Object and \FAIR{} initiatives which are 
explicitly devoted to improving how open-access data sets are 
published.  \Cnineteen{} exposes several 
lacunae in the Research Object protocol, for example, 
which point to the need for a more detailed extension 
of this protocol.  In particular, an enhanced protocol 
should encompass: 

\begin{enumerate}[leftmargin=3pt, itemsep=4pt,topsep=11pt]
\item{} A canonical framework for archiving collections 
of data sets, not only single data sets (and not only 
groups of data sets published with a single research 
paper).  For example, all data sets published alongside 
the 43 Springer Nature articles could be unified into a 
single collection.

\item{} A code base accompanying data-set collections 
designed to help research unify the information provided.  
Curating the overall collection would involve pooling 
disparate data into common representation, and 
implementing computer code which deserializes and processes 
the unified data accordingly.  For instance, \CSV{}, 
\EPS{}, and Microsoft Word/Excel tables could be migrated 
to \XML{}, \JSON{}, or 
a more complex common format.  Customized computer code could then 
be implemented specifically to parse and merge the 
information present in single data sets within the 
overall collection.  This implementation would 
reciprocate the Research Object goal of unifying 
code and data, but again would operate at the level 
of an aggregate of research projects rather than a 
single Research Object.

\item{}  A unified data-set collection should 
be self-contained as much as possible, and should be 
built around a foundation where advanced computing 
capabilities are available in a transparent, 
standalone fashion, without requiring tools 
outside the collection itself.  One way to 
achieve this is by gravitating 
toward components that can provide 
features such as scripting and data persistence 
through components that can be shared 
in pure source-code fashion, 
such as the WhiteDB database engine 
\cite{EnarReilent} and the AngelScript 
scripting language \cite{AS}.   

\item{}  A unified data-set collection should also provide 
prototyping and remote-access tools to interface with 
web-based information spaces that host data sets 
too large to be individually downloaded.  Ideally, 
these would include simulations of remote services, which 
would help scientists understand the design of 
the remote archives and how to interface with them.
These simulations could function analogously to 
(for instance) 
PurpleData, a prototyping tool for Google's BigData 
developed by Verily (the Alphabet subsidiary developing 
an online \Covid{} portal for remote diagnostics 
and disseminating public information about the pandemic). 

\item{}  Finally, a unified research portal should  
influence the design of the web portals where associated 
texts are published.  It should be easy for readers to 
identify which articles have supplemental data files and 
to download those files if desired.  Moreover, 
textual links should be established between publication 
content and data sets --- for instance, a plot or 
diagram illustrating statistical or equational distributions 
should link to the portion of the data set from which that 
quantitative data is derived.
\end{enumerate}}

\p{This discussion has used the \Covid{} crisis as a 
lens through which to examine data-publishing limitations 
in general.  Such limitations are not specific to coronavirus in particular.  
However, the nearly unprecedented urgency of this epidemic 
reveals how both the scientific and publishing industries are still struggling 
to develop technologies and practices which keep pace 
with the intersecting needs of systematic research 
and public policy.  An optimistic projection is 
that the crisis will spur momentum toward a more 
sophisticated data-sharing paradigm --- perhaps a 
generalization of the Research Object protocol 
toward data-set collections.}

\subsection{Reviewing the CORD-19 Document Model}

\p{In order to discuss the possibilities 
and limitations of \Cnineteen{} (and potentially 
other document corpora with a similar design) it 
is worth examining how \Cnineteen{} encodes textual 
data in greater detail.  This discussion has ramifications 
outside of \Cnineteen{} itself, 
insofar as \Cnineteen{} hopefully points to 
gaps in current publishing technologies.  These gaps need to be 
addressed if publishers are to curate open-access corpora 
which truly leverage the digital and interactive 
technology available to us with modern software.}

\p{The basis of \Cnineteen{}'s infrastructure 
is a \JSON{} scheme which describes the 
document hierarchy of research articles encoded 
within the corpus.  Apart from metadata 
(consisting of basic details such as document title and authors' 
names) and bibliographic entries, all document 
content according to this schema is divided into paragraphs 
(implicitly the documents are divided into sections 
as well, but sections are notated as properties 
of the paragraphs they contain, not as a separate 
level in the hierarchy).   
Each paragraph encoding contains an underlying string 
vector (a stream of characters) and, separate and apart from 
that, character \q{spans} which point to 
references (such as Named Entites), citations, and 
equations.  This indicates that the \Cnineteen{} encoding 
uses \q{standoff annotation,} where any content 
modifying the interpretation assigned to portions of 
the main text is notated with a series of data 
structures described apart from the main text itself.}

\p{Standoff text-encoding systems may be contrasted with \XML{} 
or \HTML{}, where \q{tags} are mixed with character 
data.  For example, consider a span of text which 
quotes from another document: in \HTML{}, the 
special status of the quoted text may be 
marked by surrounding the text with \textbf{<quote>} 
start and end tags.  Syntactically, this markup system 
has the effect that tags and text are seen side-by-side: any 
content governed by the \textbf{<quote>} (i.e., the text of 
the quote itself) is printed immediately after the 
begin-tag, and the quotation ends when the last character 
is followed by an end-tag (i.e., \textbf{</quote>}).}

\p{Apart from such syntactic details, the distinction 
between tag-based markup and standoff annotation 
determines the \q{semantics} of the document, insofar 
as tags form a document hierarchy.  Continuing the 
\textbf{<quote>} example, the text-span inside the 
quote tags is represented as a \textit{child element} 
of the quote, whereas the quote itself may be a child 
element of a larger-scale entity (such as a paragraph).  
In effect, the paragraph \textit{contains} a quote, 
and the quote \textit{contains} a string of characters.  
Such nested levels of containment provide the structure 
through which hierarchical documents (formats 
such as \XML{} and \HTML{}) are interpreted.}

\p{To see the contrast with \textit{standoff} annotation, 
if one were to describe a document using a standoff annotation 
system, the notation that a particular 
span of characters belongs to a quotation would 
not be marked-up amidst the characters themselves.  
Instead, the quotation-designation would use numeric indices to 
declare that the character at a certain position in the 
main text begins a quotation, and some later character 
in the text ends that quotation.  When serializing 
documents with standoff annotation, all the characters 
in a document are typically represented as one 
character-stream, and any notation describing markup applied 
to spans within that character stream is asserted afterward, 
using indexes into the stream to demarcate element boundaries.}

\p{The \JSON{} schema used for \Cnineteen{} is not 
entirely standoff, because there is a document hierarchy 
(for example, a publication's abstract is modeled as a sibling 
element to the main body text, so abstracts and the main text represent 
an intermediate hierarchical level, contained within the overall 
document and containing individual paragraphs).  However, 
\Cnineteen{} uses in effect a standoff-annotation system 
for each paragraph, so there is no hierarchical level 
smaller than paragraphs themselves, except implicitly; after the 
text (viz., the character stream) there are subsequent notations
of spans within the paragraph (each span description is considered 
a child of the paragraph itself, as is the paragraph text).}

\p{This arrangement has consequences for text mining algorithms, which 
may be strengths or weaknesses in different contexts.  
One consequence is that the raw text is all grouped together 
in one place --- algorithms do not have to tie together child 
nodes of disparate \XML{} elements to derive a beginning-to-end 
sequence of the text belonging to any paragraph.  Instead, it 
is simply necessary to read all data in the 
\q{text} field of the relevant \q{paragraph} 
object.  The character-sequence in this text may 
contain words and sentences, but potentially other 
strings of symbols (such as chemical formulae) 
which are not explicitly marked.  This may or may not 
be desirable.  It could potentially complicate 
\NLP{} tasks, because the language-processing 
components will be fed not only sequences of English 
words but also, sometimes interspersed among ordinary 
words, technical symbol-sequences such as 
\colorq{2{\textquotesingle}-C-ethynyl} (an 
example used earlier in this chapter).  
Standoff annotations may or may not be 
effective in marking the boundaries of such 
extra-lexical sequences; certainly we 
cannot rely on Named Entity detectors to 
properly identify and demarcate the boundaries 
of all uses of technical terminology or special 
symbols (again, the limitations of automated 
annotation are discussed earlier in this chapter).}

\p{In discussing standoff annotation it is also 
worth considering how the text of \Cnineteen{} 
publications was obtained.  According to 
\Cnineteen{} documentation, most full-text 
transcriptions in the corpus were obtained from 
\PDF{} files, via a pipeline using \TEI{} (Text 
Encoding Initiative) \XML{} as an intermediate 
representation.  Necessarily, then the 
encoded text is only an approximate representation of 
the original:  

\begin{displayquote}
To provide accessible and canonical structured full text, 
we parse content from PDFs and associated paper documents.
The full text is presented in a JSON schema designed 
to preserve most relevant paper structures 
such as paragraph breaks, section headers, and 
inline references and citations. ... We recognize that 
converting between PDF or XML to JSON is lossy.
However, the benefits of a standard structured format, 
and the ability to reuse and share annotations
made on top of that format have been critical to the
success of CORD-19. ... 
Though we have made the structured full text
of many scientific papers available to researchers
through CORD-19, a number of challenges prevent 
easy application of NLP and text mining techniques 
to these papers.  First, the primary distribution 
format of scientific papers --- PDF --- is not
amenable to text processing. The PDF file format
is designed to share electronic documents rendered
faithfully for reading and printing, not for 
automated analysis of document content.  Paper content
(text, images, bibliography) and metadata extracted
from PDF are imperfect and require significant
cleaning before they can be used for analysis.
Second, there is a clear need for more scientific
content to be made easily accessible to researchers.
Though many publishers have generously made
COVID-19 papers available during this time, there
are still bottlenecks to information access. ... 
Lastly, there is no standard format for 
representing paper metadata.  Existing schemas like 
... JATS[,] Crossref [or] Dublin Core have
been adopted as representations for paper metadata.
However, there are issues with these standards; they
can be too coarse-grained to capture all necessary
paper metadata elements, or lack a strict schema. 
... Without solutions to the above problems, NLP
on COVID-19 research and scientific research in
general will remain difficult. \cite[page 6]{LucyLuWang}
\end{displayquote}

As an example of these \NLP{} issues, 
consider the challenge of demarcating all 
named entities, particularly technical 
character-sequences (such as chemical 
formulae) which are not ordinary lexemes.  
Whether or not authors explicitly mark up 
such sequences (they may well do so in that 
formulae or equations are often typeset 
differently than normal text) this markup is 
not preserved in \PDF{} versions of articles.  
As the authors of the last-cited article point 
out, many (roughly 38\%) of papers in \Cnineteen{} 
are also available in the \JATS{} (Journal Article Tag Suite) 
format, which is a more precise text encoding 
than \PDF{}.  However, even in this context \JATS{} 
does not compel authors to explicitly notate 
textual entities such as special terms or 
character-sequences --- in fact \JATS{} 
does not truly have an obvious structure or 
set of alternative structures for identifying
what would normally be considered annotation-worthy  
text spans or named entities; the closest correlates 
are probably the generic \textbf{<kwd>} (keyword) and 
\textbf{<abbrev>} (abbreviation) tags as well 
as discipline-specific options such as 
\textbf{<chem-struct>} (for chemical structures) 
and \textbf{<disp-formula>} (for mathematical expressions).  
In short, building a corpus such as \Cnineteen{} for 
rigorous text-mining is made more difficult because 
authors and publishers do not publish texts in 
formats which are optimized for text mining in the 
first place; the acknowledged limitations of 
\Cnineteen{} reflect problems of industry 
practice, not programming lacunae that could 
be alleviated with more sophisticated \NLP{} algorithms.}

\p{Having acknowledged these limitations, a discussion of 
document corpora could then reasonably pivot from 
the empirical goal of curating useful text archives from 
currently published text to examining how more 
sophisticated corpora may be published in the future.  
It is reasonable, for example, to propose 
that full-text publications be released 
\textit{both} in reader-friendly \PDF{} form 
\textit{and} in machine-readable forms such as 
\JATS{}.  This is not just an abstract proposal; 
indeed, the text of this very book has been prepared 
using a novel document-generation system 
which creates both machine-readable structured text 
and \PDF{} output, moreover with cross-referencing 
between them; notably, the positions of discursively 
important textual markers, such as sentence boundaries, 
are mapped to \PDF{} screen coordinates (the code library 
for the book includes document-generation code as well as 
the data set of coordinate positions generated 
as part of the book's publication workflow).  
In particular, it is reasonable for authors 
and editors to manually introduce textual 
annotations for content such as named entities, 
keywords, important technical terms, and other 
content which should be targeted by \NLP{} engines 
separate and apart from ordinary lexemes with 
their conventional natural-language semantics.  
Typically such specialized terms/lexemes would 
be marked up in any case because they may require 
distinct fonts or styling than their surroundings.  
It is also reasonable to manually define sentence 
boundaries via simple rules (e.g. two following spaces 
mark the end of a sentence; a single space, such as 
that following an abbreviation, indicates situations 
where a character such as a period, which could potentially mark the 
end of a sentence, is actually playing a different discursive role).}

\p{By following simple rules of document content-entry and 
lexicography, certain \NLP{} tasks, such as sentence-boundary 
and Named Entity recognition, can be optimized --- eliminating 
the need for probabilistic algorithms and relying instead 
on much less sophisticated, but more accurate, markup-parsing 
logic.  If sentence boundaries and Named Entities are 
explicitly annotated in machine-readable text encodings, then 
extracting these features is not really an issue of 
\q{Natural Language Processing} as such.  
On the other hand, \AI{}-driven analysis of document corpora 
would still require \NLP{} for other aspects of 
parsing documents; it is unreasonable to expect authors, 
for instance, to manually notate sentence parse-graphs.  
This then suggests the question of where the boundary 
lies between discursive structures which might reasonably 
be left to authors or editors to manually notate (e.g. 
sentence boundaries) and those which in practice could 
only be obtained via \NLP{} (such as part-of-speech 
tags).  Related to this question is how best to 
model \NLP{} structures, such as the trees or graphs 
representing the syntax of natural-language sentences.  
We will consider this question in subsequent chapters 
in the context of Conceptual Space Theory.}




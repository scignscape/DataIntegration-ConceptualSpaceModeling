
\documentclass{report}
\pagestyle{empty}


\setcounter{tocdepth}{3}

\newcommand{\q}[1]{{\fontfamily{phv}\selectfont ``}#1{\fontfamily{phv}\selectfont ''}} 

\newcommand{\vs}{\vspace{1.6em}} 



\usepackage[letterpaper, left=.9in,right=.8in,top=.54in,bottom=.54in, 
paperheight=12.5in,paperwidth=9.5in]{geometry}

\begin{document}

\fontsize{12}{20}\selectfont

\vspace{-1em}

\noindent{\bfseries{\huge Contents}\hfill Page No.\vspace \bigskipamount \par }
%\contentsline {chapter}{ACKNOWLEDGEMENTS}{?}
%\contentsline {chapter}{ABSTRACT}{?}
%\contentsline {chapter}{LIST OF TABLES}{?}
%\contentsline {chapter}{LIST OF FIGURES}{?}
%\contentsline {chapter}{ABBREVIATIONS}{?}


\contentsline {chapter}{\numberline {1}Introduction}{}

\contentsline {section}{\numberline {1.1}Prelude}{1}

\contentsline {section}{\numberline {1.2}Data Integration, Hypergraphs, 
and Type Theory}{2}

\contentsline {section}{\numberline {1.3}Philosophy and the Semantic Web}{4}

\contentsline {section}{\numberline {1.4}Covid, Philosophy, and Science}{6}

\contentsline {section}{\numberline {1.5}How Covid-19 May Accelerate Emerging
Trends in Research Data-Curation}{7}

\contentsline {section}{\numberline {1.6}Navigating the Proliferation of 
Research Data}{8}

\vspace{2.5em}


\noindent{\bfseries{\Large Part I: Biomedical Data Formats and 
Data Integration}}

\vspace*{3pt}


\contentsline {chapter}{\numberline {2}Data Structures Associated with Biomedical Research}{}

\contentsline {section}{\numberline {2.1}Introduction}{1}

\contentsline {section}{\numberline {2.2}Personalized Medicine in the Context of
Covid-19}{2}

\contentsline {subsection}{\numberline {2.2.1}Precision Medicine as a Catalyst for Biomedical Data Sharing}{3}

\contentsline {subsection}{\numberline {2.2.2}Software Alignment for Covid Phylogeny Studies}{4}

\contentsline {subsection}{\numberline {2.2.3}Personalized Medicine and Immuno-Profiling}{5}

\contentsline {section}{\numberline {2.3}A Review of Certain Commonly-Used
Biomedical Data Formats}{6}
\contentsline {subsection}{\numberline {2.3.1}DICOM (Digital Imaging and Communications in Medicine)}{6}
\contentsline {subsection}{\numberline {2.3.2}Next Generation Sequencing and other Genomics Formats}{7}
\contentsline {subsection}{\numberline {2.3.3}The Flow Cytometry Standard (FCS) File Format}{9}
\contentsline {subsection}{\numberline {2.3.4}Image Segmentation, Contours, and Regions of Interest}{10}

\contentsline {subsection}{\numberline {2.3.5}Common Data Models for Clinical
Research}{11}

\vspace{13pt}

\contentsline {chapter}{\numberline {3}Data Mining and Predictive 
Analytics for Cancer and Covid-19}{}

\contentsline {section}{\numberline {3.1}Introduction}{1}

\contentsline {section}{\numberline {3.2}Precision Medicine and Bioimaging}{1}

\contentsline {subsection}{\numberline {3.2.1}The Basic Synthesis Between Bioimaging and Precision Medicine}{1}


\contentsline {subsection}{\numberline {3.2.2}Case Study: The Cancer Imaging Phenomics Toolkit}{1}

\contentsline {subsection}{\numberline {3.2.3}Multi-Application Networks in the Context of Scientific Research Data}{3}


\contentsline {section}{\numberline {3.3}Precision Medicine in Trial Design}{4}

\contentsline {subsection}{\numberline {3.3.2}Customizing Clinical Trial Management Software}{5}

\contentsline {subsubsection}{\numberline {3.3.1.1}Toward Fine-Grained Sociodemographic Models}{5}

\contentsline {subsubsection}{\numberline {3.3.1.2}Measuring Cognitive and Neurological Effects}{6}

\contentsline {subsubsection}{\numberline {3.3.1.3}Aggregating Trial Data via Graph Models}{7}


\contentsline {subsection}{\numberline {3.3.2}Representing Trial Data via Object Models}{7}


\contentsline {section}{\numberline {3.4}Text and Data Mining via CORD-19}{8}
\contentsline {subsection}{\numberline {3.4.1}Overview of CORD-19}{8}
\contentsline {subsection}{\numberline {3.4.2}Data Integration within CORD-19}{10}
\contentsline {subsection}{\numberline {3.4.3}Reviewing the CORD-19 Document Model}{12}

\vs{}

\contentsline {chapter}{\numberline {4}Modular Design, Image Biomarkers, and Radiomics}{}

\contentsline {section}{\numberline {4.1}Introduction}{1}

\contentsline {section}{\numberline {4.2}Image Biomarkers (and Others) for Cardiac and Oncology Diagnostics}{1}

\contentsline {subsection}{\numberline {4.2.1}Image Registration and Radiomics for Cardiac Diagnosis}{2}
\contentsline {subsection}{\numberline {4.2.2}From Image-Annotations to Image Biomarkers}{6}
\contentsline {subsection}{\numberline {4.2.3}Tumor Histopathology and Simulation}{9}


\contentsline {section}{\numberline {4.3}Multi-Aspect Modular Design in a
Heterogeneous Data Space}{12}

\contentsline {subsection}{\numberline {4.3.1}The Overlap Between Research and Clinical Data}{13}
\contentsline {subsection}{\numberline {4.3.2}The Problem of Software Ecosystem Fragmentation}{15}

\contentsline {section}{\numberline {4.4}Data-Integration via Multi-Aspect Modules}{17}

\contentsline {subsection}{\numberline {4.4.1}Research Dissemination and Incremental Replicability}{18}
\contentsline {subsection}{\numberline {4.4.2}Heterogeneous Health Data and Data Curation}{20}
\contentsline {subsection}{\numberline {4.4.3}Modularity and the Clinical/Research Overlap}{22}


\vspace*{42pt}

\noindent{\bfseries{\Large Part II: Type Theory and Conceptual Spaces}}

\vspace*{6pt}

%\contentsline {chapter}{\numberline {3}Overview of the CORD-19 Collection}{}
%\contentsline {chapter}{\numberline {4}The Overlap Between Precision Medicine and
%Bioimaging in the Context of Covid-19 and Beyond}{}

%\contentsline {chapter}{\numberline {5}A Hypergraph-Based Theory of
%Multi-Institutional Data Sharing}{}

\contentsline {chapter}{\numberline {5}Types' Internal Structure and 
\q{Non-Constructive} (\q{NC4}) Type Theory}{}

\contentsline {section}{\numberline {5.1}Introduction}{1}

\contentsline {subsection}{\numberline {5.1.1}Cocyclic Types, Precyclic and
Endocyclic Tuples}{1}

\contentsline {subsection}{\numberline {5.1.2}Cocyclic Types for Hypernodes}{1}
\contentsline {subsection}{\numberline {5.1.3}Channelized Types and Channel Algebra}{2}
\contentsline {subsection}{\numberline {5.1.4}Constructors and Carrier States}{3}
\contentsline {subsection}{\numberline {5.1.5}Nonconstructive Type Theory}{6}

\contentsline {section}{\numberline {5.2}Types as Conceptual Structures}{8}
\contentsline {subsection}{\numberline {5.2.1}Dimensional Analysis and Axiations}{9}

\contentsline {section}{\numberline {5.3}Hypergraph Ontologies}{11}
\contentsline {subsection}{\numberline {5.3.1}Type Theoretic Foundations for
Hypergraph-Based Data Sharing}{14}
\contentsline {subsection}{\numberline {5.3.2}Hypergraphs as a Meta-Model for Data
Sharing}{15}

\vspace{5pt}

\contentsline {chapter}{\numberline {6}Using Code Models to Instantiate Data Models}{}

\contentsline {section}{\numberline {6.1}Introduction}{1}

\contentsline {section}{\numberline {6.2}Syntagmatic Graphs and Pointcut Expression Semantics}{2}

\contentsline {subsection}{\numberline {6.2.1}Query-Evaluation Foundations for Syntagmatic Graphs}{7}
\contentsline {subsection}{\numberline {6.2.2}Use-Cases for Source-Code Graphs}{8}

\contentsline {section}{\numberline {6.3}Applying Pointcut Expressions for Data Modeling}{9}
\contentsline {subsection}{\numberline {6.3.1}Code Annotation with Units of Measurement}{11}
\contentsline {subsection}{\numberline {6.3.2}Documentation by Implementation}{12}
\contentsline {subsection}{\numberline {6.3.3}Annotation-Based Reflection and Procedural Binary Equivalence}{13}
\contentsline {subsection}{\numberline {6.3.4}Meta-Procedural, Procedural, and Sub-Procedural Syntagmatic Scales}{16}
\contentsline {subsection}{\numberline {6.3.5}Case Study: Annotation and Image Markup}{16}

\contentsline {section}{\numberline {6.4}Hypergraph Representations for Data-Persistence Bridge Code}{17}
\contentsline {subsection}{\numberline {6.4.1}Multipart Relations with Roles}{20}
\contentsline {subsection}{\numberline {6.4.2}Syntagmatic Graphs and Conceptual Spaces}{22}


%\vspace*{7em}
%
%\pagebreak{}
\vspace*{3em}

\noindent{\bfseries{\Large Part III: Bioimage Annotations and Radiomics}}

\vspace*{5pt}

\contentsline {chapter}{\numberline {7}Multi-Aspect Modules and Image Annotation}{}

\contentsline {section}{\numberline {7.1}Introduction}{1}

\contentsline {subsection}{\numberline {7.1.1}Comments on Procedural and Database Aspectss}{1}

\contentsline {subsection}{\numberline {7.1.2}Assessing the Proper Scope of an
Image-Annotation Module}{2}


\contentsline {section}{\numberline {7.2}Image Annotations: Core Data Models}{4}

\contentsline {subsection}{\numberline {7.2.1}Magnitudes and Coordinates}{4}

\contentsline {subsection}{\numberline {7.2.2}Procedural and Modeling Considerations}{6}

\contentsline {subsection}{\numberline {7.2.3}Annotations with Curved Geometries or Cross-References}{7}


\contentsline {section}{\numberline {7.3}Annotations and Image Features}{8}
\contentsline {subsection}{\numberline {7.3.1}Specifying Annotations' Roles and Origins}{9}

\vspace{9pt}

\contentsline {chapter}{\numberline {8}Image Annotation as a Multi-Aspect Case-Study}{}

\contentsline {section}{\numberline {8.1}Introduction}{1}

\contentsline {subsection}{\numberline {8.1.1}Design Questions for Image-Annotation Modules}{1}

\contentsline {subsection}{\numberline {8.1.2}Procedural Data Modeling (and
the limitations of Ontologies)}{3}
\contentsline {subsection}{\numberline {8.1.3}Different Aspects of Image-Annotation Data}{5}

\contentsline {section}{\numberline {8.2}Annotations and Radiomics}{7}

\contentsline {subsection}{\numberline {8.2.1}GUI Operations Involving Images and Image-Annotations}{7}
\contentsline {subsection}{\numberline {8.2.2}Image Processing in the Context
of Broader-Scale Workflows}{9}

\contentsline {subsection}{\numberline {8.2.3}Data Profiles for Annotation and Image Markup}{11}
\contentsline {subsection}{\numberline {8.2.4}Tradeoffs Between Data Models' Narrower and Wider Scope}{13}

\vspace{9pt}

\contentsline {chapter}{\numberline {9}Conceptual Spaces and
Scientific Data Models}{}

\contentsline {section}{\numberline {9.1}Introduction}{1}

\contentsline {section}{\numberline {9.2}Verb-Centric Grammars and Information-Delta Paths}{2}

\contentsline {subsection}{\numberline {9.2.1}The Emergent Syntax/Semantics Interface}{7}


\contentsline {section}{\numberline {9.3}Conceptual and Thematic Roles}{7}

\contentsline {subsection}{\numberline {9.3.1}Disjoint Conceptual Spaces}{9}

\contentsline {subsection}{\numberline {9.3.2}Conceptual Spaces and Scientific Data}{14}


\contentsline {section}{\numberline {9.4}Delta Roles and Conceptual Space Markup Language}{15}

\contentsline {subsection}{\numberline {9.4.1}Information Delta and Data Modeling}{15}

\contentsline {subsection}{\numberline {9.4.2}The Artificiality of Data Semantics}{17}

\contentsline {section}{\numberline {9.5}Conclusion: Toward a Scientific Data
Semantics}{18}

\contentsline {subsection}{\numberline {9.5.1}Research Data and Data Integration}{19}

\contentsline {subsection}{\numberline {9.5.2}Toward a Procedural Conceptual-Space 
Semantics}{20}



\end{document}

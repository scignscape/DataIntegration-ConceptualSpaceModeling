
\section{Text and Data Mining via CORD-19}

\p{The third methodological area we will concentrate 
on in this chapter is text and data mining, 
using analyses of large-scale document corpora 
and/or biomedical data sets to discover connections 
between research work which might be less evident 
to individuals reading publications in isolation.  
As an example of the possibilities and challenges 
of text and data mining scenarios we wilk 
consider \lCnineteen{} is a collection of research articles 
about Covid-19 which was developed (starting 
in Spring 2020) in conjunction with a White 
House \q{call to action} to spur Covid-19 research.}

\p{This specific White House initiative was described as a 
\q{call to action ... to develop new text and data mining techniques that can help the science community answer high-priority scientific questions related to COVID}.\footnote{\i{See} \href{https://www.whitehouse.gov/briefings-statements/call-action-tech-community-new-machine-readable-covid-19-dataset/}{https://www.whitehouse.gov/briefings-statements/call-action-tech-community-new-machine-readable-covid-19-dataset/}}  As raw data for this initiative, the US government helped
spearhead a consortium of industry and academic 
institutions, headed by the Allen Institute for AI Research, 
who curated a \q{machine-readable Coronavirus literature collection} 
which includes article metadata and (in most cases) 
publication text  
for over 280,000 coronavirus research papers (as of 
mid-2021) \cite{LucyLuWang}.  This 
corpus is paired with links to publisher portals 
(including Springer Nature, 
Wiley, Elsevier, the American Society for Microbiology, and the New England Journal of Medicine) providing full open access to \Covid{}-related 
literature; these resources collectively constitute 
\Cnineteen{} (the \q{Covid-19 Open Research Dataset}).}

\p{The \Cnineteen{} 
collection was formulated with the explicit goal 
of promoting both text mining and data mining solutions 
to advance coronavirus research.  This means that 
\Cnineteen{} is intended to be used both as a document 
archive for text mining and as a repository for 
finding and obtaining coronavirus data for subsequent 
research.  The White House announcement directly requests 
institutions to develop \textit{additional} technologies 
which would help scientists and jurisdictions to 
take advantage of \Cnineteen{} as it was initially 
published.  In short, \Cnineteen{} was released with the 
explicit anticipation that industry and academia would 
augment the underlying data by layering on additional 
software.}

\p{Despite the obvious benefit to researchers, the 
health-care community, and the public at large 
in publishers choosing to release a substantial 
quantity of Covid-19 related literature in 
Open-Access fashion, \Cnineteen{} is not without 
certain limitations.  These largely stem from 
how the articles are encoded into an ostensibly 
(\JSON{}-based) machine-readable format.  To be 
fair, the problems we identify here reflect the 
current authors' own personal assessments of 
the \Cnineteen{} corpus; they are not broad 
criticisms which have been asserted by researchers 
working directly with \Cnineteen{} or discussed in 
peer-reviewed literature.  With that caveat, however, 
we assert that certain issues deserve mention: 

%\vspace{-2em}
\begin{description}
	
\item[Transription Errors]  
Transcription errors can cause the machine-readable 
text archive to misrepresent the structure 
and content of documents, hindering text-mining 
technology that targets the archive.  In \Cnineteen{}, 
for instance, there are cases of scientific notation and terminology 
being improperly encoded.  As a concrete example, \colorq{2{\textquotesingle}-C-ethynyl} is encoded in \Cnineteen{} as \colorq{2 0 -C-ethynyl}, 
which could stymie text searches 
against the \Cnineteen{} corpus (see \cite{Eyer} for 
the human-readable publication where this error is 
observed; the corresponding index in the corpus is 9555f44156bc5f2c6ac191dda2fb651501a7bd7b.json).

\item[Poorly Indexed Research Data]  Although 
\Cnineteen{} provides a structured representation 
of a large collection of research 
\textit{papers}, there is no easy-to-use tool 
for finding research \textit{data} 
through \Cnineteen{}.

\item[Poorly Integrated Research Data]  The 
research data which \textit{can} be 
accessed through \Cnineteen{} evinces 
a wide variety of technical fields 
and formats, with distinct software 
requirements; as a result, it is a 
difficult task to merge and integrate 
different data sets related to 
\Covid{}.  At present, \Cnineteen{} 
does not include any software tools 
or computer code that would facilitate 
data integration.   

\item[Disconnect Between Text Data and Publisher Portals]  
Although most of the \Cnineteen{} manuscripts represent 
peer-reviewed literature which can be accessed through 
document portals (for instance, the 
National Center for Biotechnology Information website), 
the \Cnineteen{} archival schema does not represent 
these links (except indirectly via Document Object 
Identifiers).  As such, there is no easy way for 
researchers to find and read publications which have been 
flagged by text-mining algorithms as being potentially 
of interest to them.  Furthermore, there is no direct 
mechanism to enlarge the \Cnineteen{} corpus with papers 
newly added to publisher portals.
\end{description}}

\p{To clarify the final comment: the Allen Institute for AI, which 
curated \Cnineteen{}, encourages publishers to contribute new 
(or newly-available) articles to the corpus.  However, integration 
with \Cnineteen{} is not developed as a formal step 
in the publication workflow.  In particular, publishers 
are not themselves generating machine-readable document infosets 
that can be integrated with the \Cnineteen{} schema (which, 
in turn, causes transcriptions errors and other problems 
as just outlined).}



\p{With respect to text mining, an immediate problem arises 
in \Cnineteen{}'s archive-construction methodology: 
especially, how the text was parsed from \PDF{} files.  
This is a process 
which almost inevitably causes imprecise or inaccurate text representation, which can degrading the quality of 
the archive unless manual or automated corrections are made.  In particular, the \Cnineteen{} library evinces  
transcription errors, as mentioned above (especially in relation to 
technical or scientific phrases and terminology); 
scientific notation in particular may be improperly 
encoded.  Moreover, there is no 
semantic marking identifying that (say) the \colorq{2 0 -C-ethynyl} 
text segment has a specific technical meaning.  These 
errors or limitations arise in part from unavoidable 
anomalies which occur when reading texts from \PDF{} files 
rather than from machine-readable, structured 
formats such as \XML{}.}
 
\p{It is also worth observing that the \JSON{} format used 
for encoding \Cnineteen{} manuscripts presents some 
logistical challenges for any operations related to 
text-mining or to cross-referencing publications and 
data sets.  In particular, \Cnineteen{} makes 
partial use of \q{standoff annotation}; specifically, 
document features such as citations and references are 
notated through character offsets into the paragraph where 
they appear.  As a result, accurately reading these document 
elements requires synthesizing data points parsed from several 
distinct objects in the \JSON{} code, which is only feasible given 
a client library built to interface with the \Cnineteen{} files 
in accord with their specific schema.  Such a client 
library would implement convenience procedures to handle 
recurring tasks, such as obtaining the full bibliographic 
reference affixed to a given location in a manuscript.}

\p{With respect to \textit{data} mining in the \Cnineteen{} 
context, the limitations in the currently available raw 
\Cnineteen{} data are even more pronounced than in the 
context of text mining.  In particular, 
neither the article 
metadata nor the full open-access document collections have 
any mechanism for actually obtaining data published 
alongside research papers, or even identifying which papers 
have accompanying data in the first place.  The Springer 
Nature collection which was originally one component within \Cnineteen{} illustrates the limitations of this relatively unstructured 
data-publishing approach (this following analysis will 
focus on Springer Nature, but the problems identified 
are no less pronounced on the other \Cnineteen{} portals 
--- 
if anything, because Springer Nature 
allows readers to browse articles in \HTML{} within the 
web portal directly, one can ascertain whether research data 
exists for an article without downloading and reading 
it; with other \Cnineteen{} resources it is actually 
harder to locate supplemental data when available).  
Initially, the Springer Nature portal 
encompassed 43 articles, 
of which 15 were accompanied by research data that 
could be separately downloaded (this number does 
not include papers that document research 
findings only indirectly, via tables or graphics 
printed inline with the text).  Collectively 
these articles referenced over 30 distinct data 
sets (some papers were linked to multiple 
data sets), forming a data collection which could 
be a valuable resource for \Covid{} research --- 
not only through the raw data made available 
but as a kernel around which new coronavirus data 
could accumulate.  However, there is currently no 
mechanism to make this overall collection available 
as a single resource.\footnote{As \Cnineteen{} has 
evolved, the publisher-specific sections therein 
appear to be merged into portals such as 
Springer Nature directly, so our above comments 
based on isolating Springer Nature articles are 
probably more applicable to the original 
archive design than the current technology.  However, 
insofar as the current portal simply defers to 
publisher-specific search features, we would argue that 
accessing Covid-19 data sets through \Cnineteen{} 
is if anything more difficult than before.}}

\p{This problem demonstrates, among other things, 
how document-metadata formats such as 
the Research Object protocol are limited in applying 
only to \textit{single} articles.  As a result, there is no 
commensurate protocol for publishing \textit{groups} 
of articles which are tied to groups of data sets unified 
into an integral whole.  Open-access \Covid{} papers also 
reveal limitations of exiting online document portals, 
especially with respect to how publications are 
linked to data sets.  In particular, there is no clear indication that a 
given paper is associated with downloadable data; usually 
readers ascertain this information only by reading or 
scrolling down to a \q{supplemental materials} or 
\q{data availability} addendum near the end of the article.  
Moreover, because the Springer Nature portal (and similar 
publisher resources) aggregates 
papers from multiple sources, there is no consistent 
pattern for locating data sets; each journal or 
publisher has their own mechanism for alerting readers 
to the existence of open-access data and citing where 
they could be downloaded.}

\subsection{Data Integration within CORD-19}
\p{Aside from the issues which are likely to hinder text and data mining 
across \Cnineteen{}, the 
collective group of \Covid{} data sets also illustrates 
the limitations of information spaces pieced 
together from disconnected raw data files with little 
additional curation.  The files included in this 
group of data sets encompass a wide array of file types 
and formats, including \FASTA{} (which stands for Fast-All, 
a genomics format), \SRA{} (Sequence Read Archive, for 
\DNA{} sequencing), \PDB{} (Protein Data Bank,  
representing the \ThreeD{} geometry of protein 
molecules), \MAP{} (Electron Microscopy Map), \EPS{} 
(Embedded Postscript), and \CSV{} (comma-separated values).  
There are also tables represented in Microsoft Word 
or Excel formats.  Although these various formats are 
reasonable for storing raw data, not all of them 
are actually machine-readable; in particular, 
the \EPS{}, Word, and Excel files need manual processing 
in order to use the information they provide in a 
computational manner.  A properly curated data collection 
would need to unify disparate sources 
into a common machine-readable representation (such as \XML{}).}

\p{Going further, productive data curation should also 
aspire to \textit{semantic} integration, unifying disparate 
sources into a common data model.  For example, multiple 
spreadsheets among the Springer Nature \Covid{} data sets 
hold sociodemographic and epidemiological information relevant 
to modeling the spread of the disease.  These different 
sources could certainly be integrated into a canonical 
social-epidemiology-based representational paradigm which 
recognizes the disparate data points which might be 
relevant for tracking the spread of \Covid{} (with the 
potential to unify data from many countries and 
jurisdictions).}

\p{This is not only an issue of data
\textit{representation} (viz., how data is physically 
laid out), but also of data types and computer code.  
According to the Research Object protocol, 
data sets should include a code base 
which provides convenient computational access to the 
published data.  In the case of \Covid{}, this entails 
creating a sociodemographic and epidemiological code 
library optimized for \Covid{} information, which would 
be the primary access point for researchers seeking to 
use the data which has been published in conjunction with 
the 43 manuscripts examined here that were aggregated 
on Springer Nature, along with any other coronavirus 
research which comes online.  Similar comments 
apply not only to tabular data represented in spreadsheet 
or \CSV{} form, but to more complex molecular or 
microscopy data that needs specialized scientific software 
to be properly visualized.}

\p{Considering the overall space of \Covid{} data, it is 
unavoidable that some files require special applications 
and cannot be directly merged with the overall collection.  
For instance, there is no coherent semantics for 
unifying Protein Data Bank files with sociodemographics 
and epidemiology; files of the former type have specific scientific 
uses and can only be understood by special-purpose software.  
Nevertheless, a well-curated data-set collection can 
make using such special-purpose data as convenient as 
possible.  In the case of Protein Data Bank, a downloadable 
\Covid{} archive can include source code for \IQmol{}, a 
molecular-visualization application that supports 
\PDB{} (among other file formats) and has few 
external dependencies (so it is relatively easy to 
build from source).}

\p{Indeed, a curated \Covid{} archive 
might include an enhanced version of software 
such as that \IQmol{} prioritizes 
\Covid{} research, with the goal of integrating biomolecular 
and social-epidemiological data as much as possible.  
For example, as \Covid{} potentially mutates in different 
ways in different geographic areas, it will be important 
to model the connections between \q{hard} scientific 
\Covid{} information and sociodemographics.  
As the pandemic evolves, genomic and biochemical information 
may be linked to particular strains of virus, which 
in turn are linked to sociodemographic profiles: certain 
strains will be more prevalent in certain populations.  
Consequently, models of \Covid{} variants will need to be 
formulated and then integrated with both chemical/molecular 
data and sociodemographic/epidemiological data.  Different 
\Covid{} strains then form a bridge linking these different 
sorts of information; researchers should be able to pass 
back and forth from molecular or genomic visualizations of 
\Covid{} to social-epidemiological charts and tables based 
on viral strains.  Ideally, capabilities for this 
sort of interdisciplinary data integration would be 
provided by a \Covid{} archive as enhancements to applications, 
such as \IQmol{}, that scientists would use to study the 
published data.}

\p{It is worth noting that a data-mining platform requires 
\textit{machine-readable} open-access research data, 
which is a more stringent requirement than simply publishing 
data alongside which can be understood by domain-specific 
software.  For example, radiological imaging can be a source 
of \Covid{} data insofar as patterns of lung 
scarring, such as \q{ground-glass opacity}, is a leading 
indicator of the disease.  Consequently, diagnostic 
images of \Covid{} patients are a relevant kind of 
content for inclusion in a \Covid{} data set 
(see \cite{Shi} as a case-study).  However, 
diagnostic images are not in themselves 
\q{machine readable.}  When medical imaging is 
used in a quantitative context (e.g., applying 
Machine Learning for diagnostic pathology), 
it is necessary to perform Image Analysis to convert the raw data 
(viz., in this case, radiological graphics) into 
quantitative aggregates (for instance by using image 
segmentation to demarcate geometric boundaries and 
then defining diagnostically relevant features, such 
as opacity, as a scalar field over the segments).  
In short, even after research data is openly published 
by article authors, it may be necessary to perform 
additional analysis on the data for it to be 
a full-fledged component of a 
machine-readable information space.\footnote{%
This does not mean that diagnostic images (or 
other graphical data) should not be placed in a 
data set; only that computational reuse of such 
data will usually involve certain numeric 
processing, such as image segmentation.  
Insofar as this subsequent analysis is performed, 
the resulting data should wherever possible 
be added to the underlying image data as a 
supplement to the data set.}} 

\p{Another concern in developing an integrated \Cnineteen{} 
data collection is that, logistically speaking, 
not all \Covid{} data is practical 
to reuse as a downloadable package.  This is especially 
true for genomics; several of the aforementioned 
43 coronavirus papers included data published via 
online data banks capable of hosting data sets that 
are too large for an ordinary computer.  In these 
situations scientists formulate queries or 
analytic scripts that are sent remotely to the online 
repositories, so that researchers access the actual 
published data only indirectly.  Nevertheless, access to 
these data sets can still be curated as part of 
a \Covid{} package; in particular, computer code 
can be provided which automates the process of 
networking with remote genomics archives through the 
accession numbers and file-formats which those archives 
recognize.}

\p{As a final point on the topic of integrating 
disparate \Cnineteen{} research data, note that 
an overarching framework for indexing \Covid{} data 
sets would also facilitate the process of cross-referencing 
article text and research data.  In particular, 
the annotation system employed for 
\Cnineteen{} could profitably be enhanced 
by a system of \textit{microcitations} that apply 
to portions of manuscripts \textit{as well as} data sets.  
In the publishing context, a microcitation is defined as a 
reference to a partially isolated fragment of a larger 
document, such as a table or figure illustration, or a 
sentence or paragraph defining a technical term, 
or (in mathematics) the statement/proof of a definition, axiom, 
or theorem.  In data publishing, \q{data citations} are 
unique references to data sets in their entirety or to 
smaller parts of data sets.  A data microcitation is then a 
fine-grained reference into a data set: for example, 
\q{the precise data records actually used in a study} (as 
defined by the Federation of Earth Science Information Partners; 
see \cite{ESIP}), 
one column in a spreadsheet, or one statistical parameter in a 
quantitative analysis.}

\p{Ideally, the text-mining and 
data-mining notions of microcitation should be combined into a unified 
framework.  In particular, text-based searches against 
the \Cnineteen{} corpus should also try to find matches in the 
data sets accompanying articles within the corpus.  As a concrete example, 
a concept such as \q{expiratory flow} appears in \Cnineteen{} 
both as a table column in research data and as a medical concept 
discussed in research papers; a unified microcitation framework 
should therefore map \textit{\color{drp}{expiratory flow}} as a keyphrase 
to both textual locations and data set parameters.  
Similarly, a concept such as 
\textit{\color{drp}{2{\textquotesingle}-C-ethynyl}} (mentioned earlier 
in the context of transcription errors) 
should be identified both as a phrase in 
article texts and as a molecular component 
present within compounds whose scientific 
properties are investigated through \Cnineteen{} 
research data, so that a search for this 
concept can trigger both publication and 
data-set matches. 
Implementing this kind of unified search mechanism requires that data 
sets be \textit{annotated} with techniques similar to 
those used for marking up Natural Language techniques.}


\p{Considering the inter-disciplinary nature of \Covid{} research, 
it is unavoidable that different scientists will need 
different sorts of specialized software to analyze the 
kinds of information relevant to their research.  For 
instance, the 
computational techniques applicable to diagnosing 
coronavirus infection are scientifically very different 
from those used for genomic or epidemiological studies 
of the disease; it is impractical to expect 
pathologists to use the same software as bioinformaticians 
studying the pathogen, or for either to use the same 
software as virologists modeling the (potential or observed) 
spread of the disease.  In short, even if 
scientists from disparate disciplines start with a 
common pool of raw data, they will need to 
analyze this data through a diverse set of 
supplemental computational tools, which will vary 
not only across disciplines but also in terms of 
the software and laboratory facilities available 
to different researchers through their institutions.  
In this sense it is impossible to unify all 
\Cnineteen{} data into a \textit{fully} self-contained 
information space.}

\p{Nevertheless, committing to \q{standalone} data publishing 
remains a valuable goal even in a context where published 
data sets will invariably migrate to different 
digital ecosystems.  Although scientists may use 
external digital tools \textit{when necessary} to 
perform certain calculations, or when interfacing with 
laboratory equipment, we strongly recommend that 
the degree of variation across different domain-specific extensions to 
\Cnineteen{} be greatly minimized.  
Ideally, that is, the version of \Cnineteen{} 
(along with its supporting technology) 
found in a biomedical setting should be as 
similar as possible to that found in a 
biochemical context, or a health-policy 
context.  In the absence of 
any initiative to limit this drift, \Cnineteen{} 
could easily devolve into a federation of 
separate resources which have no interconnection 
apart from their nominal focus on \Covid{}.}


\p{This section has highlighted limitations of data 
sets published in conjunction with coronavirus articles made 
available as open-access resources on Springer Nature 
(and, by extension, \Cnineteen{}).  
The central point here is to argue for a distinct data-curation 
stage in the publication process, with data curators 
playing a role distinct from that of both 
authors and editors.\footnote{%
The point here is not to critique the work of individual 
authors; curating data sets according to exacting 
scientific standards demands a separate vein of 
expertise which typically lies outside researchers' 
disciplinary scope.  The point is rather that 
publishers should recognize data curation as a 
distinct process and skill-set complementary 
to both writing and editing research works.}  
Moreover, the discussion has hopefully highlighted problems 
with current data-sharing paradigms, even those such 
as the Research Object and \FAIR{} initiatives which are 
explicitly devoted to improving how open-access data sets are 
published.  \Cnineteen{} exposes several 
lacunae in the Research Object protocol, for example, 
which point to the need for a more detailed extension 
of this protocol.  In particular, an enhanced protocol 
should encompass: 

\begin{enumerate}[leftmargin=3pt, itemsep=4pt,topsep=11pt]
\item{} A canonical framework for archiving collections 
of data sets, not only single data sets (and not only 
groups of data sets published with a single research 
paper).  For example, all data sets published alongside 
the 43 Springer Nature articles could be unified into a 
single collection.

\item{} A code base accompanying data-set collections 
designed to help research unify the information provided.  
Curating the overall collection would involve pooling 
disparate data into common representation, and 
implementing computer code which deserializes and processes 
the unified data accordingly.  For instance, \CSV{}, 
\EPS{}, and Microsoft Word/Excel tables could be migrated 
to \XML{}, \JSON{}, or 
a more complex common format.  Customized computer code could then 
be implemented specifically to parse and merge the 
information present in single data sets within the 
overall collection.  This implementation would 
reciprocate the Research Object goal of unifying 
code and data, but again would operate at the level 
of an aggregate of research projects rather than a 
single Research Object.

\item{}  A unified data-set collection should 
be self-contained as much as possible, and should be 
built around a foundation where advanced computing 
capabilities are available in a transparent, 
standalone fashion, without requiring tools 
outside the collection itself.  One way to 
achieve this is by gravitating 
toward components that can provide 
features such as scripting and data persistence 
through components that can be shared 
in pure source-code fashion, 
such as the WhiteDB database engine 
\cite{EnarReilent} and the AngelScript 
scripting language \cite{AS}.   

\item{}  A unified data-set collection should also provide 
prototyping and remote-access tools to interface with 
web-based information spaces that host data sets 
too large to be individually downloaded.  Ideally, 
these would include simulations of remote services, which 
would help scientists understand the design of 
the remote archives and how to interface with them.
These simulations could function analogously to 
(for instance) 
PurpleData, a prototyping tool for Google's BigData 
developed by Verily (the Alphabet subsidiary developing 
an online \Covid{} portal for remote diagnostics 
and disseminating public information about the pandemic). 

\item{}  Finally, a unified research portal should  
influence the design of the web portals where associated 
texts are published.  It should be easy for readers to 
identify which articles have supplemental data files and 
to download those files if desired.  Moreover, 
textual links should be established between publication 
content and data sets --- for instance, a plot or 
diagram illustrating statistical or equational distributions 
should link to the portion of the data set from which that 
quantitative data is derived.
\end{enumerate}}

\p{This discussion has used the \Covid{} crisis as a 
lens through which to examine data-publishing limitations 
in general.  Such limitations are not specific to coronavirus in particular.  
However, the nearly unprecedented urgency of this epidemic 
reveals how both the scientific and publishing industries are still struggling 
to develop technologies and practices which keep pace 
with the intersecting needs of systematic research 
and public policy.  An optimistic projection is 
that the crisis will spur momentum toward a more 
sophisticated data-sharing paradigm --- perhaps a 
generalization of the Research Object protocol 
toward data-set collections.}

\subsection{Reviewing the CORD-19 Document Model}

\p{In order to discuss the possibilities 
and limitations of \Cnineteen{} (and potentially 
other document corpora with a similar design) it 
is worth examining how \Cnineteen{} encodes textual 
data in greater detail.  This discussion has ramifications 
outside of \Cnineteen{} itself, 
insofar as \Cnineteen{} hopefully points to 
gaps in current publishing technologies.  These gaps need to be 
addressed if publishers are to curate open-access corpora 
which truly leverage the digital and interactive 
technology available to us with modern software.}

\p{The basis of \Cnineteen{}'s infrastructure 
is a \JSON{} scheme which describes the 
document hierarchy of research articles encoded 
within the corpus.  Apart from metadata 
(consisting of basic details such as document title and authors' 
names) and bibliographic entries, all document 
content according to this schema is divided into paragraphs 
(implicitly the documents are divided into sections 
as well, but sections are notated as properties 
of the paragraphs they contain, not as a separate 
level in the hierarchy).   
Each paragraph encoding contains an underlying string 
vector (a stream of characters) and, separate and apart from 
that, character \q{spans} which point to 
references (such as Named Entites), citations, and 
equations.  This indicates that the \Cnineteen{} encoding 
uses \q{standoff annotation,} where any content 
modifying the interpretation assigned to portions of 
the main text is notated with a series of data 
structures described apart from the main text itself.}

\p{Standoff text-encoding systems may be contrasted with \XML{} 
or \HTML{}, where \q{tags} are mixed with character 
data.  For example, consider a span of text which 
quotes from another document: in \HTML{}, the 
special status of the quoted text may be 
marked by surrounding the text with \textbf{<quote>} 
start and end tags.  Syntactically, this markup system 
has the effect that tags and text are seen side-by-side: any 
content governed by the \textbf{<quote>} (i.e., the text of 
the quote itself) is printed immediately after the 
begin-tag, and the quotation ends when the last character 
is followed by an end-tag (i.e., \textbf{</quote>}).}

\p{Apart from such syntactic details, the distinction 
between tag-based markup and standoff annotation 
determines the \q{semantics} of the document, insofar 
as tags form a document hierarchy.  Continuing the 
\textbf{<quote>} example, the text-span inside the 
quote tags is represented as a \textit{child element} 
of the quote, whereas the quote itself may be a child 
element of a larger-scale entity (such as a paragraph).  
In effect, the paragraph \textit{contains} a quote, 
and the quote \textit{contains} a string of characters.  
Such nested levels of containment provide the structure 
through which hierarchical documents (formats 
such as \XML{} and \HTML{}) are interpreted.}

\p{To see the contrast with \textit{standoff} annotation, 
if one were to describe a document using a standoff annotation 
system, the notation that a particular 
span of characters belongs to a quotation would 
not be marked-up amidst the characters themselves.  
Instead, the quotation-designation would use numeric indices to 
declare that the character at a certain position in the 
main text begins a quotation, and some later character 
in the text ends that quotation.  When serializing 
documents with standoff annotation, all the characters 
in a document are typically represented as one 
character-stream, and any notation describing markup applied 
to spans within that character stream is asserted afterward, 
using indexes into the stream to demarcate element boundaries.}

\p{The \JSON{} schema used for \Cnineteen{} is not 
entirely standoff, because there is a document hierarchy 
(for example, a publication's abstract is modeled as a sibling 
element to the main body text, so abstracts and the main text represent 
an intermediate hierarchical level, contained within the overall 
document and containing individual paragraphs).  However, 
\Cnineteen{} uses in effect a standoff-annotation system 
for each paragraph, so there is no hierarchical level 
smaller than paragraphs themselves, except implicitly; after the 
text (viz., the character stream) there are subsequent notations
of spans within the paragraph (each span description is considered 
a child of the paragraph itself, as is the paragraph text).}

\p{This arrangement has consequences for text mining algorithms, which 
may be strengths or weaknesses in different contexts.  
One consequence is that the raw text is all grouped together 
in one place --- algorithms do not have to tie together child 
nodes of disparate \XML{} elements to derive a beginning-to-end 
sequence of the text belonging to any paragraph.  Instead, it 
is simply necessary to read all data in the 
\q{text} field of the relevant \q{paragraph} 
object.  The character-sequence in this text may 
contain words and sentences, but potentially other 
strings of symbols (such as chemical formulae) 
which are not explicitly marked.  This may or may not 
be desirable.  It could potentially complicate 
\NLP{} tasks, because the language-processing 
components will be fed not only sequences of English 
words but also, sometimes interspersed among ordinary 
words, technical symbol-sequences such as 
\colorq{2{\textquotesingle}-C-ethynyl} (an 
example used earlier in this chapter).  
Standoff annotations may or may not be 
effective in marking the boundaries of such 
extra-lexical sequences; certainly we 
cannot rely on Named Entity detectors to 
properly identify and demarcate the boundaries 
of all uses of technical terminology or special 
symbols (again, the limitations of automated 
annotation are discussed earlier in this chapter).}

\p{In discussing standoff annotation it is also 
worth considering how the text of \Cnineteen{} 
publications was obtained.  According to 
\Cnineteen{} documentation, most full-text 
transcriptions in the corpus were obtained from 
\PDF{} files, via a pipeline using \TEI{} (Text 
Encoding Initiative) \XML{} as an intermediate 
representation.  Necessarily, then the 
encoded text is only an approximate representation of 
the original:  

\begin{displayquote}
To provide accessible and canonical structured full text, 
we parse content from PDFs and associated paper documents.
The full text is presented in a JSON schema designed 
to preserve most relevant paper structures 
such as paragraph breaks, section headers, and 
inline references and citations. ... We recognize that 
converting between PDF or XML to JSON is lossy.
However, the benefits of a standard structured format, 
and the ability to reuse and share annotations
made on top of that format have been critical to the
success of CORD-19. ... 
Though we have made the structured full text
of many scientific papers available to researchers
through CORD-19, a number of challenges prevent 
easy application of NLP and text mining techniques 
to these papers.  First, the primary distribution 
format of scientific papers --- PDF --- is not
amenable to text processing. The PDF file format
is designed to share electronic documents rendered
faithfully for reading and printing, not for 
automated analysis of document content.  Paper content
(text, images, bibliography) and metadata extracted
from PDF are imperfect and require significant
cleaning before they can be used for analysis.
Second, there is a clear need for more scientific
content to be made easily accessible to researchers.
Though many publishers have generously made
COVID-19 papers available during this time, there
are still bottlenecks to information access. ... 
Lastly, there is no standard format for 
representing paper metadata.  Existing schemas like 
... JATS[,] Crossref [or] Dublin Core have
been adopted as representations for paper metadata.
However, there are issues with these standards; they
can be too coarse-grained to capture all necessary
paper metadata elements, or lack a strict schema. 
... Without solutions to the above problems, NLP
on COVID-19 research and scientific research in
general will remain difficult. \cite[page 6]{LucyLuWang}
\end{displayquote}

As an example of these \NLP{} issues, 
consider the challenge of demarcating all 
named entities, particularly technical 
character-sequences (such as chemical 
formulae) which are not ordinary lexemes.  
Whether or not authors explicitly mark up 
such sequences (they may well do so in that 
formulae or equations are often typeset 
differently than normal text) this markup is 
not preserved in \PDF{} versions of articles.  
As the authors of the last-cited article point 
out, many (roughly 38\%) of papers in \Cnineteen{} 
are also available in the \JATS{} (Journal Article Tag Suite) 
format, which is a more precise text encoding 
than \PDF{}.  However, even in this context \JATS{} 
does not compel authors to explicitly notate 
textual entities such as special terms or 
character-sequences --- in fact \JATS{} 
does not truly have an obvious structure or 
set of alternative structures for identifying
what would normally be considered annotation-worthy  
text spans or named entities; the closest correlates 
are probably the generic \textbf{<kwd>} (keyword) and 
\textbf{<abbrev>} (abbreviation) tags as well 
as discipline-specific options such as 
\textbf{<chem-struct>} (for chemical structures) 
and \textbf{<disp-formula>} (for mathematical expressions).  
In short, building a corpus such as \Cnineteen{} for 
rigorous text-mining is made more difficult because 
authors and publishers do not publish texts in 
formats which are optimized for text mining in the 
first place; the acknowledged limitations of 
\Cnineteen{} reflect problems of industry 
practice, not programming lacunae that could 
be alleviated with more sophisticated \NLP{} algorithms.}

\p{Having acknowledged these limitations, a discussion of 
document corpora could then reasonably pivot from 
the empirical goal of curating useful text archives from 
currently published text to examining how more 
sophisticated corpora may be published in the future.  
It is reasonable, for example, to propose 
that full-text publications be released 
\textit{both} in reader-friendly \PDF{} form 
\textit{and} in machine-readable forms such as 
\JATS{}.  This is not just an abstract proposal; 
indeed, the text of this very book has been prepared 
using a novel document-generation system 
which creates both machine-readable structured text 
and \PDF{} output, moreover with cross-referencing 
between them; notably, the positions of discursively 
important textual markers, such as sentence boundaries, 
are mapped to \PDF{} screen coordinates (the code library 
for the book includes document-generation code as well as 
the data set of coordinate positions generated 
as part of the book's publication workflow).  
In particular, it is reasonable for authors 
and editors to manually introduce textual 
annotations for content such as named entities, 
keywords, important technical terms, and other 
content which should be targeted by \NLP{} engines 
separate and apart from ordinary lexemes with 
their conventional natural-language semantics.  
Typically such specialized terms/lexemes would 
be marked up in any case because they may require 
distinct fonts or styling than their surroundings.  
It is also reasonable to manually define sentence 
boundaries via simple rules (e.g. two following spaces 
mark the end of a sentence; a single space, such as 
that following an abbreviation, indicates situations 
where a character such as a period, which could potentially mark the 
end of a sentence, is actually playing a different discursive role).}

\p{By following simple rules of document content-entry and 
lexicography, certain \NLP{} tasks, such as sentence-boundary 
and Named Entity recognition, can be optimized --- eliminating 
the need for probabilistic algorithms and relying instead 
on much less sophisticated, but more accurate, markup-parsing 
logic.  If sentence boundaries and Named Entities are 
explicitly annotated in machine-readable text encodings, then 
extracting these features is not really an issue of 
\q{Natural Language Processing} as such.  
On the other hand, \AI{}-driven analysis of document corpora 
would still require \NLP{} for other aspects of 
parsing documents; it is unreasonable to expect authors, 
for instance, to manually notate sentence parse-graphs.  
This then suggests the question of where the boundary 
lies between discursive structures which might reasonably 
be left to authors or editors to manually notate (e.g. 
sentence boundaries) and those which in practice could 
only be obtained via \NLP{} (such as part-of-speech 
tags).  Related to this question is how best to 
model \NLP{} structures, such as the trees or graphs 
representing the syntax of natural-language sentences.  
We will consider this question in subsequent chapters 
in the context of Conceptual Space Theory.}


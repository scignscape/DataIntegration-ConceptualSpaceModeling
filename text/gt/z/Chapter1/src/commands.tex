
\usepackage{graphicx}
\usepackage{xcolor}

\setlength{\columnsep}{8.5mm}

\usepackage[letterpaper, left=.84in,right=.83in,top=.5in,bottom=.56in, 
paperheight=12.5in,paperwidth=9.5in]{geometry}

%\usepackage[letterpaper, left=.925in,right=.925in,top=.56in,bottom=.56in, 
%paperheight=12.5in,paperwidth=9.5in]{geometry}


\usepackage{enumitem}

\setlist[description]{leftmargin=4pt}


\newcommand{\mdash}{---}
\newcommand{\q}[1]{{\fontfamily{phv}\selectfont ``}#1{\fontfamily{phv}\selectfont ''}} 
\newcommand{\sq}[1]{{\fontfamily{phv}\selectfont `}#1{\fontfamily{phv}\selectfont '}} 

\newcommand{\JaneJohn}{\i{Jane}\hspace{4pt}and\hspace{3pt}\i{John}}


\newcommand{\qb}[1]{\q{\b{#1}}}
\newcommand{\qbcomma}[1]{\q{\b{#1},}}


\colorlet{codegr}{black!80!blue}

\let\OldFootnoteSize\footnotesize
\renewcommand{\footnotesize}{\scriptsize}

\newif\iffootnote
\let\Footnote\footnote
\renewcommand\footnote[1]{\begingroup\footnotetrue\Footnote{#1}\endgroup}

\newcommand{\AcronymText}[1]{{\iffootnote\begin{footnotesize}{\textsc{#1}}\end{footnotesize}%
\else\begin{OldFootnoteSize}{\textsc{#1}}\end{OldFootnoteSize}\fi}}

\newcommand{\JSON}{\AcronymText{JSON}}

\newcommand{\SQL}{\AcronymText{SQL}}
\newcommand{\Cpp}{\AcronymText{C++}}
\newcommand{\RDF}{\AcronymText{RDF}}

\newcommand{\Lisp}{Lisp}


\newcommand{\SDL}{\AcronymText{SDL}}
\newcommand{\RPC}{\AcronymText{RPC}}
\newcommand{\API}{\AcronymText{API}}
\newcommand{\HTTP}{\AcronymText{HTTP}}
\newcommand{\Java}{\AcronymText{Java}}


\newcommand{\JSON}{\AcronymText{JSON}}

\newcommand{\HTML}{\AcronymText{HTML}}


\newcommand{\GUI}{\AcronymText{GUI}}

\newcommand{\NoSQL}{\AcronymText{NoSQL}}
\newcommand{\Qt}{\AcronymText{Qt}}

\newcommand{\EHR}{\AcronymText{EHR}}
\newcommand{\ABI}{\AcronymText{ABI}}
\newcommand{\GCC}{\AcronymText{GCC}}

\newcommand{\RoI}{\AcronymText{RoI}}

\newcommand{\ECL}{\AcronymText{ECL}}
\newcommand{\C}{\AcronymText{C}}
\newcommand{\Clasp}{Clasp}
\newcommand{\CSharp}{\AcronymText{C\#}}

\newcommand{\LLVM}{\AcronymText{LLVM}}

\newcommand{\XML}{\AcronymText{XML}}
\newcommand{\DTD}{\AcronymText{DTD}}
\newcommand{\PDF}{\AcronymText{PDF}}
\newcommand{\IDE}{\AcronymText{IDE}}

\newcommand{\visavis}{\makebox{vis-\`a-vis}} 
\newcommand{\NLP}{\AcronymText{NLP}}
\newcommand{\AI}{\AcronymText{AI}}

\newcommand{\Gardenfors}{G\"ardenfors}



\newcommand{\TCIA}{\AcronymText{TCIA}}

\newcommand{\AIMLib}{\AcronymText{AIMLib}}

\newcommand{\MatiasOstaVelez}{\AcronymText{Mat\'ias Osta V\'elez}}

\newcommand{\AIM}{\AcronymText{AIM}}
\newcommand{\caBIG}{\AcronymText{caBIG}}


\newcommand{\OMOP}{\AcronymText{OMOP}}
\newcommand{\PCOR}{\AcronymText{PCOR}}
\newcommand{\CDISC}{\AcronymText{CDISC}}

\newcommand{\CSV}{\AcronymText{CSV}}
\newcommand{\OBO}{\AcronymText{OBO}}

\newcommand{\RNA}{\AcronymText{RNA}}
\newcommand{\mRNA}{\AcronymText{RNA}}


\newcommand{\HDF}{\AcronymText{HDF}}
\newcommand{\ASDF}{\AcronymText{ASDF}}


%\newcommand{...}

\usepackage{letterspace}

\PassOptionsToPackage{hyphens}{url}
\usepackage[colorlinks=true]{hyperref}

\usepackage[preserveurlmacro]{breakurl}

\newcommand{\bhref}[1]{\href{#1}{\burl{#1}}}


\colorlet{urlclr}{red!10!magenta!70!orange}

\hypersetup{
 urlcolor = urlclr!50!black,
 urlbordercolor = cyan!60!black,
 linkcolor = red!30!black,
 citecolor = orange!30!black,
 citebordercolor = yellow!30!black,
} 

\newcommand{\biburl}[1]{ {\fontfamily{gar}\selectfont{\textcolor[rgb]{.2,.6,0}%
			{\scriptsize \textls*[-70]{\url{#1}}}}}}


\newcommand{\subsectionbox}[1]{\subsection{\makebox{#1}}}

\newcommand{\bibtitle}[1]{{\small \textit{#1}}}
\newcommand{\intitle}[1]{{\hspace{3pt}\textls*[-80]{\texttt{\textit{#1}}}}\hspace{-1pt}}

\newcommand{\hoctitle}[1]{\noindent{}{\fontsize{14}{18}\selectfont\textbf{#1}

}\vspace{1em}}

\usepackage{xfrac}

\usepackage[T1]{fontenc}

\let\OldI\i
\renewcommand{\i}[1]{\textit{#1}}


\makeatletter
\newcommand{\oneptsmaller}[1]{%
  \begingroup
  \fontsize{\dimexpr\f@size pt-1pt}{\f@baselineskip}\selectfont
  #1%
  \endgroup
}
\makeatother

\let\OldB\b
\newcommand{\bb}[1]{\textbf{\oneptsmaller{#1}}}
\newcommand{\bx}[1]{\textbf{\oneptsmaller{#1}}}
\renewcommand{\b}[1]{\makebox{\bb{#1}}}


\newcommand{\p}[1]{

\vspace{.65em}#1}

\newcommand{\nl}[1]{\\

}

\usepackage[flushmargin]{footmisc}

\usepackage{setspace}
\usepackage{tikz}
\usetikzlibrary{shapes,positioning}
\usetikzlibrary{arrows}
%\usetikzlibrary{decorations.pathmorphing}
\usetikzlibrary{arrows.meta}

%\usepackage{tkz-euclide}
\usetikzlibrary{fit}


\usepackage{pbox}
\usepackage{soul}

\usetikzlibrary{matrix}

\usepackage{amsmath}

\usepackage[raggedright]{titlesec}
\titlespacing*{\section}
{0pt}{3.5ex}{0ex}

\titlespacing*{\subsection}
{0pt}{3ex}{-.5ex}

\titleformat{\section}
  {\raggedright\normalfont\fontsize{11}{13}\bfseries}{\thesection}{1em}{}

\titleformat{\subsection}
  {\raggedright\normalfont\fontsize{10}{12}\bfseries}{\thesubsection}{.5em}{}



%\usepackage{sectsty}

%\sectionfont{\fontsize{11}{15}\selectfont}

%\sectionfont{\Large}
%\subsectionfont{\large}

\newcommand{\absolutelynopagebreakstart}{\par\nobreak\vfil\penalty0\vfilneg\vtop}
\newcommand{\absolutelynopagebreakend}{\par\xdef\tpd{\the\prevdepth}\prevdepth=\tpd}

\raggedbottom{}




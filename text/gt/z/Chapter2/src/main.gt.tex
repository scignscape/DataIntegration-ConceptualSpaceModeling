
\p{\:\+Covid-19, cancer, and Cardiac Care are among
the most significant health-care challenges
of our time. \> The devastation wrought by
the SARS-CoV-2 pandemic, both in terms of
lives lost and global economic impact, is
well-known. \> Cancer and Heart Disease are
if anything even deadlier. \> Covid-19 erupted into
the world's attention in early 2020; cancer and
Heart Disease have not had their own
single crisis moment,
but they have been at the forefront of scientific
attention for decades, spurring treatment
innovations and scientific breakthroughs.\;\<
}

\p{\:\+Covid-19 is a coronavirus very similar to the
SARS strain that caused many deaths in 2004;
like all viruses in their class, these pathogens
use spike proteins to invade the human host's
respiratory system, spreading to the lungs and
sometimes affecting cardiac and neurological
functioning as well. \> Many cases of Covid-19
are mild or asymptomatic, but for still-obscure
reasons about 16\% of infected individuals
require hospitalization, including a 3-4\% mortality
rate. \> All this is widely known, reported in everyday
newspapers and web sites as well as scientific journals.\;\<
}

\p{\:\+But how \i{do} we know these details? \> Different facets of
Covid-19 \mdash{} its genetics, proteins, biomechanics,
epidemiology, mutations, treatment options, and so forth
\mdash{} are the province of different biomedical
specializations. \> The complete picture of
Covid-19 therefore has to be assembled from many different
lines of research. \> Much information about the
virus was acquired soon after the start of the pandemic,
and the development of effective vaccines has progressed
at a torrid pace. \> Such accelerated results,
unprecedented in medical history, reflect well on the
current state of biomedical knowledge and competence
of the practitioners of biomedical sciences worldwide.
\> While Covid-19 data is imperfect, and much remains
mysterious about this disease, scientists' opinions should
generally be taken seriously (and not politicized or
obfuscated for ideological reasons).\;\<
}

\p{\:\+Nevertheless, there is no harm in trying to get a
deeper understanding of \i{how} Covid-19 science
has progressed so quickly. \> For anyone approaching
Covid-19 from an angle outside of biology or
health care proper \mdash{} be that public policy,
ethics, economics, sociology, or technology
(and so on) \mdash{} trying to understand
the chain of scientific reasoning is not a matter
of pre-empting scientific expertise, but
of trying to comprehend the Covid-19 crisis
in a well-informed manner. \> What were the crucial
insights gleaned from SARS-CoV-2 genomics, morphology,
and biomechanics, as well as from contact tracing
and clinical observations, which allowed a
useful picture of Covid-19 to fit together in
just a few months in 2020? \> How does the
cryptic numeric data driving models of SARS-CoV-2's
genes, proteins, or anatomy get translated
to intuitive pictures of the virus as an active biological
agent, with its specific infectious tendencies
and deadly effects on human hosts?\;\<
}

\p{\:\+To be sure, a thorough recount of the scientific history
unfolding through the Covid-19 pandemic would
require a much longer book than this one.
\> We will not, in other words, directly attempt
to excavate scientists' detective work
in early 2020 in the manner we just hypothesized,
however much that is a story that inevitably
\i{will} get told sometime after the pandemic
has receded. \> But we \i{can} touch on one
very specific field within this overarching
trajectory, namely the role of
computer software and the kind of data
which forms a first step toward comprehensive
models of diseases (and their treatments and interventions).
\> This applies, of course, to cancer/oncology and
Cardiac Care as much as to Covid-19.\;\<
}

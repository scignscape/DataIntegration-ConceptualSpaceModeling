
\section{Personalized Medicine in the Context of Covid-19}
\p{Within just a few months after Covid-19 became a global pandemic, 
doctors and researchers were developing and experimenting with a 
diverse range of treatments and remedies against SARS-CoV-2.  
The treatments which were in some sense 
tested or adopted include Monoclonal Antibodies, 
remdesivir, dexamethasone and other steroids, anti-inflamatory drugs
(such as tocilizumab and sarilumab), 
convalescent plasma, baricitinib (a rheumatoid arthritis 
drug for which Covid-19 is an off-label use), and 
anti-malarial drugs (specifically chloroquine and 
hydroxychloroquine).\footnote{See \bhref{https://www.mayoclinic.org/diseases-conditions/coronavirus/expert-answers/coronavirus-drugs/faq-20485627}.}  
These treatments, that is to say, were either approved (albeit 
provisionally or \q{for emergency use}) by the United States \FDA{} and other 
government agencies, pending clinical trials, 
or reported anecdotally to be helpful in combating Covid-19.  In either 
case they were relatively widely adopted in clinical settings.  
More rigorous attempts to validate claims of any treatments' 
effectiveness, however, have proven inconclusive (at least 
as of mid-2021) --- certainly 
no pharmaceutical or clinical intervention has demonstrated success 
rates comparable to the first wave of vaccines that were approved 
toward the end of 2020.  In short, every well-studied Covid-19 intervention 
(other than vaccination) appears to be successful in some patients and 
not others.  These mixed results generate questions of their 
own --- what are the biologic mechanisms that 
cause different treatments to play out differently 
in distinct patient populations?}

\p{Of course, it is possible that associations between 
treatments and outcomes --- at least in the SARS-CoV-2 context 
--- is driven mostly by random chance, or that treatments 
are correlated to, rather than causative of, outcomes.  
The Covid-19 mortality rate among hospitalized patients 
appears on average to be less than 25\% (even though  
over four million people worldwide have died from 
the pandemic, as of summer 2021).  Accordingly, many people 
even with serious cases of Covid-19 will recover just 
via statistical distribution.  While it 
is possible that treatments received in 
hospitals account for some of the recovery rate ---  
the mortality rate amongst hospitalized patients has diminished 
over time, which implies that hospitals have become more 
skilled in caring for Covid-19 patients and that treatment 
plans have become more refined --- many 
patients hospitalized for the disease might well have recovered 
anyhow (even without additional pharmaceutical or clinical 
interventions).  
Given this interpretive uncertainty, it has been difficult 
for researchers to definitively show that particular 
treatments do in fact improve recipients' chances of 
recovering from Covid-19.  Nevertheless, both observational evidence and 
clinical trials suggest that a number of Covid-19 treatments 
do have some positive benefits, even if not for all patients.}

\p{This situation then leads to the question of why 
particular treatments benefit some patients more than 
others.  Such considerations are of scientific interest, 
of course --- clarifying how drugs or other interventions 
interact with the SARS-CoV-2 virus enhances our knowledge 
about its infectious mechanisms and the progression of 
Covid-19 --- but more immediately it is of clinical 
interest, because doctors need guidelines on 
when to administer which treatment(s) to which individual 
patients.  Since no Covid-19 remedy (apart from vaccination) clearly 
outperforms others under most circumstances, scientists have 
attempted to study Covid-19 treatments with the goal of 
giving doctors more detailed information to work 
with when making these sorts of clinical decisions.  The overarching 
project of a significant subset of Covid-19 research, 
in short, has been to establish criteria allowing doctors 
to predict which treatments are most likely to succeed 
for individual patients, given details of their 
immunological profile and prior medical history.}

\p{In this sense the goals of Covid-19 research overlap with 
methodologies that have been established in the context 
of other research areas --- notably cancer and AIDS --- 
under the general rubric of \q{Precision Medicine.}  
As a field of research, Precision Medicine is focused on 
the correlation between patient-specific biomedical details 
and the likelihood that particular interventions will have 
positive effects for particular patients.  In contrast to 
conventional clinical trials, which consider patient-specific details 
only at a rather coarse level --- merely observing obvious 
data-points such as age, gender, and ethnicity --- precision-medicine 
research attempts to curate a much more detailed picture of 
patients' medical and sociodemographic profiles, either as 
part of formal trials or as observational details in a clinical 
setting.  Ideally, Precision Medicine is motivated by the 
goal of analyzing patient-profile-to-outcomes correlations 
not only retroactively (detecting statistical patterns in 
prior outcomes) but also prognostically.  In effect, once a 
particular sort of treatment has been identified as 
especially favorable for patients with certain characteristics, 
it is reasonable to project that future patients with similar 
characteristics should be given similar treatments.  
This intuitive and obvious point, however, masks empirical 
and practical difficulties --- in particular, it is not 
obvious how to quantify the notion that current patients 
are \q{similar} to prior ones.  Indeed, one of the 
provisional results of precision-medicine research thus far 
has been to highlight how statistically significant points of 
resemblance between patient profiles do not necessarily line 
up with how we \textit{intuitively} group patients together by 
visible factors such as age, race, or gender.}

\p{In sum, Precision Medicine has been guided by the thesis that 
compiling detailed patient-centered information can advance 
clinical medicine by identifying which treatment options 
are more likely to succeed for individual patients.  In the 
context of Covid-19, both the diversity of legitimate 
clinical interventions and the failure of any one 
treatment to show unambiguous success for a broad spectrum of 
patients point to the potential usefulness of patient-centered 
approaches.  If doctors have detailed information about 
Covid-19 patients' immunological profiles and 
clinical history they can --- at least in theory --- make 
informed decisions about which treatment options have a 
higher probability of success.  Such predictions would 
not be a matter of guesswork; instead, statistical analysis 
of Covid-19 cases in the past, preferably backed by 
Machine Learning, would detect correlations between 
patient data and recommended treatments.  In effect, the 
spectrum of possible Covid-19 interventions (antibodies, 
steroids, convalescent plasma, and so forth) serves as a 
natural classifier, grouping patients into clusters based 
on pre-treatment profiles indicating that one or another 
Covid-19 intervention has, with some probability, a good 
chance of helping the patient.  Researchers have therefore 
been seeking empirical clues for how to classify patients 
into categories based on recommended treatment plans.}

\p{We will examine in this chapter how the 
goals of Precision Medicine have spurred scientists to propose 
more sophisticated data-integration 
and data management tools in the context of clinical trials 
and biomedical laboratory investigations.  Aside from 
opening new avenues for empirical research, the \textit{goals} 
of Precision Medicine have spurred new ways of 
thinking about the computational ecosystem which supports 
biomedical research and clinical practice. 
}

\subsection{Precision Medicine as a Catalyst for Biomedical Data Sharing}

\p{Precision medicine is based on the scientific realization that the effectiveness of a certain clinical intervention --- for instance, 
immunotherapy as a cancer treatment --- is dependent on factors that 
vary significantly between and among patients.  In the case of immunotherapy for cancer, assessing 
the likelihood of favorable outcomes requires genetic, serological, and oncological tests which need to be 
administered prior to the commencement of treatment.  
Therefore, analysis of precision-medicine outcomes requires detailed 
immunoprofiling --- building immunological profiles of each patient before 
(and perhaps during/after) the immunotherapy regimen --- alongside 
an evaluation of how well the patient responds 
to the treatment, with the best outcome being 
cancer remission or non-progression.}

\p{The contemporary emergence of Precision  
Medicine is driven, in part, by advances in diagnostic 
equipment which enable immunological profiles to be 
much more fine-grained than in previous decades.  This 
level of patient-profiling detail enables granular three-way analysis 
involving (1) patients' immunology; (2) treatment 
regimens; and (3) clinical outcomes.  This three-way approach 
has the goal of 
identifying signals in immunological profiles that indicate 
which therapeutic interventions are most likely to engender 
favorable outcomes.  As with any statistical analysis, 
predictive accuracy increases in proportion to the amount 
of data available.  As such, further refinement of 
precision medicine  
depends in part on data aggregation: pooling a number of 
observational studies where patients' immunological profiles 
are described, in detail, alongside reports on treatment plans 
and patient outcomes.}

\p{Within the scientific research community, organizations such as the 
Society for Immunotherapy of Cancer (\SITC{}) and 
the Parker Institute for Cancer Immunotherapy (\PICI{}) 
have developed 
programs and tools to share immunotherapy research 
data, which join older (but less cutting-edge) 
projects, such as Cancer Commons or the \RSNA{} 
(Radiological Society of North America) Image Share program.  
In the Covid-19 population we are similarly 
presented with multidimensional patient data.  
Such information traverses 
molecular testing to identify the virus's genetic material 
or the unique markers of the pathogen itself; antigen testing 
(albeit less accurate) to identify specific proteins found on 
the outer surface of the virus; mapping the genomes of SARS-CoV-2 
to learn how it mutates; analyzing blood samples for the presence/absence 
of antibodies in response to a prior SARS-CoV-2 infection; using 
high-dimensional flow cytometry to perform taxonomic breakdown of 
patients into distinct immunotypes related to disease severity and 
clinical parameters; profiling patients' immunological state prior 
to the start of treatment and quantifying patients' immunological 
response once treatment has begun; calculating plasma viscosity (\PV{}) 
for detection of unusually high levels of fibrinogen --- 
leading to atypical blood clots (that are refractory to 
standard anticoagulant therapy); monitoring 
cognitive, neurological, or cardiovascular  
symptoms in \q{long haulers}; and so forth.}

\p{One area where data-integration is significant for Covid-19 research 
is that of protein biomechanics.  For instance,
\cite{QiongqiongAngelaZhou} identifies  
64 different proteins which are therapeutically 
relevant to Covid-19.  Each of these proteins can be connected to molecular data, 
bioassay information and, in many cases, to clinical 
trials that test therapies where the specific protein acts as a 
target for inhibiting SARS-CoV-2.  
In combination, aggregating all of this information yields a
heterogeneous (but interconnected) 
data space, characterized 
by data derived from multiple scientific disciplines and multiple 
laboratory methods.}


%\section{Overcoming Impediments to Data-Sharing Initiatives in Precision Medicine}
%\subsection{Data-Sharing Solutions for Immunoprofiling/Immunotherapy and Covid-19}

\p{Because biomedical diagnostic and investigative 
laboratories need to use highly specialized software, 
raw laboratory data is often excluded from data-sharing initiatives.  
Instead, the information which is shared between 
hospitals or research centers tends to be a simplified 
overview --- adhering to standards curated by 
groups such as  
\OMOP{} (Observational Medical Outcomes Partnership) 
Common Data Model or \CDISC{} (Clinical Data Interchange Standards Consortium) 
--- skews more toward succinct summaries of diagnoses, treatments, and/or 
outcomes.  Limitations in sharing more granular 
diagnostic or prognostic data have been identified as obstacles diminishing 
the value of data-sharing initiatives.  As one example, in a 
paper discussing research  
into the sequencing of immunoglobulin repertoires (Ig-seq), 
\cite{SimonFriedensohn} comments: 

\vspace*{.5em}
\begin{displayquote}
A major challenge when performing Ig-seq is the production
of accurate and high-quality datasets [because] the conversion of
mRNA ... into antibody sequencing libraries relies on a number of 
reagents and amplification steps ... which potentially introduce errors
and bias [that] could alter quantitation of critical 
repertoire features. ... One way to address
this is by implementing synthetic control standards, 
for which the sequence and abundance is known prior to sequencing, thus
providing a means to assess quality and accuracy.
\end{displayquote}
The underlying 
problem in this context is how different laboratories may use 
different techniques and protocols to achieve similar diagnostic/investigative 
goals (\i{see also} \cite{LauraLopezSantibanezJacome}, 
\cite{MarksDeane}, \cite{MartijnMVanDuijn}, \cite{YanlingWu}, 
etc.).  Consequently, when data is merged from multiple hospitals, it is likely that 
the raw data derives from different labs, which can lead to situations where  
protocol variations across each site can introduce errors
and bias that contaminate the aggregate data-sharing results.  
(We should note that Ig-seq and related \q{repertoire sequencing} 
is quite relevant to coronavirus immunology because 
these techniques help quantify patients' immune response 
to infection, which is consequential both for 
explaining the differences between mild and severe 
cases and observing the effectiveness of interventions 
such as vaccinations or antibody treatments; 
\i{see} \cite{KatjaFink}, \cite{AnastasiaAMinervina}, 
\cite{ChristophSchultheiss}, \cite{NataliaTFreund}, \cite{JacobDGalson}, 
\cite{AleksandraMWalczak}, \cite{CheMaiChang}, 
 \cite{SupriyaRavichandran}, \cite{RishiRGoel}, 
etc.)}

\p{In the context of Covid-19, Shrestha \textit{et al.} 
\cite{GentleSunderShrestha} argue 
for \q{precision-guided studies} to be prioritized \q{[r]ather 
than conducting trials using the conventional trial designs and 
poor patient selection} (page 1).  To accomplish this, the authors 
recommend \q{large multicenter trials} which incorporate 
\q{predictive enrichment strategies ... to 
identify and thus target specific phenotypes [patient-profile 
characteristics], potentially raising the possibility of positive 
trial outcomes.}  The underlying problem identified by Shrestha \textit{et al.} 
is acquiring sufficiently large trial cohorts 
in contexts where trials are to be targeted at 
fine-grained patient populations with specific pre-treatment 
immunological profiles.  This problem can be 
ameliorated by merging prospective patients from 
multiple institutions, such that concurrent trials spanning 
multiple health-care settings can be launched as part 
of a comprehensive approach to comparing treatment options.  
Such large multicenter trials  
can produce \q{large cohorts of patients in a shorter 
time period} while also steering patients toward 
more favorable treatment courses.}

\p{In short, Shrestha \textit{et al.} 
explicitly challenge the conventional wisdom which 
assigns \q{gold standard} status only to \textit{randomized} 
trials, arguing that the benefits of larger trial sizes 
and quicker trial initiations outweigh whatever 
statistical value is compromised by earmarking 
patients into trials based on educated guesses as 
to favorable outcomes (which is warranted by 
patient-care ethics in any case).  The authors also  
argue for a \q{robust data 
infrastructure} which would combine the efforts of clinicians, researchers, 
and data scientists.  In effect, aggressive data-curation would 
substitute for double-blind trial design as a means 
of ensuring the scientific value of trial outcomes.}

\subsection{Software Alignment for Covid Phylogeny Studies}% across Multiple Populations}
\p{The study of SARS-CoV-2 mutations is another area  
where software alignment can prove to be important.   
Analyses in 2020 suggested that variations in the 
viral strain causing Covid-19 symptoms may be partly 
responsible for divergent immunological responses to 
the virus across the patient 
population \cite{OsmanShabir}.  If one 
patient 
responds 
either less favorably or more favorably than the average patient-response 
to a 
given treatment, clinicians need to assess whether this 
difference can be explained solely by the patient's 
prior immunological profile or whether the patient has 
been exposed to a genetically divergent viral strain.
In 2021, of course, predictions about SARS-CoV-2 mutations 
came to pass, with at least four variants appearing to be 
sufficiently dangerous (by virtue of their infectiousness 
and/or lessened effectiveness of treatments or vaccines) 
to undo progress which has been made against Covid-19 
in many parts of the world.}

\p{Comprehensive models of Covid-19 variants require a 
combination of genetic, proteomic, anatomic, and 
epidemiological information, because mutations can 
only take hold if they modify the virus's morphology 
and/or infectiousness in ways that are conducive to 
that strain replicating.  The \b{delta} strain, 
for example, which (as of mid-2021) has been the 
most wide-spread mutation \cite{LizSzabo}, 
carries a genetic variation (designated 
as \q{\b{D614G}}) 
which results in a denser 
array of spike proteins, thereby reinforcing 
the biomechanic pathway which the virus uses 
to infect the host's respiratory system 
\cite{BetteKorber}, \cite{AnwarMohammad}, 
\cite{UtsavPandey}, \cite{LizhouZhang}, 
\cite{DonaldJBenton}, \cite{ErikVolz}, 
\cite{ShiZhao}, etc.}

\p{Modeling SARS-CoV-2 evolution across the globe is a 
massive project.  There have been over 200 million Covid-19 cases 
worldwide (as of mid-2021), in virtually every nation on earth, so a 
complete phylogenetic picture of SARS-CoV-2 in 
humans would need to pool data from many different 
healthcare systems.  Yet, even technically 
detailed analysis of the phylogeny of SARS-CoV-2, such as that 
conducted by Dearlove \textit{et al.} (as 
reported at the end of 2020) 
only considered 27,977 patients (about 0.1\% of 
global cases), with almost half from the 
United Kingdom \cite{BethanyDearlove}.  
Hence, achieving something resembling a holistic picture of 
SARS-CoV-2 mutations and how they might affect clinical 
treatments, would require many parallel studies analogous 
to that of Dearlove \textit{et al.}.}

\p{This then raises questions of study alignment: calculating 
viral phylogeny requires making technical decisions about how genetic 
sequences should be acquired and analyzed, decisions which 
may vary among research teamss.  
For example, Dearlove \textit{et al.} describe several computational 
steps which they had performed 
both to normalize each SARS-CoV-2 genome sequence in their 
data set for cross-comparison and to run predictive 
simulations (used to estimate whether divergence between 
sequence-pairs are the result of localized, random mutations or, 
conversely, an indication that SARS-CoV-2 is evolving into 
further distinct strains).    
Clustering SARS-CoV-2 genomes into variants --- that 
is, identifying which mutations are random and which 
appear to be propagating to subsequent viral generations 
--- involves making computational and biological 
assumptions, such as how to statistically 
marshal genomic data so as to quantify the prevalence of a 
mutation, and how to estimate whether a particular mutation confers 
an adaptive benefit to the viral agent (e.g., an 
ability to elude antibodies targeting structural proteins).}

\p{Given that modeling viral phylogeny requires 
certain computational assumptions and biological 
guesswork, data from multiple studies can only be 
reliably integrated if there is some degree of 
alignment across their methodology.  As such, 
research teams should document their protocols 
in a manner that permits assessment as to whether 
protocol differences might compromise the 
resulting data.  One way to achieve this is to 
model the protocol itself as a datatype in a 
general-purpose programming 
language (such as \Cpp{}, for sake of argument).  For each study, such as 
Dearlove \textit{et al.} cited above, there would then 
be a \Cpp{} object encapsulating details of 
the researchers' protocols and computational workflows.  
Protocol-alignment would in this context 
be one part of a common framework to quantify the 
epidemiological significance of SARS-CoV-2 mutations.}

\p{In short, a holistic 
global picture of SARS-CoV-2 must represent 
SARS-CoV-2 mutations which have been deemed 
phylogenetically and/or clinically significant (i.e., having 
potential either to influence the overall evolution of 
Covid-19 and/or to have some bearing on clinical treatments), 
and must \textit{also} represent divergent SARS-CoV-2 strains 
carrying those mutations.   
These data-points 
are then be the basis of further details such as: 
when did a given strain and/or mutation first appear?  Is 
the strain/mutation geographically localized?  What is the 
proportion of different strains/mutations in a geographic 
area?  Is there evidence that different strains/mutations 
affect a patient's immunological response to Covid-19 and/or 
the effectiveness of vaccines, antibody regimens, steroids, 
or other clinical interventions?  How can genetic mutations within 
the SARS-CoV-2 virus be correlated with structures in the spike 
proteins encoded by the viral genes?  This last question points 
to the importance of integrating genomic data with \ThreeD{} molecular 
models (\textit{see} \cite{WingerCaspari}, \cite{SandipanChakraborty}, 
\cite{IvanMercurio}, \cite{AliFAlsulami}, \cite{SumanPokhrel}, 
\cite{VictorPadillaSanchez}, \cite{FedaaAli}, for example).  
Whereas data structures modeling the viral genome 
are composed of nucleotides --- and, at a higher scale, Open Read 
Frames (\ORF{}s) --- data structures describing the biophysics of 
glycoproteins involve protein architecture and chemical 
bonds \cite{LiangweiDuan}, \cite{AnshumaliMittal}, 
\cite{XiuyuanOu}, \cite{GiwanSeo}, \cite{HanhTNguyen}, 
etc.  Analyzing how 
\makebox{SARS-CoV-2} genes 
affect the production of glycoproteins, therefore, requires 
annotating and cross-referencing nucleotide/\ORF{} data structures 
with \ThreeD{} molecular models encoded in formats 
such as \MOL{} or Protein Data Bank (\PDB{}) 
\cite{YasunoriWatanabe}, \cite{YongfeiCai}, \cite{AlexandraCWalls}, 
\cite{GennadyMVerkhivker}, \cite{CheolminKim}, etc.}

\p{Object-Oriented models for genomic phylogeny analysis 
have been presented in \cite{JoshuaBSinger} 
(Java) and \cite{GuanghongZuo} (\Cpp{}), although 
these do not provide object models to extend 
from genetic information toward clinical, proteomic, 
or imaging data.  We are not aware of Object-Oriented models 
proposed as representational devices 
for Covid Phylogeny in this more holistic and 
interdisciplinary, but this form of 
software design would be consistent with code 
developed in contexts such as oncology, for 
example the computational simulation of 
tumor growth, which we will discuss in Chapter 4.  
Covid-Phylogeny Object Models 
could serve as a nexus for merging temporal and 
geographical data concerning the epidemiology of \makebox{SARS-CoV-2} 
mutations with genomic data demonstrating  
which mutations are significant, as well as clinical 
data tracking correlations between mutations and treatment outcomes.  
Supporting multi-trial data integration would, therefore, 
also introduce new requirements for clinical trial software, 
which we discuss further next chapter.}

\subsection{Personalized Medicine and Immuno-Profiling}
\p{While some of the data related to immunological profiles 
may be sociodemographic or part of a patient's medical history (fitting 
nicely within conventional Clinical Research 
Network models), contemporary 
immunoprofiling is powered by 
highly specialized diagnostic equipment and methods --- 
which require special-purpose file formats and software.  
For instance, one dimension of immunological profiling is \q{immune repertoire};
the more robust a person's repertoire, the wider variety of antibodies they can produce to fight off pathogens.  Immune repertoire is often measured by studying genetic diversity in B-cells; in recent years, 
this has been done using \q{Next Generation Sequencing} (\NGS{}), which produces files in formats 
such as \FASTQ{}.  Another dimension of immunological profiling is quantifying the proportions of different sorts of blood cells in a patient's blood sample, which is typically done via Flow Cytometry or Mass Cytometry, yielding \FCS{} (Flow Cytometry Standard) files.  The immunological evaluations which are the goal of these methods are sometimes called Cell-Type Classification or \q{Automated Cell-type Discovery and Classification} 
(\ACDC{}).}

\p{As outlined in the below table (Figure~\ref{fig:tabl}), immunological profiles draw on a diversity of data formats and lab/data-acquisition modalities (this table is not intended to be a complete list of criteria or file formats, but rather to indicate the range of diagnostic technologies and data formats which are relevant to immunoprofiling).  This table hopefully indicates how immuno-oncology data sharing 
is inherently more complex when compared to data-sharing 
initiatives which focus primarily on 
clinical outcomes --- such as  
\OMOP{}, \CDISC{}, or the Patient-Centered Outcomes Research Network 
(\PCORnet{}).}


\begin{figure*}
\caption{Table Outlining Several Data Formats and the Contexts where they are Acquired}
{\large
\vspace{1.5em}
\begin{tabular}{ l | r | c }
\underline{Immunological Profile Dimension} \hspace{.5em} & \underline{Lab/Clinical Method} \hspace{.5em}   & \underline{File Format} \\
Sociodemographic                & \hspace{.3em} Scan Patient Records \hspace{.3em} & CSV, SQL \\
Medical History                 & \hspace{.3em} Scan Patient Records \hspace{.3em} & CDISC, OMOP, PCOR \\
Immune Repertoire               & \hspace{.3em} NGS \hspace{.3em}                 & FASTA, FASTQ \\
Cell Type Classification        & \hspace{.3em} Flow/Mass Cytometry \hspace{.3em} & FCS \\
Immune/Tumor Microenvironment         & \hspace{.3em} Confocal Microscopy \hspace{.3em} & \hspace{.4em} DICOM, OME-TIFF, CoCo

\end{tabular} 
\vspace{1.5em}
}
\label{fig:tabl}
\end{figure*}




\p{Formats such as the  \OMOP{} Common Data Model (\CDM{}), the \PCORnet{} \CDM{}, 
or the \CDISC{} specifications promote data sharing primarily through 
\SQL{}-style tables, where 
data analysis and extraction can be achieved via conventional 
\SQL{} queries.  The situation is very different, however, when 
the data that must be exchanged derives from specialized hardware 
and software, which demands special-purpose file formats, parsers, 
and query engines.  This problem-space is accentuated 
when preparing multi-site sharing that can span dozens of 
hospitals, research centers, and/or laboratories.  
Problems of cross-institutional data integration 
will be analyzed further next chapter.}

\p{Prior to that discussion, the following section 
will examine in detail the file formats and data structures 
which were briefly mentioned thus far in this chapter.  
Our goal in this discussion is to document the 
kinds of data which are endemic to different branches 
of biomedical research.  A separate analysis --- one 
which to some extent depends on first describing the data 
formats involved --- concerns how to fuse multiple 
data formats into a common overarching format, such as a 
\q{Common Data Model of Everything.}  One recurring 
theme we will encounter in the subsequent discussion is the 
goal of merging certain data profiles into others 
(e.g., \FCS{} into \DICOM{}), or else adopting 
common interchange formats (such as \XML{}) in lieu of 
domain-specific binary formats which demand 
special-purpose parsers.  We will examine both the strengths 
and weaknesses of proposals for adopting common formats 
rather than idiosyncratic \q{legacy} formats that 
are tied to particular laboratory methodologies for 
historical reasons.  However, our main purpose in the 
current discussion is to establish basic facts about 
data formats in current use.  Later chapters will 
analyze problems connected to the integration of these 
formats into common data models.} 

\p{Note also that the following inventory of biomedical 
data formats is by no means exclusive.  In particular, we 
have largely neglected antigen tests, biochemical assays, 
and many other lab techniques which rely on chemical 
reactions to obtain lab/diagnostic findings.  Our discussion 
here is oriented more toward methodologies which require 
relatively complex intermediate computational processing 
to arrive at clinically useful findings.}



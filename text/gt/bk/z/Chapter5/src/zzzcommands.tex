
\usepackage{amssymb}
\usepackage{amsmath}


\definecolor{DarkRed}{rgb}{.2,.0,.1}

\colorlet{orrr}{orange!40!red}
\colorlet{orrbl}{orrr!85!blue}
\colorlet{orrb}{orrbl!80!DarkRed}



\newcommand{\CodeMinted}[1]{{\color{codegr}{{%
\fontfamily{lmss}\fontseries{b}\selectfont{#1}}}}}

\newcommand{\CodeMintedo}[1]{{\color{orange!40!black}{{%
\fontfamily{lmss}\fontseries{b}\selectfont{#1}}}}}

\newcommand{\CodeMintedr}[1]{{\color{red!40!black}{{%
\fontfamily{lmss}\fontseries{b}\selectfont{#1}}}}}

\newcommand{\CodeMintedch}[1]{{\color{chc}{{%
\fontfamily{lmss}\fontseries{b}\selectfont{#1}}}}}

\newcommand{\codeText}[1]{\CodeMinted{#1}}
\newcommand{\codeTexto}[1]{\CodeMintedo{#1}}
\newcommand{\codeTextr}[1]{\CodeMintedr{#1}}

\newcommand{\codeTextch}[1]{\CodeMintedch{#1}}



\newcommand{\lclc}[1]{{\color{orrb}{#1}}}

\newcommand{\lcl}[2]{{\resizebox{!}{#1}{\color{orrb}{#2}}}}
\newcommand{\ty}{{\lcl{8pt}{\ensuremath{\mathfrak{t}}}}}

\newcommand{\caltypeT}{\ensuremath{\ty}}
\newcommand{\calS}{\ensuremath{\mathcal{S}}}

\newcommand{\typeTp}{\lclc{\ensuremath{\ty'}}}
\newcommand{\typeTpp}{\lclc{\ensuremath{\ty''}}}

\newcommand{\vVarPrime}{\codeTextr{\ensuremath{v'}}}
\newcommand{\vVarDblPrime}{\codeTextr{\ensuremath{v''}}}
\newcommand{\vVar}{\codeTextr{\ensuremath{v}}}



\newcommand{\TyS}{\codeTextr{$\mathbb{T}$}}

\newcommand{\TXLTyS}{$\codeTextr{\mathfrak{L}_\codeTextr{\mathbb{T}}}$}

\newcommand{\TXLTySChi}{$\codeTextr{\mathfrak{L}_\codeTextr{\mathbb{T}}\chiussr}$}

\newcommand{\tColl}{\codeTextr{\ensuremath{\mathcal{T}}}}

\newcommand{\nth}{\codeTextr{\ensuremath{n}th}}
\newcommand{\nNum}{\codeTextr{\ensuremath{n}}}

\newcommand{\ageFF}{\codeTexto{${\lceil}45{\rceil}$}}
\newcommand{\Nath}{\codeTexto{${\lceil}$Nathaniel${\rceil}$}}

\newcommand{\nodeNOne}{$N_1$}
\newcommand{\nodeNTwo}{$N_2$}

\newcommand{\NThree}{\AcronymText{N3}}

\newcommand{\ceila}[1]{${\lceil}$#1${\rceil}$}
\newcommand{\angla}[1]{${\langle}$#1${\rangle}$}

\newcommand{\NathFF}{\codeTexto{\angla{\ceila{Nathaniel}, \ceila{46}}}}
\newcommand{\NathFFBD}{\codeTexto{\angla{\ceila{Nathaniel}, \ceila{46}, %
\ceila{Brooklyn}, \ceila{Democrat}}}}

\newcommand{\BrookDem}{\codeTexto{\angla{\ceila{Brooklyn}, \ceila{Democrat}}}}

\newcommand{\nameAge}{\codeTexto{${\langle}$\codeText{name}, \codeText{age}${\rangle}$}}

\newcommand{\suigeneris}{\i{sui generis}}
 
\newcommand{\struct}{\codeText{struct}}
\newcommand{\CStructs}{\codeText{struct}s}
\newcommand{\CStructsArrays}{\codeText{struct}s/arrays}

\newcommand{\CStrucst}{C\codeText{struct}}

\newcommand{\throw}{\codeText{throw}}
\newcommand{\exception}{\codeTextch{exception}}

\newcommand{\float}{\codeText{float}}

\newcommand{\chiuss}{\raisebox{-1pt}{$^\chiu$}}
\newcommand{\chiussr}{\raisebox{-1pt}{$^\chiur$}}

\newcommand{\chiur}{\codeTextr{\ensuremath{\chi}}}
\newcommand{\chiu}{\codeText{\ensuremath{\chi}}}

\newcommand{\TySChi}{\TyS\chiussr}


\newcommand{\lCh}{\chsnt{lam}}
\newcommand{\rCh}{\chsnt{ret}}
\newcommand{\xCh}{\chsnt{exc}}

\newcommand{\lrCh}[1]{\chsnt{lam, ret, #1}}

\newcommand{\lrChwow}[1]{\chsnt{lam!, ret!, #1}}

\newcommand{\lsrxCh}{\chsnt{lam, sig, ret, exc}}

\newcommand{\lrChblank}{\chsnt{lam, ret}}

\newcommand{\rChSize}{\chsntsz{ret}}
\newcommand{\chChSize}{\chsntsz{ch}}

\newcommand{\rNoMixx}{\rCh{\chcolor{$\nshortparallel$}}\xCh}
\newcommand{\rOneOrx}{\rCh{\chcolor{$\asymp$\xCh}}}

\newcommand{\rChSizeleOne}{\codeText{\rChSize{} $\leq$ 1}}
\newcommand{\rChSizeLe}{\codeText{rChSizele}}
\newcommand{\Ch}{\chsnt{ch}}

\newcommand{\lrxSimple}{\lrChwow{exc?}}

\newcommand{\lrxDetailed}{\makebox,\chnt{exc?\%}}}

\newcommand{\excl}{\chcolor{\ensuremath{\asymp}}}

\newcommand{\lrxTotal}{\makebox{%
\lrCh{exc} {\colonblg} %
\chnt{lam!*},\chnt{ret!\%},\chnt{exc?\%}%
{ }{\colonblg}{\colonblg} \chnt{ret}{\excl}\chnt{exc}}}

\newcommand{\lr}{\makebox}}

\newcommand{\lsrx}{\makebox{\makebox{\lsrxCh}%
{ }{\colonblg} \chnt{lam!*},\chnt{sig?\%},\chnt{ret!\%},\chnt{exc?\%}}}

\newcommand{\sCh}{\codeTextch{sigma}}


\newcommand{\lrx}{\makebox{%
\lrCh{exc}}}

\let\OldLambda\lambda


\renewcommand{\lambda}{\codeTextch{lambda}}
\renewcommand{\return}{\codeTextch{return}}


\newcommand{\returnch}{\codeTextch{return}}

\newcommand{\this}{\codeText{this}} 

\renewcommand{\DH}{\AcronymText{DH}}
\newcommand{\CH}{\AcronymText{CH}}

\newcommand{\lambdaPLUSreturn}{{\lambda}{\codeTextr{+}}{\return}}

\newcommand{\capturePLUSexception}{{\capturech}{\codeTextr{+}}{\exceptionch}}

\newcommand{\archiveDate}[1]{{\footnotesize (archived #1)}}

\newcommand{\lxr}{\codeText{lxr}}

\newcommand{\fnote}[1]{\codeText{#1}}

\newcommand{\codeinclude}{\codeText{\#include}}
\newcommand{\codeconstruct}{\codeText{construct}}

\newcommand{\USH}{\acronymText{USH}}

\newcommand{\fxy}{\codeText{\makebox{f(x, y)}}}
\newcommand{\fSym}{\codeText{f}}

\newcommand{\fofx}{\codeText{\makebox{f(x)}}}
\newcommand{\fofy}{\codeText{\makebox{f(y)}}}

\newcommand{\fFuns}{\codeText{f}}
\newcommand{\hFun}{\codeText{h}}
\newcommand{\hfx}{\codeText{h(f(x))}}
\newcommand{\fxdoth}{\codeText{f(x).h()}}

\newcommand{\typesH}{\codeText{types.h}}

\newcommand{\QDataStream}{\codeText{QDataStream}}


\newcommand{\negOne}{\resizebox{!}{6.5pt}{\codeText{-1}}}

\newcommand{\NaN}{\codeTextrr{6pt}{NaN}}


\newcommand{\qNaN}{\q{\NaN}}

\newcommand{\RGB}{\AcronymText{RGB}}


\newcommand{\kVar}{\codeTextr{\ensuremath{\mathcal{K}}}}


\newcommand{\tyR}{\resizebox{!}{6.5pt}{\codeTextr{\ensuremath{\mathscr{r}}}}}


\newcommand{\rlstsize}{\codeText{\ranged{0, list.size()-1}}}
\newcommand{\intrlstsize}{\codeText{int\ranged{0, list.size()-1}}}

\newcommand{\fIntG}{\codeText{int f(int g)}}
\newcommand{\fIntI}{\codeText{int f(int i)}}

\newcommand{\retFive}{\codeText{return 5}}

\newcommand{\QString}{\codeText{QString}}
\newcommand{\QObject}{\codeText{QObject}}
\newcommand{\ptrv}{\codeText{void*}}

\newcommand{\aVal}{\codeText{a}}

\newcommand{\switch}{\codeText{switch}}

\newcommand{\ztooh}{\codeText{0-100}}

\newcommand{\GPS}{\AcronymText{GPS}}
\newcommand{\HCS}{\AcronymText{HCS}}

\newcommand{\qHCS}{\q{\HCS}}


\newcommand{\Zero}{\codeTextr{\ensuremath{0}}}
\newcommand{\vVarPrime}{\codeTextr{\ensuremath{v'}}}
\newcommand{\vVarDblPrime}{\codeTextr{\ensuremath{v''}}}
\newcommand{\vVar}{\codeTextr{\ensuremath{v}}}

\newcommand{\eVar}{\codeTextr{\ensuremath{e}}}


\newcommand{\operatorB}{\codeText{operator[]}}

\newcommand{\lrgAcronymText}[1]{\textscc{#1}}

\newcommand{\NCFour}{\AcronymText{NC4}}

\newcommand{\qNCFour}{\q{\NCFour}}

\newcommand{\lrgNCFour}{\lrgAcronymText{NC4}}


\newcommand{\ZeroToOneHundred}{\codeText{\ranged{0,100}}}

\newcommand{\OneHundred}{\codeTextr{\ensuremath{100}}}


\newcommand{\codeTextrr}[2]{\resizebox{!}{#1}{\codeTextr{#2}}}

\newcommand{\cCar}{{\lcl{7pt}{\ensuremath{\mathfrak{c}}}}}
\newcommand{\cCarOne}{{\lcl{7pt}{\ensuremath{\mathfrak{c_1}}}}}
\newcommand{\cCarTwo}{{\lcl{7pt}{\ensuremath{\mathfrak{c_2}}}}}



\newcommand{\nNumGtZero}{\codeTextr{\ensuremath{n > 0}}}

\newcommand{\CLang}{\AcronymText{C}}



\newcommand{\mbegin}{\codeText{begin()}}
\newcommand{\mend}{\codeText{end()}}

\newcommand{\tyOne}{\codeText{${\ty}_1$}}
\newcommand{\tyTwo}{\codeText{${\ty}_2$}}

\newcommand{\tyOneTotyTwo}{\tyOne \codeText{$\rightarrow$} \tyTwo}

\newcommand{\chK}{\codeText{$\mathcal{K}$}}






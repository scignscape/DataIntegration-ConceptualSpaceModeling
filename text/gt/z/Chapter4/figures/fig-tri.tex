

\begin{figure}[t!]


\caption{Triangular Relationship Between Simulations, 
Imaging, and Clinical/Pathology/Diagnostic Data}	


\begin{center}

\hspace*{-3em}\begin{tikzpicture}[tips=proper]

\pgfdeclarelayer{background}
\pgfsetlayers{background,main}

\tikzstyle{every node}=[font=\small]

\colorlet{blbl}{blue!70!black}
\colorlet{blblor}{blbl!70!orange}

\colorlet{gror}{green!40!orange}


%\node (dummy) at (0,0) {};

\colorlet{blor}{black!80!orange}
\colorlet{blbrw}{black!80!brown}


\node (sim) at (0,-3) [draw=urlclr, draw opacity=.1, 
line width=2mm, 
align=right,
ellipse,fill=blbrw!20,
inner sep = 5, text width=1.6cm] 
{Simulations (\q{in silico})};
%{\scalebox{1}[2]{Simulations} \scalebox{1}[2]{(\q{in silico})}};


\node (simgui) [above=5mm of 
sim.north west, xshift=-6mm, align=right,text width=1.6cm] 
{gui};

\node (simser) [right=-2mm of 
simgui.east, yshift=2mm, align=right,text width=1.6cm] 
{seriali\-zation};

\node (simdb) [below=21mm of 
simgui.south, xshift=9mm, align=right,text width=1.6cm] 
{database};

\node (simproc) [right=5mm of 
simdb.east, yshift=-2mm, align=right,text width=1.6cm] 
{procedural model};


\node (im)  [draw=urlclr, draw opacity=.1, 
line width=2mm, 
right=26mm of sim, draw, fill=blbrw!20,
ellipse,align=left,inner sep = 5, text width=1.5cm] 
{Imaging/\\Radiomics};
%{\scalebox{1}[2]{Imaging/} \scalebox{1}[2]{Radiomics}};



\node (imgui) [above=4pt of 
im.north, xshift=-8mm, align=right,text width=1.6cm] 
{gui};

\node (imser) [right=5mm of 
imgui.east, yshift=1mm, align=left,text width=1.6cm] 
{seriali\-zation};

\node (imdb) [below=16mm of 
imgui.south, xshift=-8mm, align=right,text width=1.6cm] 
{database};

\node (improc) [right=1mm of 
imdb.east, yshift=-5mm, align=right,text width=1.6cm] 
{procedural model};



\node (cl)  [draw=urlclr, draw opacity=.1, 
line width=2mm, 
above=16mm of sim, xshift=26mm, fill=blbrw!20,
ellipse, draw, align=right, inner sep = 5] 
{Clinical/Lab};

\node (clgui) [above left=3mm of 
cl.north west, align=right,text width=1.6cm] 
{gui};

\node (clser) [right=23mm of 
clgui.east, align=right,text width=1.6cm] 
{serialization};

\node (cldb) [below=8mm of 
clgui.south, xshift=-1mm, align=right,text width=1.6cm] 
{database};

\node (clproc) [right=32mm of 
cldb.east, yshift=1mm, align=right,text width=1.6cm] 
{procedural model};


\begin{pgfonlayer}{background}
%\begin{scope}[on background layer] 


\draw (clgui) -- (clproc);
\draw (cldb) -- (clser);

\draw ([xshift=5mm]imgui.south) -- (improc);
\draw (imdb) -- (imser);

\draw ([xshift=5mm]simgui.south) -- (simproc);
\draw (simdb) -- ([xshift=1mm]simser.south);


\draw[blbl,draw opacity=.4, line width=3mm] (sim) -- (im) -- (cl) -- (sim);

\begin{scope}{even odd rule}
      \clip
        (current bounding box.south west)
        rectangle (current bounding box.north east)
        let
          \p{sim} := ($(sim.north) - (sim.center)$),
          \p{im} := ($(im.north) - (im.center)$),
          \p{cl} := ($(cl.north) - (cl.center)$)
        in
          (sim) circle[radius=\y{sim}]
          (im) circle[radius=\y{im}]
          (cl) ellipse[radius=\y{cl},xscale=4]
      ;
      \fill [top color=gror!40, bottom color=blbl, fill opacity=1] (sim.center) -- (im.center) -- (cl.center) -- cycle;

      \fill [left color=red!80!orange, 
right color=green,shading angle=30, 
fill opacity=.3] (sim.center) -- (im.center) -- (cl.center) -- cycle;

%      \fill [left color=blbl, 
%right color=transparent,%shading angle=30, 
%fill opacity=.3] (sim.center) -- (im.center) -- (cl.center) -- cycle;


      \fill [left color=blue!78!yellow, 
right color=cyan!70!yellow,shading angle=6, 
fill opacity=.1] (sim.center) -- ([xshift=-3cm]im.center) -- 
([xshift=-1.7cm,yshift=-2cm]cl.center) -- cycle;

      \fill [right color=black!78!yellow, 
left color=cyan!70!yellow,shading angle=6, 
fill opacity=.1] (im.center) -- ([xshift=3cm]sim.center) -- 
([xshift=1.4cm,yshift=-2cm]cl.center) -- cycle;

    \end{scope}

%\fill [] (sim.center) -- (im.center) -- (cl.center) -- cycle;

%\end{scope} 
\end{pgfonlayer}

\end{tikzpicture}

\end{center}


\label{fig:triangle}
\end{figure}






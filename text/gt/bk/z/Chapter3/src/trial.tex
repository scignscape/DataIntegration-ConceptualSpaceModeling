
\section{Precision Medicine in Trial Design}
\p{\noindent{}Converting basic research to 
clinical practice directly benefiting patients
--- sometimes called \i{translational informatics},
a \q{research cycle, which involves the translation of knowledge
and evidence [to] provision of evidence-based care
in the clinical or public health settings}   
\cite[page 2]{PhilipROPayne} --- is sometimes 
represented as a two-stage process which 
first involves translating research to 
clinical trials, and then formulating 
point-of-care practices on the basis of trial 
results (\i{ibid.}).  Data-sharing 
initiatives need to pay particular attention to 
the logistics of translational informatics in 
contexts where granular patient-specific information is important, 
such as immunoprofiling.  Questions which should be addressed include 
(1) where data is to be hosted; (2) how participating 
institutions should submit data to a central repository; 
(3) how participating institutions and/or outside investigators 
should access previously-deposited data; (4) how to 
ensure anonymization of patient-specific records; (5) how to ensure 
that different labs used by different hospitals are utilizing 
compatible protocols, so that results from multiple labs/hospitals 
can be seamlessly merged in a shared data commons; 
(6) how to ensure proper alignment between software employed 
at different institutions; and (7) how to incorporate data 
curated within the context of a multi-institutional data-sharing initiative 
into scientific papers documenting 
research findings.  Each of these areas of concern 
are technically demanding because of the complex and 
heterogeneous nature of immunological profiles.}

\p{As a case-study in clinical-trial software 
engineering, consider again the proposals 
in Shrestha \textit{et al.} \cite{GentleSunderShrestha} 
which we reviewed last chapter.  As these authors 
recommend, Covid-19 trials should be designed to focus on 
specific patient groups which are more likely to benefit from the interventions 
that form the basis of the relevant clinical trials.  Moreover, 
toward the goal of applying precision medicine to Covid-19 clinical practice, 
it should be possible to construct a quantifying domain of patient-profile signals
(antecedent to trial commencement) 
to quantify the statistical probability that a given treatment will have a 
favorable patient outcome in relation to all the prior data in a patient's 
profile.  Since researchers assume that 
certain factors in a patient's profile 
will be statistically correlated with favorable outcomes in conjunction 
with specific treatment plans, part of the trial's purpose 
is to determine which parts of the patient profile 
are, in fact, statistically relevant.}

\p{In practical terms, then, setting up Covid-19 trials would involve 
defining patient-selection criteria and implementing systems to 
screen for patients who may be good candidates for different trials.  
This would require two steps: (1) constructing a format where trial criteria 
can be rigorously notated; and (2) implementing software at 
participating sites to search for trial candidates.  
This software would need to access, represent, and analyze 
patient-specific trial-eligibility factors that covering a 
broad spectrum of data types (sociodemographics,  
medical history, lab/image results, etc.).  
Proposing an automated recruitment 
engine for cancer trials, \cite{SaytaSSahoo} argued (in 2014) that

\begin{displayquote}
Many ... tools have been developed for accessing
the institutional data warehouse to screen patients for clinical 
trials or for creating an alert system for physicians ... 
However, ... existing systems have limited access to the complete patient information, such as the latest laboratory test results, and are not integrated with the clinical systems used in routine patient care.  Further, existing tools have limited support for structured entry of trial information, interactive user interfaces (UIs) that allow clinicians to review the matching results and re-execute the matching process with updates to patient records.  [Although] there has been extensive research in creating formal, computable representation of eligibility trial specifications that can be used together with electronic representation of patient data in EHR systems [an] important challenge for computational representation of trial information is
the lack of suitable interfaces for entering eligibility criteria. (page 2)
\end{displayquote}  
\noindent{}Ten years earlier, \cite{JohnSMcIlwain} advocated 
for trial models \q{focused on interoperability, using modern object-oriented techniques} (page 3):  
\begin{displayquote}
Object-oriented software ... is
more reliable because fewer software code changes are required as one's needs evolve; and, therefore, more 
changes can be made without effectively changing the core structure of the 
application ... The difference between object-oriented software and traditional software is substantial. (page 6)
\end{displayquote} 
\noindent{}However, he concludes, \q{Very few CTMS products are object-oriented.}  
Both \cite{SaytaSSahoo} and \cite{JohnSMcIlwain}, we should 
note, are not neutral observers, but writing to highlight 
features of software their own teams designed.  Whatever 
the merits of their systems, these design principles have not 
apparently been incorporated into mainstream \CTMS{} 
in intervening years.}

\p{Given the sheer scale of the SARS-CoV-2 pandemic, 
there are likely to be many candidates for almost any Covid-19 
trial.  However, methods for recruiting patients would 
need to be aligned across multiple institutions, at 
least in the case of multi-site trials.  For example, in the 
context of antigen tests (measuring virus antibody levels) 
the US Centers for Disease Control recommends 
or has authorized a large list of 
assays performed by many different companies, using 
many different biochemical 
methods.\footnote{\textit{See}, for example, 
\bhref{https://www.fda.gov/medical-devices/coronavirus-disease-2019-covid-19-emergency-use-authorizations-medical-devices/vitro-diagnostics-euas\#individual-antigen}.}
Because the data format resulting from immunoassays depends 
on the specific biochemical mechanism which (within each 
assay) yields quantitative data, a broad spectrum of 
antigen tests requires a diverse array of data formats 
which need to be integrated.  As such, whenever 
Covid-19 immunoassays are considered 
as factors in immunological profiles for mapping 
patients to appropriate Covid-19 clinical trials, querying for good 
trial candidates means querying across a 
wide spectrum of structurally different data types that 
correspond to this broad array of antigen tests --- specifically 
to the mechanisms through which laboratory instruments generate 
quantifiable data and to the computational procedures which process 
such data.  Here is an example of why specialized trial software 
can be warranted: the heterogeneity of trial data, 
may call for 
integrative procedures implemented directly 
in programming languages such as \Cpp{} 
(rather than query languages such as \SQL{}).  
The complexity of trial criteria was a motive 
for the software-based \CTMS{} systems we cited 
earlier; Covid-19 serves as a trenchant case-study 
supporting recommendations along those lines 
(\i{see also}, say, \cite{LiMinLiu}, 
\cite{CrengutaBogdan}, \cite{MichaelSouillard}, 
\cite{CanhamOhmann}, 
\cite[especially pages 20-29]{KrisVerlaenen}).}


\subsection{Customizing Clinical Trial Management Software}
\p{Once trials embrace heterogeneous data models
(which require special-purpose software for accessing 
some of the trial data),  Clinical Trial Management Systems 
(\CTMS{}) requirements become 
more complex.  In these situations, \CTMS{}s may need 
to model and in some cases replicate complex computational 
workflows, such as those employed by Dearlove \textit{et al.} 
for calculating SARS-CoV-2 genomic sequences from patients' blood samples.  
The \CTMS{} software may also need to interoperate 
with domain-specific applications, as in bioimaging and 
image analyses, signal processing (e.g., for \EKG{} analysis), 
Flow Cytometry, biochemical assays, genomic analysis, 
epidemiological modeling and so forth.  If possible, such 
applications should be configured or extended to work 
with the clinical trial software.  For instance, if a \DICOM{} 
(Digital Imaging and Communications in Medicine) client 
is used to study an image derived from a specific trial 
--- e.g., a radiological scan of a Covid-19 patient's lungs 
--- the \DICOM{} software could be provided with a plugin 
that would show trial information in a separate window, 
which could then be juxtaposed with the main-image 
view.    
In this context, software alignment means that 
all institutions participating in a trial could use 
the \textit{same} plugins, so that the trial's central 
\CTMS{} system could interoperate with special-purpose 
software in a consistent manner.  This would also 
aid in establishing, as part of the trial design, 
protocols for depositing special-purpose data assets  
(such as \FCS{} or \DICOM{} files) alongside 
clinical data and \eCRF{}s.}

\p{A further benefit of \CTMS{} customization is that 
custom software adds flexibility for trial design.  
By definition, trials allow researchers to test 
biomedical hypothesis in a controlled manner.  
Trials are, therefore, defined around the premise that 
observational information resulting from the trial 
is empirically significant, revealing something new 
about what the trial was designed to investigate.  
For instance, a Covid-19 trial might assess how 
well patients with varying prior immunological 
profiles respond to monoclonal antibody (\mAb{}) treatments.  
The relevant observations in this case derive from 
the subsequent course of the disease for each 
patient, as well as potential adverse reactions, but there are  
inherent complicating factors: were patients receiving other 
treatments as well?  For patients who recover, how do 
we know that the antibodies expedited that recovery? 
How quickly was the recovery?  And, did patients continue 
to suffer from Covid-related symptoms even when they 
were no longer infectious?  Situational details specific to the trial ---  
such as each patient's \mAb{} dosage level, 
prior Covid-19 risk factors, or viral-load change 
over time --- also belong in the trial's unique 
data models.  Moreover, a comprehensive investigation could well incorporate both 
information about the patient's unique immunological profiles and the 
nature of the SARS-CoV-2 variant/strain found in the patient.}

\p{In addition, customizing trial-management software 
has the added potential benefit of greater 
flexibility for incorporating personalized/patient-centered 
data into the overall trial results.  Since patients'  
reactions to interventions are difficult to anticipate in 
full specificity ahead of time (especially when subjective 
experience is taken into account) allowing data 
models to evolve during the course of a trial 
can help trial designs respect the experiential 
dimensions intrinsic to patient-centered 
paradigms of care.}

\subsubsection{Toward Fine-Grained Sociodemographic Models}
\p{Patient-centric data could likewise include sociodemographic information 
about the patient, supplemented by epidemiological 
metrics, such as contact tracing.  Designers need to identify, 
for example, what dimensions of patients' immunological 
and sociodemographic profiles are likely to be 
consequential when analyzing treatment outcomes 
(indeed, biomedical research has been criticized 
in recent years for bias
toward certain populations, e.g., white, middle-class non-seniors).  
This results in
uncertainties as to how well trial results carry over to populations at large --- 
that is, populations characterized by a heterogeneous mix of demographic factors. 
Researchers can mitigate these concerns by demonstrating 
sociodemographic diversity among trial participants. 
Those goals, along with more precise predictive 
analytics, could be advanced by adopting 
more detailed sociodemographic reporting 
standards \cite{AaronMOrkin}, \cite{MarittKirst}, 
\cite{ShelbyMeyer}, etc.}

\p{Demonstrating sociodemographic 
diversity, however, 
calls for transparency about how sociodemographic 
details are represented.  The process of grouping 
patients into ethnic/racial and/or socioeconomic 
strata can be equivocal at times.  For example, if a trial 
participant is a graduate student at the University of 
Chicago, should their socioeconomic status be assessed 
on the basis of their own income or that of their parents?  
If their zip code places them on 
the South Side of Chicago, a region with both a prestigious 
campus and pockets of extreme poverty, should they be 
demographically classified alongside residents 
of that neighborhood?  What about a varsity linebacker 
who appears to be in excellent health before a Covid-19 infection?  
Should his status as an athlete be taken to indicate 
being extremely fit prior to the disease, or might 
his background as a football player intimate a potential history of 
brain trauma which may compound his neurological damage due to 
Covid-19?}

\p{In short, sociodemographic data can
be notoriously imprecise. It is well-known that patients of lower socioeconomic status have higher Covid-19 infection rates and mortality rates than 
patients of higher socioeconomic status,
but this disparity may
be explained by several causal factors: more infectious workplaces, inferior post-infection treatment, poorer state of health before the onset of Covid, and so forth.  
Teasing apart these factors demands fine-grained
analysis of the individual patient's 
pre-Covid history; sociodemographic generalizations can exclude
important details. For example, for purposes of analysis, in the case of a graduate student with
middle-class parents but no health insurance of their own, how should we quantify their degree
of access to health care? How well does geographic location serve as a proxy for socioeconomic
status?}

\p{The case of a middle-class student living in a well-off near-campus corner of an otherwise
impoverished neighborhood suggests that geospatial metrics (such as zip code or congressional
district) are imperfect proxies for wealth; but other factors --- such as air/water pollution or the
risk of being the victim of a crime --- may be statistically correlated among geographically proximate
residents even if they have otherwise divergent sociological profiles.
These examples illustrate that the value of sociodemographic data is proportionate to the level
of statistical detail with which the data may be analyzed.  Instead of a broad and vague designation, such as \q{low income,} one may want 
to derive more detailed subclasses, incorporating
information about patients' employment, access to health care, physical fitness, and so forth. Two
patients with similar income levels, for instance, may have different 
levels of access to health care 
(depending
on factors such as whether the patients have employer-based insurance or geographic proximity
to healthcare facilities) or different levels of exposure to 
Covid-19 in the workplace, depending on the nature of their job.
Even if a trial does not quantify 
granular sociodemographic assessments --- such as patients' 
workplace conditions and access to health care, which might 
serve as a more accurate  
estimation of how socioeconomic status causally affects 
disease outcomes --- subsequent researchers may determine 
ways to analyze or add on to data generated by a trial (e.g., 
through follow-up studies of enrolled patients), 
so as to make such sociodemographic 
granularity part of the quantifiable framework.}

\subsubsection{Measuring Cognitive and Neurological Effects}  
\p{Similar interpretive issues of interpretation also apply to post-treatment 
observations.  How should researchers decide which observations 
qualify as clinically significant 
consequences of Covid-19?  
We have seen that as the pandemic has unfolded, 
a fair number of cases have been cited in the professional literature 
describing Covid-19 patients who suffer certain
cognitive/neurological effects, such as muscular fatigue or weakness, 
mental confusion or poor concentration (sometimes referred to as \q{brain fog}), 
or symptoms of Guillain–Barr\'{e} syndrome spectrum. 
Almost certainly, some 
of these symptoms may be the product of cognitive/neurological effects due 
to SARS-CoV-2.  
At the same time, Covid-19 patients --- 
even those who fight off the infection 
successfully or who test positive but remain asymptomatic
--- may find their lives so disrupted by the pandemic 
that this may (indirectly) cause cognitive and neurological problems.
For instance, prolonged inactivity 
(for a typically active person), which commonly occurs as the result of  
a quarantine, may contribute to poor concentration and other 
diminished forms of mental acuity \cite[say]{SoniaDifrancesco}.  
Given the fact that most 
people's lives during the pandemic are not \q{normal,} 
it may be difficult to establish 
which symptoms experienced by a patient are actually biological effects of
the disease itself or, alternatively, 
indirect consequences of  
lifestyle restrictions.  
This sort of ambiguity also applies to potential adverse 
side-effects of Covid-19 treatment.  
Rigorous design practice suggests, for instance, 
that parameters should be established framing the post-treatment 
and post-recovery window of time where patients' symptoms might be noted as 
potential effects of the disease or of the administered therapies 
themselves.  How long after
a treatment is administered should a patient's 
symptoms be considered potential side-effects of
the therapy itself or, alternatively, the result of 
nagging uncertainties and a 
disrupted lifestyle imposed by the pandemic?}

\p{The fact that these questions have no predetermined answers indicates 
that trial designs need to anticipate a shifting clinical and 
information landscape ad the trial evolves.  
For example, because SARS-CoV-2 was initially believed to  
affect lung functioning primarily, the risk of long-term 
cognitive/neurological damage was not widely anticipated when 
considering treatment over the course of Covid-19 infection.  
Consequently, because tests of a cognitive/neurological/physiological/radiographic 
(except for basic lung scans, with respect to radiology) 
had not been a common facet of early Covid-19 trials/observational 
studies, there were no 
corresponding data structures included in those studies for capturing this full spectrum of neuropsychological and neurological data.}

\p{Systematically tracking Covid-19's 
cognitive/neurological deficits entails neurological, laboratory, neuropsychological and radiographic tests to understand the full extent of their cognitive and neurological impairments. For example, the use of
\MRI{}s when considering Covid-related ischemic  
stroke \cite{RossWPaterson}, \cite{FifiMocco}, 
the use of electrophysiological tests, cerebrospinal 
fluid tests (\CSF{}), or the \MRC{} 
(Medical Research Council) Scale for muscle-strength test when considering 
Covid-related 
Guillain-Barr\'{e} symptoms \cite{SamirAbuRumeileh}, 
or the use of neuropsychological tests, such as Trail Making Test 
(\TMT{}), Sign Coding Test (\SCT{}), 
Continuous Performance Test (\CPT{}), and Digital Span Test 
(\DST{}) --- which measure a patient's executive abilities (letter and number recognition mental flexibility, visual scanning, and motor function) and sustained and selective attention, along with other cognitive and neurological functioning --- when considering 
cognitive/neuropsychological impairments after a serious bout with 
Covid-19 \cite{HetongZhou}.}

\p{These cognitive, neurological, physiological, laboratory, 
and radiographic data 
structures thereby become 
an integral part of the information relevant to trial evaluation, 
because they document symptoms 
which are presumptively attributable to Covid-19.  
However, in practice, prior to clinicians having been alerted to 
the fact that Covid-19 
may cause lingering cognitive/neuro\-logical damage, %physiological or neurological data 
Covid-19 trials were not designed to 
incorporate neurological or cognitive 
data in a systematic manner.  This scenario points to how 
trial data models can benefit from 
a built-in capacity to be redesigned while the 
trial is ongoing, so as to accommodate new and emerging 
information.  On this basis it is reasonable to 
adopt for clinical trials malleable information models such as 
graph databases and/or Object-Oriented software 
components.}

\subsubsection{Aggregating Trial Data via Graph Models}
\p{A useful data-integration case-study is the University of Pennsylvania's Carnival project (which achieves 
data integration by adopting property-graph databases, illustrating 
the flexibility of graph models in a way that we can 
also apply to trial design).  Carnival synthesizes heterogeneous biomedical data  
by translating information from disparate sources into a common 
property-graph representation and then querying this 
data with the Gremlin Virtual Machine.  Gremlin is a 
\q{step-based} virtual machine where \q{steps} between 
potential focus elements in a property graph play the role 
of primitive processing instructions; querying and traversing 
property graphs involves executing a series of Gremlin 
steps.
%\footnote{The theoretical foundations of step-based Virtual Machines 
%are presented in Marko Rodriguez, 
%``Stream Ring Theory,'' February 14, 2019 (\bhref{https://zenodo.org/record/2565243#.X3vzqS4pDeQ}).} 
Most Gremlin implementations are based on the Java programming language and 
the Java Virtual Machine 
(\JVM{}), so that queries 
themselves are written in a \JVM{} language (Groovy, in the 
case of Carnival).  The challenge 
for any database engine which employs a relatively complex data-representation 
strategy --- such as a hypergraph, property-graph, tuple-store (a 
collection of records with varying numbers of fields) or a multi-dimensional 
(possibly sparse) array --- is to efficiently map 
the high-level data structures manipulated within the database itself 
to the lower-level memory units which are stored to disk.  Lower-level 
data structures are typically modeled via simpler database constructions 
such as key-value stores, memory cells, or relational tables, 
so there must be a translation pipeline between high-level 
structures (properties, hypernodes, hyperedges, and so forth) and low-level   
points (record cells, shared memory address, elements in key-value pairs, etc.)}

\p{Consider how patient profiles usually notate the 
medications each patient is taking --- any patient (a node) could in principle be 
connected to any medication (another node); some patients 
may be taking \textit{no} medications, others may take just \textit{one}, and some may  
take \textit{two or more}.  Also, connections between patients and medications 
can be the basis of further details that emerge over time 
(and are registered in the graph) inasmuch as medications are prescribed 
to patients by a specific doctor at a specific time, in a specific 
dosage, in response to specific diagnostic tests, and so forth.  
In short, information can \q{fan out} from the patient-to-medication 
connection in a relatively free-form manner.  In general, then, as a 
subset of overall patient profiles, information about 
medication evinces the structural features which are, in many 
contexts, optimally represented via free-form labeled graphs.    
Meanwhile, patient profiles may also 
consider medical history, which can be modeled as a graph with detailed 
logical and temporal inter-node connections.  According to this 
representational strategy, each patient's history is a series of 
events and observations which are temporally ordered --- it is 
possible to query or traverse the graph in a manner which takes 
before/after relations 
into account --- and where there are also logical or causal connections 
defined between nodes.  For instance, an edge might assert that 
a given medication was prescribed to a patient (an event) \textit{because of} 
the results of a given lab test (an observation).  In these examples, different sorts of clinical data --- sociodemographic, 
pharmacological, medical-history --- are modeled according to different 
sorts of graph structures (hypernodes, nonschematic labeled edges, temporalized 
graphs, and so forth).}

\p{Insofar as different formations within data-structures 
tend to be conveyed via different graph-database constructions, 
the partition of graph-database technology itself 
into competing (partially incompatible) systems 
(distinct architectures, query languages, query-evaluation 
strategies, and so on) can also be a hindrance 
to data-integration, even after adopting flexible 
graph-database technologies.  This is one motivation 
for the hybrid graph models we discuss in later 
chapters, which attempt to recognize numerous 
structuring elements endemic to different flavors 
of graph systems, accommodated into a single hypergraph-based 
query framework.}

\subsection{Representing Trial Data via Object Models}
\p{As a concrete example of open-ended data-integration 
concerns being anticipated as a central element of 
trial design, consider how Object-Oriented models 
for Covid Phylogeny (which we briefly considered 
last chapter) could serve as a nexus for integrating SARS-CoV-2 phylogenetic 
data across multiple studies and healthcare systems.  
Such an Object Model may be extended in different 
ways for different clinical trials examining a range 
of Covid-19 treatments.}

\p{In the context of antibody regimens, 
for instance, scientists need 
to quantify how well the antibodies disable 
Covid-19 spike proteins directly and/or 
how well the antibodies block 
the virus's ability to attach to human cells.  These measurements 
generate data which gauge a patient's 
immune response to Covid-19, 
whether innate (\q{naive}) or boosted by the 
administered antibodies.  Researchers 
have mined such data from different angles, 
including contrasting symptoms presented 
with different levels of severity \cite{ZacharyMontague}, 
\cite{YonggangZhou}, biochemically describing 
the mechanisms of immune response so as 
to augment or simulate that response via 
monoclonal antibodies or similar antiviral 
therapies \cite{PingpingWang}, \cite{YaoQingChen}, 
identifying 
factors explaining act-risk groups' 
greater susceptibility to severe cases 
\cite{LisaPaschold}, or comparisons of SARS-CoV-2 against 
other coronaviruses \cite{TeresaAydillo} 
(these citations are representative examples 
of work conducted by many research groups; 
they could be expanded with similar references 
we included in Chapter 2 when discussing 
immune repertoires, for example).   
  %\cite{}, 
  %\cite{PingpingWang}, 
  %\cite{TeresaAydillo}, .
A Covid-19 software ecosystem 
should then ensure that such immune-response 
data can be effectively parsed and integrated into Object Models 
describing how SARS-CoV-2 is evolving around the 
globe, so researchers (as much as logistically feasible) 
could track each patients' immune response in light of their particular acquired 
SARS-CoV-2 strain/variant and their personal immunological 
profile.  The relevant information for geospatial and 
sociodemographically tagged studies would then include 
patients' prior immuno-profiling and risk assessment, 
immune response (naive or affected by treatment) and 
clinical outcomes, as well as genetic 
tests for the viral variant.  Cross-sectionally 
mapping such data geographically and along 
sociodemographic/socioeconomic contours could 
provide a holistic picture of the pandemic 
at a global scale.}  

%, and to competently track both current and 
%emerging SARS-CoV-2 strains throughout the population.}


\p{Object Models intended to facilitate such 
globally-scoped data integration could then facilitate trials designed 
according to the protocol proposed by Shrestha \textit{et al.}, 
discussed above, where each trial would study a 
preselected (non-random) cohort of patients 
for whom both pre-treatment immunoprofiling data and post-treatment 
outcomes data would be available, so as to compare multiple trials 
against one another.  Object Models customized for each trial 
would generate a data framework through which 
the causal relations between patient profiles and treatment 
outcomes would be investigated.  Customized trial software 
would, accordingly, provide a Reference Implementation demonstrating 
each trial-specific Object Model.  If adopted for multiple trials conducted 
across multiple clinics/hospitals, trials' 
data models may help doctors better understand which 
aspects of patient profiles are particularly 
significant when matching Covid-19 patients to the most 
salutary treatment.}  


\begin{figure}[t!]

\caption{Checkerboard Pattern as a (Very Simple) Texture}	

\begin{center}

\hspace*{-1em}\begin{tikzpicture}[tips=proper]

\tikzstyle{every node}=[font=\small]

\colorlet{blbl}{blue!70!black}
\colorlet{blblor}{blbl!70!orange}

\colorlet{gror}{green!40!orange}


\node (dummy) at (0,0) {};

\colorlet{blor}{black!80!orange}

%\draw (0,0) rectangle (4,4);
\fill (-.25,.25) rectangle (-.4,.4);
\fill (.25,.25) rectangle (.75,.4);
\fill (-.25,-.25) rectangle (-.4,-.75);


\node (b11) at (0,0) [fill=black,inner sep=0,minimum width=5mm,minimum height=5mm] {};
\node (b22) at (.5,-.5) [fill=black,inner sep=0,minimum width=5mm,minimum height=5mm] {};


\colorlet{bbb}{blbl!80!brown}
\node (cblabel) at (-1,-1.4) [align=right,inner sep = 4, rounded corners = 5,
fill=bbb!20!black,fill opacity=.8,text=black,
text opacity=1] {\colorbox{white}{Checkerboard Pattern}};



\node (circ) [circle, very thick, double,
draw=red!68!orange, minimum width=6mm] at (.25,-.25){};

\draw ([yshift=-.5mm,xshift=2mm]
 circ.north east) [draw=red!68!orange, >=stealth', 
 line width=1.5mm,
 bend right=20, ->] edge (2.2,1.1) ;

\draw ([yshift=-.5mm,xshift=2mm]
 circ.north east) [draw=white,% >=stealth', 
 line width=.4mm,
 bend right=19,  shorten >=4mm] edge (2.2,1.1) ;


\fill (2.25,1.25) rectangle (2.4,1.4);
\fill (2.4,1.25) rectangle (2.55,1.1);

\draw [black!50] (2.27,1.12) rectangle (2.53,1.38);

\draw (2.4,.6) --  (2.4,1.9) [draw=gray];
\draw (1.4,1.25) --  (3.4,1.25) [draw=gray];


\draw [dashed,color=blblor!40!purple] (2.3,1.35) -- (1.3,1.35);
\draw [dashed,color=blblor!40!purple] (2.3,1.35) -- (2.3,2.15);

%\draw [dashed,color=blblor!40!purple] (2.14,1.48) -- (1.36,1.48);
%\draw [dashed,color=blblor!40!purple] (2.14,1.48) -- (2.14,2.08);

\draw [dashed,color=blblor!40!purple] (2.14,1.48) -- (1.36,2.08);

\draw [dashed,color=blblor!40!purple] (2.5,1.15) -- (3.5,1.15);
\draw [dashed,color=blblor!40!purple] (2.5,1.15) -- (2.5,.45);

\draw [dashed,color=blblor!40!purple] (2.64,1.01) -- (3.32,.41);


\node (ptl) at (.8,1.9) [fill=brown!20] {Pixel Top-Left};
\node (ptr) at (3.8,1.9) [fill=brown!20] {Pixel Top-Right};

\node (pbl) at (.2,.9) [fill=brown!20] {Pixel Bottom-Left};
\node (pbr) at (4.4,.7) [fill=brown!20] {Pixel Bottom-Right};

\node (amlabel) at (.2,3.2) [align=right] {Color-Change (Gradient) Matrix 
(for the \\top-right pixel):\hspace{-5pt} };
\node (am) at (4.5,3) {%
\begin{minipage}{2cm}
$\begin{pmatrix}
1 & 0 & 0 \\
1 & 0 & 0 \\
0 & 1 & 1
\end{pmatrix}$
\end{minipage}
 };


\node (back) [circle, fill=red!38!brown, 
minimum width=3cm, yscale=.6] at (3,-.6){};

\node (elllabel) at (4,-1.6) [align=right,
fill=red!38!brown,fill opacity=.8,text=black,
text opacity=1] {\colorbox{white}{Diagram of 14 Pixel Types}};



\fill [color=cyan!10!brown, draw=white] (3,0) rectangle (3.1,-.5);
\fill [color=orange!40, draw=white] (3.1,0) rectangle (3.2,-.5);

\fill [color=cyan!10!green, draw=white] (3,-.5) rectangle (3.1,-.6);
\fill [color=cyan!10!red, draw=white] (3.1,-.5) rectangle (3.2,-.6);
\fill [color=purple!60, draw=white] (3,-.6) rectangle (3.1,-.7);
\fill [color=cyan!gror, draw=white] (3.1,-.6) rectangle (3.2,-.7);

\colorlet{cg}{cyan!10!green}
\fill [color=cg!30, draw=white] (3,-.5) rectangle (2,-.6);

\fill [color=yellow!70!blblor, draw=white] (3,-.6) rectangle (2,-.7);

\fill [color=pink!10!gror, draw=white] (3.2,-.5) rectangle (4,-.6);

\fill [color=purple!60!gror, draw=white] (3.2,-.6) rectangle (4,-.7);


\fill [color=cyan!70!blblor, draw=white] (3,-.7) rectangle (3.1,-1.2);
\fill [color=pink!90!gror, draw=white] (3.1,-.7) rectangle (3.2,-1.2);

\colorlet{bry}{brown!20!yellow}
\fill [color=gray!20, draw=white] (3,-.7) rectangle (2,-1.2);
\fill [color=bry!30, draw=white] (3.2,-.7) rectangle (4,-1.2);

\fill [color=bry!30, draw=white] (3,-.5) rectangle (2,0);
\fill [color=gray!20, draw=white] (3.2,-.5) rectangle (4,0);

%\fill [color=brown!40] (3.1,-.7) rectangle (3.2,-1.2);



\node (longtext) at (1.5,-9.7) 
[draw=bry!50!purple,double,rounded corners=30, 
inner sep=10pt] {%
\begin{minipage}{8.5cm}
{\fontsize{9}{11}\fontfamily{phv}\fontseries{m}\selectfont%
All pixels in this image can be sorted into 
14 groups: white pixels surrounded entirely 
by white; black pixels surrounded by black; 
white (respectively black) pixels on 
horizontal or vertical edges between 
black and white edges; and pixels 
at black/white corners.  Each of these 
groups can be associated with a distinct 
gradient matrix showing jumps in color 
values along eight directions relative to 
each individual pixel (the diagram above shows 
some of these groups by assigning them 
distinct colors, though the 
actual pixels are only black/white).

\vspace{6pt}
For real-life images, checkerboard patterns would 
presumably only be exact, so the gradient matrices 
would have entries other than perfect 1s and 0s.  
However, each matrix could still be clustered 
into one of the 14 canonical matrices for the 
pixel groups, allowing the pixels to be 
classified, and same-group pixels should be 
arranged according to a pattern matching 
that diagrammed above.  Patterns of 
groups in the neighborhood of individual 
points can similarly be gathered into matrices, e.g.: 

\vspace{-2pt}
\begin{center}
\begin{minipage}{2cm}
$\begin{bmatrix}
Group 1 & Group 2 & Group 3 \\
Group 1 & (center) & Group 3 \\
Group 1 & Group 2 & Group 3
\end{bmatrix}$
\end{minipage}
\end{center}

\vspace{2pt}
The group-number of the center for each such 
matrix would predict the surrounding groups, 
and a secondary matrix could check 
whether these predictions are accurate 
(using \q{1} to mean a correct group, 
and \q{0} an unexpected one): 

\begin{center}
\begin{minipage}{2cm}
$\begin{Bmatrix}
1 & 1 & 1 \\
1 & (center) & 1 \\
1 & 1 & 1
\end{Bmatrix}$
\end{minipage}
\end{center}

The closer these matrices are to having 
all 1s, adding 
over all points in the image, measures 
how closely the image approximates a 
perfect checkerboard.

\vspace{6pt}
This may not be the most efficient algorithm 
to perform such an analysis; we 
describe it here simply as an illustration 
of techniques used for detecting more 
complex textures.

 

}
\end{minipage}
};




%\fill (-.25,-.25) rectangle (-.4,-.75);




\end{tikzpicture}
\end{center}


\label{fig:checkerboard}
\end{figure}


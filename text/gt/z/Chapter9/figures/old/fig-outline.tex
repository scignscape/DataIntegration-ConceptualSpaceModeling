
%\colorlet{footnotecolor}{red!40!black}

\makeatletter
\let\ofnt\@makefntext

\renewcommand\@makefntext[1]{%
\parindent 1em\noindent
\hb@xt@.7em{%
%\hss
\@textsuperscript{\normalfont\textcolor{purple!60!red}{\@thefnmark}}}#1}
\makeatother

\begin{figure}[t!]

\caption{Diagram of Annotation-Associated Datatypes}	

\edef\savedfootnotenumber{\number\value{footnote}}

\let\oldthefootnote=\thefootnote

\renewcommand{\thefootnote}{\textbf{\color{purple!60!red}{\alph{footnote}}}}


\setcounter{footnote}{0}

\begin{center}

\hspace*{-1em}\begin{tikzpicture}[tips=proper]

\tikzstyle{every node}=[font=\small]

\colorlet{blbl}{blue!70!black}
\colorlet{blblor}{blbl!70!orange}

\colorlet{gror}{green!40!orange}


\node (dummy) at (0,0) {};

\colorlet{blor}{black!80!orange}

\node (ann) at (3,-3) [draw, align=right,inner sep = 5] {Annotation};

\node (anns)  [above=6mm of ann, draw, align=right,inner sep = 5] {Annotation Set\footnotemark{}};

\node (anne)  [above=6mm of anns, draw, align=right,inner sep = 5] {Annotation Environment};

\node (gi)  [right=6mm of ann, draw, align=right,inner sep = 5] {Ground Image};
\node (gimd)  [below=6mm of gi, draw, align=right,inner sep = 5] {Image Metadata};




\node (shape)  [left=6mm of ann, draw, align=right,inner sep = 5] {Annotation Shape};

\node (pointset)  [below=6mm of shape, draw, align=right,inner sep = 5] {Point Set};
\node (lengthset)  [below=6mm of pointset, draw, align=right,inner sep = 5] {Length Set\footnotemark{}};

\node (spec)  [below=6mm of lengthset, draw, align=right,inner sep = 5] {Special Curve};


\node (visual)  [below=12mm of spec, draw, align=right,inner sep = 5] {Visual/\\Presentation};
\node (aref)  [below=6mm of visual, draw, align=center,inner sep = 5] {Annnotation\\Reference};



\node (isequence)  [below=6mm of gimd, draw, align=right,inner sep = 5] 
{Image Sequence\footnotemark{}};
\node (iseries)  [below=6mm of isequence, draw, align=right,inner sep = 5] 
{Image Series\footnotemark{}};

\node (ari)  [below=12mm of iseries, draw, align=right,inner sep = 5] 
{Annotation\\Region Image\footnotemark{}};



\end{tikzpicture}

\setcounter{footnote}{\savedfootnotenumber}}

\end{center}


\label{fig:outline}
\end{figure}

\begingroup
  \renewcommand*{\thefootnote}{\alph{footnote}}
\footnotetext[1]{Annotation Sets could potentially contain other annotation sets.}
\footnotetext[2]{We use \q{length} set to denote the set of any magnitudes 
needed to geometrically specify the desired annotation-shape.}
\footnotetext[3]{Inage \q{Sequence} referring to transformed versions of a ground 
image (after grayscaling, morphological operators, or related modifications 
intended to make the image more amenable to analysis).}
\footnotetext[4]{Inage \i{Sequence} referring to logically related ground images.}
\footnotetext[4]{\q{Annotation Region Image} refers to a mask-image 
(e.g. a binary or grayscale image) intended to demarcated 
the contours of an annotation shape.}
\endgroup


\makeatletter
\let\@makefntext\ofnt
\makeatother




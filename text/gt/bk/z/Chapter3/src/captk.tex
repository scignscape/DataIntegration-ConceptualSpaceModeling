






\p{However, making interop work across distinct software 
ecosystems add development complexity: the 
requirements for implementing novel analytic 
methods are not only to compose executable 
code making the methods computationally realizable, 
but to package that code into a functional unit that 
can interoperate with other clinical and imaging 
software.  This problem, in turn, engenders 
various software-engineering techniques and frameworks.  
A good example is the Cancer Imaging 
Phenomics Toolkit (\CaPTk{}), developed 
at the Center for Biomedical Image Computing %
and Analytics (\CBICA{}) at 
the University of Pennsylvania Perelman School of 
Medicine, which is an extensible platform for implementing 
analytic modules as peers to a central \PACS{} system 
(a detailed discussion of \CaPTk{} module implementional 
is outside the scope of this chapter, but 
this book's suppelemental material include a 
more technical overview of \CaPTk{}; for now 
we just use \CaPTk{} as a case-study 
in modular design for bioimaging).   
In particular, \CaPTk{} provides an implementation 
(apparently the only \Cpp{}-based implementation) 
of \CWL{}, using this workflow model in 
conjunction with the \Qt{} Reflective Programming 
system to implement workflows connecting the 
central \CaPTk{} application with its analytic 
extensions.\footnote{No native \sC{} or \sCpp{} 
libraries are described on the \sCWL{} website 
among the tools and parsers available for \sCWL{}, 
but \sCaPTk{} is mentioned on a corresponding discussion 
thread concerning \sCpp{} libraries.  It seems 
therefore that the \sCaPTk{} \q{utilities} repository 
provides the de-facto standard \sCpp{} implementation of 
\sCWL{}, at least according to the \sCWL{} group 
themselves.}   In effect, \CaPTk{} achieves a 
workflow and messaging protocol for what they 
term \q{native,} \q{standalone} applications, 
yielding an extensible architecture through 
which new image-analysis techniques can be 
integrated into an underlying \PACS{} system.} 
 
\p{Certain comparisons can be made 
between \CaPTk{}, whose architectural 
innovations are centered on workflow 
management and multi-application networking, 
and \SeDI{}, whose novel features 
focus on data alignment and integration.  
Both of these projects expand the analytic 
capabilities of diagnostic-imaging 
systems by promoting common data and 
code representations, enlarging the 
space of metadata available for query-evaluation 
and/or the range of quantitative techniques 
available for image analysis.  Both rely 
on a canonical description format 
(\CWL{}, in the case of \CaPTk{}, and 
a novel \RDF{} ontology, in the case of \SeDI{}) 
which is not widely implemented by other 
\PACS{} systems.  Their concerns also overlap 
insofar as different analytic methods generate 
different kinds of image data, which need to be 
integrated into the total space of data 
available for a \PACS{} system and/or an image 
repository.  While the data-integration 
approaches chosen by these two projects are 
specific to the respective software applications which 
is their main result, both \SeDI{} and \CaPTk{} 
point to evident limitations in the scope 
of current diagnostic imaging software: 
failure to properly integrate image metadata 
(including clinical and outcomes data) into a 
multi-patient space optimized for query 
evaluation and data mining; and failure to 
integrate many diverse image-analysis 
methodologies into a common execution framework.}

\p{One take-away from this overview of 
\SeDI{} and \CaPTk{} is that new diagnostic 
imaging software can incorporate some version 
of the data models and protocols implemented 
by these two projects.  On a broader level, however, 
the concrete examples of \SeDI{} and \CaPTk{} 
point to limitations in current 
frameworks such as \DICOM{}.  The integrative logic of 
\SeDI{} and \CaPTk{} is based on specific 
data structures --- \DICOM{} headers and 
\CWL{}, respectively --- and arose out of 
practical limitations in existing \DICOM{} 
software.  Integration problems are not exhausted by 
deriving solutions in one specific area: 
for example, merging \DICOM{} metadata 
into \RDF{} graphs may successfully align 
data structures conforming to current 
\DICOM{} specifications, but does not 
guarantee integration of novel extensions or 
supplements to \DICOM{}.  Much as \DICOMRT{} 
extended \DICOM{} to incorporate radiation-therapy 
recommendations, predictive modeling and  
patient-centered Comparative Effectiveness reach 
could easily lead to new data standards as 
researchers seek to integrate imaging data 
with treatment plans and outcomes evaluation.  
Because \CaPTk{} is extensible as part of 
its essential design, this specific 
applications can serve as a useful 
case-study for the analytic convergence 
or cross-referencing between image-analysis 
and outcomes/patient-centered data, 
although in terms of large-scale adoption 
more conventional \PACS{} clients may also 
be used simply because \CaPTk{} has certain 
software-engineering innovations which 
make it an outlier from an 
implementation point of view.}















\p{Simultaneously, the science of diagnostic imaging 
is also expanding as new image-analytic techniques 
prove to be effective at detecting signals 
within image data, often complementing the work 
of human radiologists.  The proliferation of 
image-analysis methodologies places a new emphasis 
on \textit{extensibility,} where radiological 
software becomes more powerful and flexible 
because new analytic modules may be plugged in 
to a central \PACS{} system.  A representative example of 
this new paradigm is \CaPTk{} (Cancer Imaging  
Phenomics Toolkit), which we 
will discuss below.  The \CaPTk{} project
provides a central application 
which supplies a centralized User Interface and 
takes responsibility for acquiring and loading 
radiographic images.  The \CaPTk{} core application is 
then paired with multiple \q{peer} applications 
which can be launched from \CaPTk{}'s main window, 
each peer focused on implementing specific 
algorithms so as to transform and/or to extract feature vectors 
from images sent between \CaPTk{} and its plugins.}

\p{Both the patient-outcomes focus in building image 
repositories and the integration of novel 
Computer Vision algorithms depend, at their core, 
on rigorous data sharing.  Taking the 
\RSNA{} Covid-19 repository as a case study for 
promoting research into post-diagnostic outcomes, 
this repository is possible because an international 
team of hospitals and institutions have agreed to 
pool radiological data relevant to SARS-CoV-2 infection 
according to a common protocol.  Taking \CaPTk{} as a 
case-study in multi-modal image analysis, this 
system is likewise possible because analytic modules 
can be wrapped into a plugin mechanism which allows 
many different algorithms to be bundled into a common 
software platform.  Of course, these two areas 
of data-sharing overlap: one mission of repositories 
such as the \RSNA{}'s is to permit many different 
analyses to be performed on the common image 
assets.  The results of these analyses then become 
additional information which enlarges the 
repository proportionately.  If \CaPTk{} modules 
are used to analyze the \RSNA{} Covid-19 images, 
for example, there has to be a mechanism for 
exporting the resulting data outside the \CaPTk{} 
system, so that the analyses may be integrated into 
the repository either directly or as a supplemental 
resource.}
  
\p{This example demonstrates how software such as 
\CaPTk{} may be extended to support the curation 
of image repositories dedicated to Patient Outcomes 
and Comparative Effectiveness Research (\CER{}), insofar 
as analytic data generated by (for example) \CaPTk{} components 
can acquire the capability to share data according 
to repository protocols.  A further level of 
integration between \CaPTk{} and \CER{} initiatives 
(again, staying with \CaPTk{} as a 
representative example of bioimaging applications in general) 
can be achieved if one observes that clinical 
outcomes may be part of the analytic parameters 
used by \CaPTk{} modules.  As presently constituted, 
\CaPTk{} analytic tools are focused on extracting 
quantitative (or quantifiable) features from 
image themselves, without considering additional 
patient-centered context.  There is no technical 
limitation, however, which would prevent the 
\CaPTk{} system from sharing more detailed clinical 
information with its modules, allowing these 
analytic components to cross-reference image features 
with clinical or patient information.  This 
then raises general questions about sharing 
clinical data \i{as well as} information 
derived from bioimage analysis, which we 
will review over the course of this chapter.}



%\colorlet{footnotecolor}{red!40!black}

\makeatletter
\let\ofnt\@makefntext

\renewcommand\@makefntext[1]{%
\parindent 1em\noindent
\hb@xt@.7em{%
%\hss
\@textsuperscript{\normalfont\textcolor{purple!60!red}{\@thefnmark}}}#1}
\makeatother

\begin{figure}[t]

\caption{Diagram of Annotation-Associated Datatypes}	

\edef\savedfootnotenumber{\number\value{footnote}}

\let\oldthefootnote=\thefootnote

\renewcommand{\thefootnote}{\textbf{\color{purple!60!red}{\alph{footnote}}}}


\setcounter{footnote}{0}

\begin{center}

\hspace*{-1em}\begin{tikzpicture}[tips=proper]

\tikzstyle{every node}=[font=\small]

\colorlet{blbl}{blue!70!black}
\colorlet{blblor}{blbl!70!orange}

\colorlet{gror}{green!40!orange}


\node (dummy) at (0,0) {};

\colorlet{blor}{black!80!orange}

\node (ann) at (3,-3) [draw, align=right,inner sep = 5] {Annotation};

\node (anns)  [above=6mm of ann, draw, align=right,inner sep = 5] {Annotation Set\footnotemark{}};

\draw (ann) [shorten <= 1mm,shorten >= 1mm] -- (anns);

\node (anne)  [above=6mm of anns, draw, align=right,inner sep = 5] {Annotation Environment};

\draw (anne) [shorten <= 1mm,shorten >= 1mm] -- (anns);

\node (meta) at (2,0.5) [draw, align=left,inner sep = 5] {Metadata 
(annotation about annotation(s))};

\draw ([xshift=1mm]ann.north west) [shorten <= 1mm,shorten >= 1mm] -- 
  ([xshift=-3mm,yshift=4mm,]ann.north west) -- 
  ([xshift=-27mm,yshift=4mm,]ann.north west) -- 
  ([xshift=-27mm]ann.north west |- meta.south);

\draw (anns.west) [shorten <= 1mm,shorten >= 1mm] -- 
  ([xshift=-20mm]anns.west) -- 
  ([xshift=-20mm]anns.west |- meta.south);


\draw (anne.west) [shorten <= 1mm,shorten >= 1mm] -- 
  ([xshift=-10.2mm]anne.west) -- 
  ([xshift=-10.2mm]anne.west |- meta.south);


\node (prov) at (6.2,-1.5) [draw, align=left,inner sep = 5] {Provenance};

\draw (ann.north east) [shorten <= 1mm,shorten >= 1mm] -- (prov.south west);

\draw (anns.east) [shorten <= 1mm,shorten >= 1mm] -- (prov.west);



\node (gi)  [right=6mm of ann, draw, align=right,inner sep = 5] {Ground Image};

\draw (ann) [shorten <= 1mm,shorten >= 1mm] -- (gi);

\node (gimd)  [below=4mm of gi, draw, align=right,inner sep = 5] {Image Metadata};

\draw (gimd) [shorten <= 1mm,shorten >= 1mm] -- (gi);



\node (shape)  [left=6mm of ann, draw, align=right,inner sep = 5] {Annotation Shape};

\draw (ann) [shorten <= 1mm,shorten >= 1mm] -- (shape);

\node (pointsetmid)  [below=6mm of shape, align=right] {};
\node (pointset)  [right=-3mm of pointsetmid, draw, align=right,inner sep = 5] {Point Set};

\draw (shape) [shorten <= 1mm,shorten >= 3mm] -- (pointsetmid);

\node (lengthsetmid)  [below=6mm of pointsetmid, align=right] {};

\node (lengthset)  [right=-8mm of lengthsetmid, draw, align=right,inner sep = 5] {Length Set\footnotemark{}};

\draw ([xshift=-4mm]shape.south)
 [shorten <= 1mm,shorten >= 3mm] -- ([xshift=-4mm,yshift=-14mm]shape.south);

\node (spec)  [below=6mm of lengthsetmid, draw, align=right,inner sep = 5] {Special Curve};

\draw ([xshift=-9mm]shape|-shape.south)
 [shorten <= 1mm,shorten >= 1mm] -- ([xshift=-9mm]shape|-spec.north);

\node (visual)  [below=6mm of spec, draw, align=right,inner sep = 5] {Visual/\\Presentation};

\draw ([xshift=-6mm]ann.south) [shorten <= 1mm,shorten >= 1mm] -- 
  ([xshift=-6mm]ann.south |- visual) -- (visual);

\node (aref)  [below=5mm of visual, draw, align=center,inner sep = 5] {Annnotation\\Reference};

\draw ([xshift=-3mm]ann.south) [shorten <= 1mm,shorten >= 1mm] -- 
  ([xshift=-3mm]ann.south |- aref) -- (aref);


\node (isequence)  [below=5mm of gimd, draw, align=right,inner sep = 5] 
{Image Sequence\footnotemark{}};

\draw ([xshift=1mm]gi.east) [shorten <= 1mm,shorten >= 1mm] -- 
  ([xshift=4mm]gi.east) -- 
  ([xshift=4mm]gi.east |- isequence.east) -- (isequence.east);



\node (iseries)  [below=5mm of isequence, draw, align=right,inner sep = 5] 
{Image Series\footnotemark{}};

\draw (isequence) [shorten <= 1mm,shorten >= 1mm] -- (iseries);

\node (text)  [below=6mm of iseries, draw, align=right,inner sep = 5] 
{Free Text/Codes\footnotemark{}};

\draw ([xshift=6mm]ann.south) [shorten <= 1mm,shorten >= 1mm] -- 
  ([xshift=6mm]ann.south |- text) -- (text);


\node (ari)  [below=6mm of text, draw, align=right,inner sep = 5] 
{Annotation Region Image\footnotemark{}};

\draw ([xshift=3mm]ann.south) [shorten <= 1mm,shorten >= 1mm] -- 
  ([xshift=3mm]ann.south |- ari) -- (ari);




\end{tikzpicture}

\setcounter{footnote}{\savedfootnotenumber}

\end{center}


\label{fig:outline}
\end{figure}

\begingroup
  \renewcommand*{\thefootnote}{\alph{footnote}}
\footnotetext[1]{Annotation Sets could potentially contain other annotation sets.}
\footnotetext[2]{We use \q{length} set to denote the set of any magnitudes 
needed to geometrically specify the desired annotation-shape.}
\footnotetext[3]{Inage \q{Sequence} referring to transformed versions of a ground 
image (after grayscaling, morphological operators, or related modifications 
intended to make the image more amenable to analysis).}
\footnotetext[4]{Inage \i{Sequence} referring to logically related ground images.}
\footnotetext[5]{Codes could be one or more diagnostic/prognostic 
terms selected from a Controlled Vocabulary or could be embedded 
as keywords in ordinary (free-form) text accompanying the annotation.}
\footnotetext[6]{\q{Annotation Region Image} refers to a mask-image 
(e.g. a binary or grayscale image) intended to demarcated 
the contours of an annotation shape.}
\endgroup


\makeatletter
\let\@makefntext\ofnt
\makeatother



